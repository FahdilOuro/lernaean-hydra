\documentclass{vldb}
\usepackage{verbatim}
\usepackage{graphicx}
\usepackage{multirow}
\usepackage{algorithm}
\PassOptionsToPackage{noend}{algpseudocode}
\usepackage{algpseudocode}
%\pagenumbering{arabic}
\pagenumbering{gobble}
%\definecolor{mygray}{rgb}{0.81, 0.81, 0.77}
\usepackage[labelfont=bf,textfont=bf]{caption}
\usepackage[labelfont=bf,textfont=bf,singlelinecheck=off,justification=raggedright]{subcaption}
\newcommand{\setHeight}{2cm}
\usepackage[normalem]{ulem}
\usepackage{longtable}
\usepackage{array}
\usepackage{url}
\usepackage{balance}
\usepackage{color}
\usepackage[table]{xcolor}
\usepackage{hhline}
\usepackage{paralist}
\usepackage{amsmath}
\let\proof\relax  
\let\endproof\relax
\usepackage{amsthm}
\usepackage{bm}
\usepackage[normalem]{ulem} 
\newtheorem{defn}{Definition}
\newtheorem{thm}{Theorem}[section]
\newtheorem{lem}[thm]{Lemma}


\graphicspath{../experiments/plots/paper/}


%\definecolor{myred}{RGB}{112, 31, 40}
%\definecolor{myblue}{RGB}{39, 70, 89}
%\definecolor{mygreen}{RGB}{104, 196, 37}
\definecolor{myred}{RGB}{189, 52, 67}
\definecolor{mygreen}{RGB}{19, 136, 8}
\definecolor{myblue}{RGB}{16, 52, 166}

\sloppy
% Define the switch for the different paper versions
% The switch Journal defines the journal version
% The switch Future defines the full draft with future work

\newif\ifJournal
\newif\ifFuture
\newif\ifGrey
\newif\ifColor

\Greytrue
%\Colortrue

%To activate the journal version, uncomment the line below
%\Journaltrue

%To activate the Future work version, uncomment the line below
%\Futuretrue

%To activate the Full draft  uncomment both switches

% By default the conference version is activated since
% switches are set to False by default.
\vldbTitle{Return of the Lernaean Hydra: Experimental Evaluation of Data Series Approximate Similarity Search}
\vldbAuthors{Karima Echihabi, Kostas Zoumpatianos, Themis Palpanas, and Houda Benbrahim}
\vldbDOI{https://doi.org/10.14778/3368289.3368303}
\vldbVolume{13}
\vldbNumber{3}
\vldbYear{2019}


\begin{document}
	%\title{\bf\Large Taming the Lernaean Hydra of Data Series Similarity Search:\\ An Experimental Evaluation of the State of the Art}
	\title{Return of the Lernaean Hydra: Experimental Evaluation of\\ Data Series Approximate Similarity Search}		%{\Large\bf[Experiments and Analysis Paper]}
	
	
	
	\numberofauthors{4}
	
	\author{
		% You can go ahead and credit any number of authors here,
		% e.g. one 'row of three' or two rows (consisting of one row of three
		% and a second row of one, two or three).
		%
		% The command \alignauthor (no curly braces needed) should
		% precede each author name, affiliation/snail-mail address and
		% e-mail address. Additionally, tag each line of
		% affiliation/address with \affaddr, and tag the
		% e-mail address with \email.
		%
		% 1st. author
		\alignauthor
%		Karima Echihabi \\
		\affaddr{
			Karima Echihabi\\
			IRDA, Rabat IT Center, \\
			ENSIAS, Mohammed V Univ.}
		%\email{{karima.echihabi@ifnilink.com}}\\
		\affaddr{{karima.echihabi@gmail.com}}\\
		% 2nd. author
		\alignauthor 
%		Kostas Zoumpatianos \\
		\affaddr{
			Kostas Zoumpatianos\\
			Harvard University} % MA, USA}
		\affaddr{{kostas@seas.harvard.edu}}
		% 3rd. author
		\alignauthor
%		Themis Palpanas \\
		\affaddr{
			Themis Palpanas\\
			Universit{\'e} de Paris}\\ % Paris, France}\\
		\affaddr{{themis@mi.parisdescartes.fr}}\\
		\and
		\alignauthor 
%		Houda Benbrahim \\
		\affaddr{
			Houda Benbrahim\\
			IRDA, Rabat IT Center, \\
			ENSIAS, Mohammed V Univ.} 
		\affaddr{{houda.benbrahim@um5.ac.ma}}
		% use '\and' if you need 'another row' of author names
	}
	% There's nothing stopping you putting the seventh, eighth, etc.
	% author on the opening page (as the 'third row') but we ask,
	% for aesthetic reasons that you place these 'additional authors'
	% in the \additional authors block, viz.
	% Just remember to make sure that the TOTAL number of authors
	% is the number that will appear on the first page PLUS the
	% number that will appear in the \additionalauthors section.
	
 \maketitle
	
\begin{abstract}
Data series are a special type of multidimensional data present in numerous domains, where similarity search is a key operation that
%A key operation in data series analysis pipelines is similarity search, which 
has been extensively studied in the data series literature. %, leading to the development of efficient exact indexing methods. 
In parallel, the multidimensional community %at large 
has studied approximate similarity search techniques. 
%In this paper, we 
We propose a %comprehensive 
taxonomy of similarity search techniques that reconciles the terminology used in these two domains, we describe modifications to data series indexing techniques enabling them to answer approximate similarity queries with quality guarantees, and we conduct a thorough experimental evaluation to compare approximate similarity search techniques under a unified framework, on synthetic and real datasets in memory and on disk. 
Although data series differ from generic multidimensional vectors (series usually exhibit correlation between neighboring values), our results show that data series techniques answer approximate %similarity 
queries with strong guarantees and an excellent empirical performance, on data series and vectors alike. 
These techniques outperform the state-of-the-art approximate techniques for vectors when operating on disk, and remain competitive in memory. 
\end{abstract}
	
	
\section{Introduction}
\label{sec:introduction}
%\noindent{\bf Data Series.}
%\footnote{When the order is based on time, it is called a \emph{time series}. We note that the order can be defined by angle (e.g., in radial profiles), mass (e.g., in mass spectroscopy), position (e.g., in genome sequences), and others~\cite{conf/sofsem/Palpanas2016}. The terms \emph{data series}, \emph{time series} and \emph{sequence} are used interchangeably.}.
\noindent{\bf Motivation.} 
A data series is a sequence of ordered real values\footnote{The order attribute can be angle, mass, time, etc.~\cite{conf/sofsem/Palpanas2016}. When the order is time, the series is called a \emph{time series}. We use \emph{data series}, \emph{time series} and \emph{sequence} interchangeably.}. 
Data series are ubiquitous, appearing in nearly every domain including science and engineering, medicine, business, finance and economics~\cite{KashinoSM99,Shasha99,humanbehaviorpatterns,volker,DBLP:conf/edbt/MirylenkaCPPM16,HuijseEPPZ14,percomJournal,windturbines,spikesorting,VALMOD,journal/jte/Williams2003,conf/compstats/Hebrail2000}. 
The increasing presence of IoT technologies is making collections of data series grow to multiple terabytes~\cite{DBLP:journal/sigmod/Palpanas15}. 
These data series collections need to be analyzed in order to extract knowledge and produce value~\cite{Palpanas2019}. 
The process of retrieving similar data series (i.e., similarity search), forms the backbone of most analytical tasks, including outlier detection~\cite{journal/csur/Chandola2009,conf/icde/boniol20}, frequent pattern mining~\cite{conf/kdd/Mueen2012}, clustering~\cite{conf/kdd/Keogh1998,conf/sdm/Rodrigues2006,conf/icdm/Keogh2011,journal/pattrecog/Warren2005}, and classification~\cite{journal/jmlr/Chen2009}. 
Thus, to render data analysis algorithms and pipelines scalable, we need to make similarity search more efficient. 

%The similarity search algorithm for data series returns the set of candidate data series in a collection that is similar to a given query series. This algorithm is often reduced to the nearest neighbor problem where data series are represented as data points in multidimensional space and their (dis)similarity is evaluated using a distance function.
%
%Although a data series can be represented as a vector in high dimensional space, conventional vector-based approaches are not adapted for the following two reasons: (a) they cannot scale to thousands of dimensions; and (b) they do not leverage the correlation between dimensions typical for data series.
%
%Similarity search methods can either return exact or approximate answers. Exact methods are costly while approximate methods offer better efficiency at the expense of losing some accuracy. We call approximate methods that do not provide any guarantees on the results $ng$-approximate, and those that provide guarantees on the approximation error, $\delta$-$\epsilon$-approximate methods, where $\epsilon$ is the approximation error and $\delta$, the probability that $\epsilon$ will not be exceeded.

%A plethora of similarity search methods have been published by the community including techniques designed for generic vectors~\cite{conf/stoc/indyk1998,conf/sigmod/Guttman1984,conf/icmd/Beckmann1990,conf/vldb/Ciaccia1997,conf/vldb/Weber1998,conf/cikm/Hakan2000,journal/tpami/jegou2011,conf/vldb/sun14,journal/pami/babenko15,journal/corr/malkov16}
%%~\cite{conf/stoc/indyk1998,journal/cacm/bentley1975,conf/sigmod/Guttman1984,conf/icmd/Beckmann1990,conf/vldb/bertchold1996, conf/vldb/Ciaccia1997,conf/vldb/Weber1998,conf/cikm/Hakan2000,journal/tpami/jegou2011,journal/iccv/xia2013,conf/vldb/sun14,journal/pami/babenko15,journal/corr/malkov16}
%and those specific to data series~\cite{conf/fodo/Agrawal1993,conf/icde/Rafiei99,conf/kdd/Karras2011,conf/kdd/Mueen2012,dpisax,ulisse,journal/vldb/linardi19,conf/bigdata/peng18,conf/kdd/ColeSZ05,conf/icdm/Camerra2010,journal/edbt/Schafer2012,conf/vldb/Wang2013,journal/kais/Camerra2014,journal/vldb/Zoumpatianos2016}.
%%~\cite{conf/fodo/Agrawal1993,conf/kdd/shieh1998,conf/icde/Rafiei99,conf/kdd/Karras2011,conf/kdd/Mueen2012,code/Mueen2017,dpisax,ulisse,journal/vldb/linardi19,conf/bigdata/peng18,conf/icde/shatkay1996,conf/kdd/Keogh1997,conf/ssdm/Wang2000,conf/kdd/ColeSZ05,conf/icdm/Camerra2010,journal/edbt/Schafer2012,conf/vldb/Wang2013,journal/kais/Camerra2014,journal/vldb/Zoumpatianos2016}.

%This work proposes a versatile index called SISS, which supports progressive query answering with probabilistic guarantees. We first describe related work, then present our proposed solution. We succinctly report the results of our extensive experimental evaluation of \emph{exact} methods~\cite{journal/pvldb/echihabi2018}, give a glimpse of some very interesting results from an ongoing experimental study focused on \emph{approximate} methods, and describe future work directions.
%%We thus differ from other experimental studies which focused on the efficiency of exact search~\cite{journal/pvldb/echihabi2018}, the accuracy of dimensionality reduction techniques and similarity measures for classification tasks~\cite{journal/dmkd/Keogh2003,conf/vldb/Ding2008,DBLP:journals/datamine/BagnallLBLK17}, or in-memory data~\cite{journals/corr/li16,conf/sisap/martin17}. 

\noindent{\bf Similarity Search.}
A large number of data series similarity search methods has been studied, supporting exact search~\cite{conf/fodo/Agrawal1993,conf/kdd/shieh1998,conf/icde/Rafiei99,conf/kdd/Karras2011,conf/kdd/Mueen2012,code/Mueen2017}, approximate search~\cite{conf/icde/shatkay1996,conf/kdd/Keogh1997,conf/ssdm/Wang2000,conf/kdd/ColeSZ05,journal/vldb/Dallachiesa2014}, or both~\cite{conf/icdm/Camerra2010,journal/edbt/Schafer2012,conf/vldb/Wang2013,journal/kais/Camerra2014,journal/vldb/Zoumpatianos2016,dpisax,journal/pvldb/kondylakis18,ulisse,journal/vldb/linardi19,conf/bigdata/peng18,dpisaxjournal,coconutjournal,conf/icde/peng20}. 
In parallel, the research community has also developed exact~\cite{journal/cacm/bentley1975,conf/sigmod/Guttman1984,conf/icmd/Beckmann1990,conf/vldb/bertchold1996, conf/vldb/Ciaccia1997,conf/vldb/Weber1998,conf/cikm/Hakan2000} and approximate~\cite{conf/stoc/indyk1998} similarity search techniques geared towards generic multidimensional vector data\footnote{A comprehensive survey of techniques for multidimensional vectors can be found elsewhere~\cite{book/multiD/samet2005}.}. 
In the past few years though, we are witnessing a renewed interest in the development of approximate methods~\cite{journal/tpami/jegou2011,journal/iccv/xia2013,conf/vldb/sun14,journal/pami/babenko15,journal/corr/malkov16}. 

This study is the first experimental comparison of the efficiency and accuracy of data series approximate similarity search methods ever conducted. Specifically, we evaluate the accuracy of both data series specific approximate similarity search methods, as well as that of approximate similarity search algorithms that operate on general multidimensional vectors. 
%In addition to the approximate search techniques with no approximation guarantees that are inherently supported by some data series exact methods, 
Moreover, we propose modifications to data series techniques in order to support approximate query answering with theoretical guarantees, following~\cite{conf/icde/Ciaccia2000}. 

Our experimental evaluation covers in-memory and out-of-core experiments, the largest publicly available datasets, extensive comparison criteria, and a new set of methods that have never been compared before. 
We thus differ from other experimental studies, which focused on the efficiency of exact search~\cite{journal/pvldb/echihabi2018}, the accuracy of dimensionality reduction techniques and similarity measures for classification tasks~\cite{journal/dmkd/Keogh2003,conf/vldb/Ding2008,DBLP:journals/datamine/BagnallLBLK17}, or in-memory high-dimensional methods~\cite{journal/tkde/li19, conf/sisap/martin17, journal/pvld/naidan2015}. 
%We note that even though previous works report experimental evaluation results, these are not conclusive, for the following reasons: a) frequently algorithms were implemented in different programming languages, b) the experiments were conducted using different settings, datasets, platforms, hardware and tuning parameters, and c) they only compared each method to a subset of the competing methods at a time. 
%As the long term health of any discipline depends on reproducible and experimentally validated results~\cite{journal/jss/Tichy1995}.
In this study, we focus on the problem of \emph{approximate whole matching similarity search in collections with a very large number of data series}, i.e., similarity search that produces approximate (not exact) results, by calculating distances on the whole (not a sub-) sequence.
This problem represents a common use case across many domains~\cite{journal/tpami/ge2014,conf/vldb/sun14,journal/pami/babenko15,journal/corr/malkov16,DBLP:conf/kdd/ColeSZ05,DBLP:conf/sigmod/ManninoA18,DBLP:conf/edbt/GogolouTPB19,Palpanas2019}. 
%%~\cite{url:adhd,url:sds,conf/compstats/Hébrail2000,SENTINEL-2}

%The experiments presented in the current paper represent 50 days of computation time, while our work involved more than 130 days of computation time in total (including initial testing and tuning of the different methods).

%  	{\color{red} Mention how this relates to the curse of dimensionality per Bellman \cite{book/mit/Bellman1961}.REPHRASE "  The expression ``curse of dimensionality'' is due to Bellman and in statistics it relates to the fact that the convergence of any estimator to the true value of a smooth function defined on a space of high dimension is very slow"}

\noindent{\bf Contributions.}
Our key contributions are as follows:

%1. We present a thorough discussion of the data series similarity search problem, formally defining its different variations that have been studied in the literature under diverse and conflicting names. Thus, establishing a common language that will facilitate further work in this area.
1. We present a similarity search taxonomy that classifies methods based on the quality guarantees they provide for the search results, and that unifies the varied nomenclature used in the literature. %, thus overcoming confusion and misunderstandings.
% and establishing a common language that will facilitate further work in this area.
Following this taxonomy, we include a brief survey of similarity search approaches supporting approximate search, bringing together works from the data series and multidimensional data research communities.
% , including both traditional signal processing techniques and modern specialized ones.
%\item

2. We propose a new set of approximate approaches with theoretical guarantees on accuracy and excellent empirical performance, based on modifications to the current data series exact methods.

3. We evaluate all methods under a unified framework to prevent implementation bias. We used the most efficient C/C++ implementations available for all approaches, and developed from scratch in C the ones that were only implemented in other programming languages. Our new implementations are considerably faster than the original ones.
% Moreover, we conducted a careful inspection of the code bases, and applied to all of them the same set of optimizations (e.g., with respect to memory management, Euclidean distance calculation, etc.), leading to considerably faster performance.
%\item

4. We conduct the first comprehensive experimental evaluation of the efficiency and accuracy of data series approximate similarity search approaches, using synthetic and real series and vector datasets from different domains, including the two largest vector datasets publicly available.
%and report results for prominent solutions that had never been compared before.
%In addition, we report the first large scale experiments with carefully crafted query workloads that include queries of varying difficulty, which can effectively stress-test all the approaches.
The results unveil the strengths and weaknesses of each method, and lead to recommendations as to which approach to use. % given different data characteristics.

5. Our results show that the methods derived from the exact data series indexing approaches generally surpass the state-of-the-art techniques for approximate search in vector spaces. 
This observation had not been made in the past, and it paves the way for exciting new developments in the field of approximate similarity search for data series and multidimensional data at large. 
 
%reveal characteristics that have not been reported in the literature, and lead to a deep understanding of the different approaches and their performance.
%\item

6. We share all source codes, datasets, and queries~\cite{url/DSSeval2}. %workloads used in our study~\cite{url/DSSeval2}. 
%This will render our work reproducible and further help the community to agree on and establish a much needed data series similarity search benchmark~\cite{journal/dmkd/Keogh2003,conf/kdd/Zoumpatianos2015,johannesjoural2018}.

%All the results, including detailed descriptions on how to reproduce them are available online~\cite{url/DSSeval}.
%\end{itemize}

%\noindent{\bf Outline.}
%The rest of this paper is organized as follows.
%In Section~\ref{sec:problem}, we define the problem of data series similarity search and introduce the notation used. 
%In Section~\ref{sec:approaches}, we survey the state-of-the-art in similarity search methods and dimensionality reduction techniques. 
%In Section~\ref{sec:experiments}, we detail the experimental framework and report and interpret our key results. In Section~\ref{sec:discussion}, we tie our key findings together in an overall synopsis. 
%Finally, Section~\ref{sec:conclusions} highlights our key contributions and identifies future research directions.

%We focus on the problem of \emph{whole matching similarity search in collections with a very large number of data series}, i.e., similarity search that produces approximate or exact results, by calculating distances on the whole (not a sub-) sequence.



%\section{Preliminaries}
\section{Definitions and Terminology}
\label{sec:problem}

%Similarity search involves finding a set of objects in a database that are similar to a query according to some definition of distance.
%A common abstraction of data series similarity search is to consider the query and candidate data series in the database to be points in a metric space and the sameness (or difference) is typically determined based on a distance function.
%Our definitions and experimental framework apply to any distance function but for simplicity and clarity, we choose to use only Euclidean distance.

Similarity search represents a common problem in various areas of computer science.
In the case of data series, several different flavors have been studied in the literature, often times using overloaded and conflicting terms.
%!!! COMMENTED OUT BY KOSTAS !!! As a result, when reading the literature, it is not always clear what problem exactly each approach has been designed to solve, and on which situations it is applicable.
%This has contributed to an overall confusion, which hinders further advances in the field.
We summarize here these variations, and provide definitions, thus setting a common language (for more details, see~\cite{journal/pvldb/echihabi2018}).

%\subsection{Definitions and Terminology}
\noindent{\bf On Sequences.}
%\begin{defn} \label{def:dataseq}
A \textit{\textbf{data series}} $S(p_1,p_2,...,p_n)$ is an ordered sequence of points, $p_i$, $1 \leq i \leq n$.
The number of points, $|S|=n$, is the length of the series.
We denote the $i$-th point in $S$ by $S[i]$; then $S[i:j]$ denotes the \textit{\textbf{subsequence}} $S(p_i,p_{i+1},...,p_{j-1},p_j)$, where $1 \leq i \leq j \leq n$.
We use $\mathbb{S}$ to represent all the series in a collection (dataset).
%\end{defn}
Each point in the series may represent the value of a single variable, i.e., \textit{\textbf{univariate series}}, or of multiple variables, i.e., \textit{\textbf{multivariate series}}.
If these values encode errors, or imprecisions, we talk about uncertain data  series~\cite{DBLP:conf/ssdbm/AssfalgKKR09,DBLP:conf/edbt/YehWYC09,DBLP:conf/kdd/SarangiM10,DBLP:journals/pvldb/DallachiesaNMP12,journal/vldb/Dallachiesa2014}.

Note that in the context of similarity search, a data series of length $n$ can be represented as a single point in an $n$-dimensional space. %$\mathbb{R}^n$.
Then the values and length of $S$ are referred to as \emph{dimensions} and \emph{dimensionality}, respectively.

%We now define the subsequence of a data series.
%\begin{defn} \label{def:datasub}
%We say that $S(c_i,c_{i+1},...,c_{i+s-1})$ is a \textit{\textbf{subsequence}} of $S(c_1,c_2,...,c_n)$, if $1 \leq i \leq n+1-s$ and $\forall \ i \leq j \leq i+s-1$, $c_{j} \in S$.
%\end{defn}
%
%In this case, we simply denote $S_{sub}(c_i,c_{i+1},...,c_{i+s-1})$ by $S_{(i,s)}$, and it holds that $|S_{(i,s)}| = s$.

%In each of these categories, the most common queries are \textit{\textbf{range queries}} and \textit{\textbf{k-NN queries}}.
%Besides, each category can be answered using \textit{\textbf{ exact, approximate, $\epsilon$-approximate, or probabilistic methods}}.

%\begin{defn} \label{def:datasum}
%Given a data series $DS$ of length $n$, a \textit{\textbf{data series summarization}} $DSS$ is an ordered sequence of $l$ real values obtained after applying a dimensionality reduction technique to $DS$ such that $|DSS| = l \ and \ 1 \leq l \leq n$. $DS$ can then be represented as a point in a real $l$-dimensional metric space $\mathbb{R}^l$.
%\end{defn}

\noindent{\bf On Distance Measures.}
A data series \textit{\textbf{distance}} is a function that measures the (dis)similarity of two data series~\cite{berndt1994using,das1997finding,DBLP:conf/edbt/AssfalgKKKPR06,DBLP:conf/icde/ChenNOT07,journal/dmkd/Wang2013,DBLP:conf/ssdbm/MirylenkaDP17}.
The distance between a query series, $S_Q$, and a candidate series, $S_C$, is denoted by $d(S_Q,S_C)$.
%\begin{defn} \label{def:eucldist}
%The \textit{\textbf{Euclidean distance}} between two data series $DS_Q \ and \ DS_C$ of length $n$, represented by two points $Q(q_1,q_2,...,q_n)$  and $C(c_1,c_2,...,c_n)$ in $\mathbb{R}^d$, is defined as follows \cite{conf/sdm/Batista2011}:
%	\[ d_{L_2}(DS_Q,DS_C) \equiv d_{L_2}(Q,C) \equiv \sqrt[2]{\sum_{i=1}^{n} \ (q_i-c_i)^2} \]
%\end{defn}
The Euclidean distance is the most widely used, and one of the most effective for large series collections~\cite{conf/vldb/Ding2008}.
%We note that an additional advantage of Euclidean distance is that in the case of Z-normalized series (mean=$0$, stddev=$1$), which are very often used in practice~\cite{conf/kdd/Zoumpatianos2015}, it can be exploited to compute Pearson correlation~\cite{conf/icde/Rafiei99}.
%In addition to the distance used to compare data series in the high-dimensional space, 
Some similarity search methods also rely on the \textit{lower-bounding distance} (distances in the reduced dimensionality space are guaranteed to be smaller than or equal to distances in the original space)~\cite{journal/kais/Camerra2014,journal/vldb/Zoumpatianos2016,journal/edbt/Schafer2012,conf/vldb/Wang2013,dpisax,ulisse,conf/vldb/Ciaccia1997,conf/kdd/Karras2011} and \textit{upper-bounding distance} (distances in the reduced space are larger than the distances in the original space)~\cite{conf/vldb/Wang2013,conf/kdd/Karras2011}. 
%A \textit{\textbf{lower-bounding distance}} is a distance defined in the reduced dimensional space satisfying the lower-bounding property, i.e., the distance between two series in the reduced space is guaranteed to be smaller than or equal to the distance between the series in the original space~\cite{conf/sigmod/Faloutsos1994}. 
%Inversely, an \textit{\textbf{upper-bounding distance}} ensures that distances in the reduced space are larger than the distances in the original space~\cite{conf/vldb/Wang2013,conf/kdd/Karras2011}.

\noindent{\bf On Similarity Search Queries.}
%We now define the different forms of data series similarity search queries.
We assume a data series collection, $\mathbb{S}$, a query series, $S_Q$, and a distance function $d(\cdot,\cdot)$.
%
%A \textit{\textbf{k-Nearest-Neighbor (k-NN) query}} $q$, finds the set $\mathbb{S} = \{C_1,..,C_k\}$ of the $k$ nearest points to the query point $Q$ in a search space $D \subseteq \mathbb{R}^d$.
A \textit{\textbf{k-Nearest-Neighbor (k-NN) query}} identifies the $k$ series in the collection with the smallest distances to the query series, while an \textit{\textbf{r-range query}} identifies all the series in the collection within range $r$ {\color{black}from} the query series.

\vspace*{-0.1cm}
\begin{defn}~\cite{journal/pvldb/echihabi2018} \label{def:knnquery}
Given an integer $k$, a \textit{\textbf{k-NN query}} retrieves the set of series $\mathbb{A} = \{ \{S_{C_1},...,S_{C_k}\} \subseteq \mathbb{S} | \forall \ S_C \in \mathbb{A} \ and \ \forall \ S_{C'} \notin \mathbb{A}, \ d(S_Q,S_C) \leq d(S_Q,S_{C'})\}$.
\end{defn}
%An \textit{\textbf{r-range query}} produces as an answer the set of series, $\mathbb{A}$, such that the distance between the query series, $S_Q$, and any candidate series, $S_C\in \mathbb{S}$, is at most $r$.
%An \textit{\textbf{r-range query}} identifies all the series in the collection within range $r$ form the query series.
\vspace*{-0.3cm}
\begin{defn}~\cite{journal/pvldb/echihabi2018} \label{def:rquery}
Given a distance $r$, an \textit{\textbf{r-range query}} retrieves the set of series $\mathbb{A} = \{S_C \in \mathbb{S} | d(S_Q,S_C) \leq r\}$.
\end{defn}
\vspace*{-0.1cm}

%\textit{S} is also known in high-dimensional geometry as $B^n_r(q)$, that is the closed \textit{n}-ball of query point \textit{Q} with radius $r$.
%It is a closed ball because the range query condition includes points whose distance from \textit{Q} is equal to $r$.
%In two-dimensional space, $B^2_r(Q)$ is the closed disk with center \textit{q} and radius $r$.
%(A similar definition has appeared in~\cite{conf/icde/Ciaccia2000}).

%\begin{defn} \label{def:distset}
%Given a point $C'$, and the set $\mathbb{S} \subseteq D \subseteq %\mathbb{R}^d$ returned by a similarity query $q$, the distance of $C_i$ to %$\mathbb{S}$ is defined as follows:
%	 \[d(C',\mathbb{S}) = r \ \textit{if  q is an $r$-range query} \]
%	 \[d(C',\mathbb{S}) = d(C',C) \ \textit{if q is a $k$-NN query}, \]
%\[\textit{where $C \in \mathbb{S}$ is the corresponding nearest neighbor}\]
%\end{defn}

%Definitions~\ref{def:wholematch} and~\ref{def:submatch} are based on the definitions that appeared in \cite{conf/sigmod/Faloutsos1994}.
%\begin{defn} \label{def:wholematch}
%Given a search space $D \subseteq \mathbb{R}^d$ of candidate data series, and a query data series $DS_Q$, a \textit{\textbf{whole matching query}} finds the data series in $D$ that match $DS_Q$.It is important to note that the query and candidate data series must have the same length.
%\end{defn}
%
%\begin{defn} \label{def:submatch}
%Given a search space $D \subseteq \mathbb{R}^d$ of candidate data series of arbitrary length, and a query data series $DS_Q$, a \textit{\textbf{subsequence matching query}} finds the set $\mathbb{S}$ of all data series $C$ in $D$ that contain subsequences $C_{sub}$ matching $Q$.
%Typically $|Q| \ll |C|$, that is the length of $Q$ is typically much smaller than that of the candidate data series $C$.
%\end{defn}
We additionally identify %the following two categories of k-NN and range queries.
%~\cite{conf/sigmod/Faloutsos1994}.
%In 
\textit{\textbf{whole matching (WM)}} queries (similarity between an entire query series and an entire candidate series), and 
%All the series involved in the similarity search have to have the same length.
%In 
\textit{\textbf{subsequence matching (SM)}} queries (similarity between an entire query series and all subsequences of a candidate series).
%In this case, candidate series can have different lengths, but should be longer than the query series. 

\vspace*{-0.1cm}
\begin{defn}~\cite{journal/pvldb/echihabi2018} \label{def:wholematch}
	A \textit{\textbf{WM query}} finds the candidate data series $S \in \mathbb{S}$ that matches $S_Q$, where $|S|=|S_Q|$. 
\end{defn}
\vspace*{-0.3cm}
\begin{defn}~\cite{journal/pvldb/echihabi2018} \label{def:submatch}
	A \textit{\textbf{SM query}} finds the subsequence $S[i:j]$ of a candidate data series $S \in \mathbb{S}$ that matches $S_Q$, where $|S[i:j]| = |S_Q| < |S|$.
\end{defn}
\vspace*{-0.1cm}

%Definitions~\ref{def:wholematch} and~\ref{def:submatch} are based on the definitions that appeared in~\cite{conf/sigmod/Faloutsos1994}.


% ; though, all the candidate series should be at least as long as the query series.

In practice, we have WM queries on large collections of short series~\cite{SENTINEL-2,url:sds}, SM queries on large collections of short series~\cite{url:adhd}, and SM queries on collections of long series~\cite{url/data/seismic}.
Note that a SM query can be converted to WM~\cite{ulisse,journal/vldb/linardi19}.
%: we create a new collection that comprises all the overlapping subsequences (each long series in the candidate set is chopped into overlapping subsequences of the length of the query), and perform a WM query against these subsequences~\cite{ulisse,journal/vldb/linardi19}.

\noindent{\bf On Similarity Search Methods.}
The similarity search algorithms (k-NN or range) that always produce correct and complete answers are called \textit{\textbf{exact}}.
Algorithms that do not satisfy this property are called %Nevertheless, we can also develop algorithms without such strong guarantees: we call such algorithms 
\textit{\textbf{approximate}}.
%As we discuss below, there exist different flavors of approximate similarity search algorithms.
%
%\begin{defn} \label{def:examatch}
%Given two points $C$ and $Q$ in $\mathbb{R}^d$, representing two data series $DS_C$ and $DS_Q$, $DS_C$ is an \textit{\textbf{exact match}} for $DS_Q$ if $C$ belongs to the set $\mathbb{S}$ returned by a similarity query $q$ per Definitions~\ref{def:rquery} and~\ref{def:knnquery}.
%\end{defn}
%
An {\textit{\bf$\bm{\epsilon}$-approximate}} algorithm guarantees that its distance results have a relative error no more than $\epsilon$, i.e., the approximate distance is at most $(1+\epsilon)$ times the exact one. 
% distance.
% (for some small $\epsilon$).
A {\bf $\bm{\delta}$-$\bm{\epsilon}$-approximate} algorithm, guarantees that its distance results will have a relative error no more than $\epsilon$ (i.e., the approximate distance is at most $(1+\epsilon)$ times the exact distance), with a probability of at least $\delta$.
An \textit{\textbf{ng-approximate}} (no-guarantees approximate) algorithm does not provide any guarantees (deterministic, or probabilistic) on the error bounds of its distance results.

\vspace*{-0.2cm}
\begin{defn}~\cite{journal/pvldb/echihabi2018} \label{def:epsmatch}
Given a query $S_Q$, and $\epsilon \geq 0$, an \textit{\textbf{$\bm{\epsilon}$-approximate}} algorithm guarantees that all results, $S_C$, are at a distance $d(S_Q,S_C) \leq (1+\epsilon)\ d(S_Q,[\text{k-th NN of }S_Q])$ in the case of a $k$-NN query, and distance $d(S_Q,S_C) \leq (1+\epsilon)r$ in the case of an r-range query.
%, $S_C \in \mathbb{S}$.
\end{defn}
%\begin{defn} \label{def:epsmatch}
%Given two points $C_\epsilon$ and $Q$ in $\mathbb{R}d$, representing two data series $DS_{C_\epsilon}$ and $DS_Q$, and a similarity query $q$, $DS_{C_\epsilon}$ is an \textit{\textbf{$\epsilon$-approximate match}} for $DS_Q$ if $d(Q,C_\epsilon)$ is guaranteed to be within an error $\epsilon$ from $d(\mathbb{S},C_\epsilon)$. The value of $\epsilon$ is a parameter known in advance.
%	 \[d(Q,C_{\epsilon}) \leq (1+\epsilon)d(\mathbb{S},C_{\epsilon})\]
%The distance $d(\mathbb{S},C_\epsilon)$ is defined the same way as in Definition~\ref{def:appmatch} by substituting ${C_x}$ with ${C_\epsilon}$.
%\end{defn}

	% \begin{defn} \label{def:epsmatch}
	% 	Given a query $S_Q$, and $0 \leq \delta \leq 1$, a \textit{\textbf{$\delta$-approximate}} algorithm guarantees that all results, $S_C$, are exact with probability at least $\delta$.
	% \end{defn}
\vspace*{-0.4cm}
\begin{defn}~\cite{journal/pvldb/echihabi2018} \label{def:probmatch}
Given a query $S_Q$, $\epsilon \geq 0$, and $\delta \in [0,1]$, a \textit{\textbf{$\bm{\delta}$-$\bm{\epsilon}$-approximate}} algorithm produces results, $S_C$, for which $Pr[d(S_Q,S_C)$ $\leq (1+\epsilon)\ d(S_Q,[\text{k-th NN of }S_Q])] \geq \delta$ in the case of a $k$-NN query, and $Pr[d(S_Q,S_C) \leq (1+\epsilon)r] \geq \delta$) in the case of an r-range query.
\end{defn}
%\begin{defn} \label{def:probmatch}
%Given two points $C^{\delta}_\epsilon$ and $Q$ in $\mathbb{R}d$, representing two data series $DS_{C^\delta_\epsilon}$ and $DS_Q$, and a similarity query $q$, $DS_{C^\delta_\epsilon}$ is
%a \textit{\textbf{$\delta$-$\epsilon$-approximate match}}, in short a \textit{\textbf{probabilistic match}}, for $DS_Q$ if $d(Q,C^\delta_\epsilon)$ is guaranteed, with probability $\delta$, to be within an error $\epsilon$ from $d(\mathbb{S},C^{\delta}_\epsilon)$. The values of $\epsilon$ and $\delta$ are parameters known in advance.
%	 \[\Pr\left[d(Q,C^\delta_\epsilon) \leq (1+\epsilon)d(\mathbb{S},C^\delta_\epsilon )\right] \geq \delta\]
%The distance $d(\mathbb{S},C^\delta_\epsilon )$ is defined the same way as in Definition~\ref{def:appmatch} by substituting ${C_x}$ with ${C^\delta_\epsilon}$.
%\end{defn}
\vspace*{-0.4cm}
\begin{defn}~\cite{journal/pvldb/echihabi2018} \label{def:appmatch}
Given a query $S_Q$, an \textit{\textbf{ng-approximate}} algorithm produces results, $S_C$, that are at a distance $d(S_Q,S_C) \leq (1+\theta)\ d(S_Q,[\text{k-th NN of }S_Q])$ in the case of a $k$-NN query, and distance $d(S_Q,S_C) \leq (1+\theta)r$ in the case of an r-range query, for an arbitrary value $\theta \in \mathbb{R}_{>0}$.
%, $S_C \in \mathbb{S}$.
\end{defn}
%\begin{defn} \label{def:appmatch}
%Given two points $C_x$ and $Q$ in $\mathbb{R}d$, representing two data series $DS_{C_x}$ and $DS_Q$, and a similarity query $q$, $DS_{C_x}$ is an \textit{\textbf{approximate match}} for $DS_Q$ if $d(Q,C_x)$ is within an error $x$ from $d(\mathbb{S},C_x)$, where $\mathbb{S}$ is the set of exact matches returned by $q$.
%	 \[d(Q,C_x) \leq (1+x) d(\mathbb{S},C_x)\]
%The distance $d(\mathbb{S},C_x)$ depends on the type of the similarity query $q$.
%\[d(\mathbb{S},C_x) = r \ \textit{if q is an $r$-range query} \]
%	 	 \[d(\mathbb{S},C_x) = d(C_x,C) \ \textit{if q is a $k$-NN query}, \]
%	 	 where $C \in \mathbb{S}$ is the actual nearest neighbor.
%	 	 For instance, if $C_x$ is the approximate $i$th nearest neighbor then $C$ is the actual $i$th nearest neighbor. Moreover,
%the value of $x$ is not known in advance.
%%	 \[d(Q,C_x) \leq (x+ d_{ref}) \ if \ d_{ref}  = 0\]
% %{\color{red} For the special case when $d(Q, C_{approx}) - d(Q,C) =0$, the effective error is defined as:
% %	 \[d(Q,C_x) \leq d(Q,C) + x\]
% %	 }
%\end{defn}

In the data series literature, \textit{ng-approximate} algorithms have been referred to as \emph{approximate}, or \emph{heuristic} search~\cite{journal/kais/Camerra2014,journal/vldb/Zoumpatianos2016,journal/edbt/Schafer2012,conf/vldb/Wang2013,dpisax,ulisse}.
Unless otherwise specified, %for the rest of this paper 
we will refer to \textit{ng-approximate} algorithms simply as approximate. Approximate matching in the data series literature
%%was first introduced in \cite{conf/kdd/Shieh2008} and
consists of pruning the search space, by traversing one path of an index structure representing the data, visiting at most one leaf, to get a baseline best-so-far (bsf) match.
In the multidimensional literature, ng-approximate similarity search is also called \textit{Approximate Nearest Neighbor (ANN)}~\cite{journal/tpami/jegou2011}, $\epsilon$-approximate 1-NN search is called \textit{c-ANN}~\cite{conf/vldb/sun14}, and $\epsilon$-approximate k-NN search is called \textit{c-k-ANN}~\cite{qalsh}, where $c$ stands for the approximation error and corresponds to $1+\epsilon$. 
%The terminology we propose is expressive enough to cover the different variations of approximate k-NN and range search.


%In the case of a $k$-NN search, a $\delta$-$\epsilon$-approximate match for query $Q$ corresponds to the point $C^\delta_\epsilon$ whose distance from the actual $k$th nearest neighbor of Q is guaranteed, with probability $\delta$, to be within a relative error $\epsilon$.
%\begin{defn} \label{def:match}
%A data series $C$ is \textit{\textbf{match}} for a query data series $Q$ if it is either an exact, approximate, $\epsilon$-approximate or probabilistic match.
%\end{defn}

%\begin{defn} \label{def:effepsilon}
%Given a query data series $S_Q$, an exact match $S_C$ and an approximate, $\epsilon$-approximate or probabilistic match $S_{C_{approx}}$, the \textit{\textbf{effective error, $\epsilon_{\text{eff}}$}} of $S_{C_{approx}}$ is:
%\[\epsilon_{\text{eff}} = \frac {d(S_Q,S_{C_{approx}}) - d(S_Q, S_C)}  {d(S_Q,S_C)}  \]
%%{\color{red} For the special case when $d(Q, C_{approx}) - d(Q,C) =0$, the effective error is defined as: \[\epsilon_eff = d(Q, C_{approx}) - d(Q,C) \]}
%\end{defn}
%It is noteworthy to point out that the $\epsilon$ in Definition~\ref{def:epsmatch} constitutes an upper bound on the actual approximation error $\epsilon_{eff}$ in Definition~\ref{def:effepsilon}.
%Without loss of generality, we do not consider the case where $d(Q,C) = 0$. This can happen in range queries with radius zero, or kNN queries where the nearest neighbor is the query point itself. In these cases, the definition for effective error can be modified to use the absolute error instead of the relative error.
%Also, we do not consider the absolute value since the difference between the approximate and actual distances is always positive.

Observe that when $\delta = 1$, a $\delta$-$\epsilon$-approximate method becomes $\epsilon$-approximate, and when $\epsilon=0$, an $\epsilon$-approximate method becomes exact~\cite{conf/icde/Ciaccia2000}.
It it also possible that the same approach implements both approximate and exact algorithms~\cite{conf/kdd/shieh1998,conf/vldb/Wang2013,journal/kais/Camerra2014,journal/vldb/Zoumpatianos2016,journal/edbt/Schafer2012}. 
%Definitions~\ref{def:epsmatch}, \ref{def:probmatch} and~\ref{def:effepsilon} are general and apply to $r$-range and $k$-NN queries.
%They generalize the definitions in~\cite{conf/icde/Ciaccia2000} and \cite{journal/acm/Arya1998} which are specific to $k$-NN queries.
%Methods that provide exact answers with probabilistic guarantees are considered $\delta$-0-approximate. 
%These methods guarantee distance results to be exact with probability at least $\delta$ ($0 \leq \delta \leq 1$ and $\epsilon$ = 0).
%(We note that in the case of $k$-NN queries, Def.~\ref{def:epsmatch} corresponds to the \emph{approximately correct NN}~\cite{conf/icde/Ciaccia2000} and \emph{$(1+\epsilon)$-approximate NN}~\cite{journal/acm/Arya1998}, while  Def.~\ref{def:probmatch} corresponds to the \emph{probably approximately correct NN}~\cite{conf/icde/Ciaccia2000}.)

\noindent{\bf Scope.}
%\subsection{Scope}
In this study, we focus on \emph{univariate} series with \emph{no uncertainty}, where each point is drawn from the domain of real values, $\mathbb{R}$, and we evaluate \emph{approximate} methods for \emph{whole matching} in datasets containing a \emph{very large number of series}, using \emph{$k$-NN queries} and the \emph{Euclidean distance}. 
This scenario is key to numerous %analysis algorithms, and is very relevant in
analysis pipelines in 
practice~\cite{journal/pattrecog/Warren2005,conf/kdd/Zoumpatianos2015,conf/sofsem/Palpanas2016,Palpanas2019}, in fields as varied as neuroscience~\cite{golay1998new}, seismology~\cite{kakizawa1998discrimination}, retail data~\cite{DBLP:conf/kdd/KumarPW02}, and energy~\cite{kovsmelj1990cross}. 
% bring back for camera ready!: The lessons learned from this study can hold for other query types, too. %distances and query workloads.
%Note also that some of the insights gained by this study could carry over to other settings, such as, $r$-range queries, dynamic time warping distance, or approximate search.





\vspace*{-0.3cm}

\section{Similarity Search Primer}
\label{sec:approaches}

Similarity search methods aim at answering a query efficiently by limiting the number of data points accessed, %and/or the footprint of each data point, 
while minimizing the I/O cost of accessing raw data on disk and the CPU cost %incurred 
when comparing raw data to the query (e.g., Euclidean distance calculations). 
These goals are achieved by exploiting summarization techniques, and using efficient data structures (e.g., an index) and search algorithms. 
%These algorithms, which can return exact or approximate answers, process data either sequentially or using an index, and try to minimize two major costs: The IO cost of accessing the raw data on disk and the CPU cost incurred when comparing the raw data to the query (in our analysis, the comparisons are Euclidean distance calculations). 
%and alternative approaches.
Note that solutions based on sequential scans are geared to exact similarity search~\cite{conf/kdd/Mueen2012,code/Mueen2017}, and cannot support efficient approximate search, since all candidates are always read.

%Sequential methods proceed in one step to answer a similarity search query. Each candidate is read sequentially from the raw data file and compared to the query. 
%Particular optimizations can be applied to limit the number of these comparisons~\cite{conf/kdd/Mueen2012}. 
%Some sequential methods work with the raw data in its original high-dimensional representation~\cite{conf/kdd/Mueen2012}, while others perform transformations on the raw data before comparing them to the query~\cite{code/Mueen2017}. 
%Sequential methods are typically used for exact search, since approximate search still requires a pass over the whole dataset.
%%ADD APPROXIMATE SEQUENTIAL METHODS IF ANY. EXHAUSTIVE vs NON EXHAUSTIVE\\
%%ADC can be used for ANN exhaustive search: all candidates are quantized, query is not so the distance between the query and candidates is approximated using ADC distance: distance between query and all candidate codes. The best candidates are selected based on this approximate distance. No post processing step with the exact ed distance is performed. To speed up, non exhaustive search was proposed in IVFADC: An inverted file is built containing the inverted list corresponding to each code. Only some inverted lists are searched: those corresponding to the w neighbors of the query code.

Answering a similarity query using an index typically involves two steps: a filtering step where the pre-built index is used to prune candidates and a refinement step where the surviving candidates are compared to the query in the original high dimensional space~\cite{conf/sigmod/Guttman1984,conf/vldb/Weber1998,conf/cikm/Hakan2000,journal/kais/Camerra2014,journal/vldb/Zoumpatianos2016,journal/edbt/Schafer2012,conf/vldb/Wang2013,conf/icmd/Beckmann1990,dpisax,ulisse}. Some exact~\cite{conf/icmd/Beckmann1990,journal/edbt/Schafer2012,conf/vldb/Weber1998,conf/cikm/Hakan2000} and approximate methods~\cite{conf/vldb/sun14,journal/pami/babenko15} first summarize the original data and then index these summarizations, while others tie together data reduction and indexing~\cite{journal/kais/Camerra2014,journal/vldb/Zoumpatianos2016,conf/vldb/Wang2013}.  Some approximate methods return the candidates obtained in the filtering step~\cite{journal/pami/babenko15}. There also exist exact~\cite{conf/vldb/Ciaccia1997} and approximate~\cite{journal/corr/malkov16} methods that index high dimensional data directly.

A variety of data structures exist for similarity search indexes, including trees~\cite{conf/sigmod/Guttman1984,conf/icmd/Beckmann1990,journal/kais/Camerra2014,journal/vldb/Zoumpatianos2016,conf/vldb/Wang2013,dpisax,ulisse,conf/vldb/sun14,journal/edbt/Schafer2012}, inverted indexes~\cite{journal/tpami/jegou2011,conf/icassp/jegou2011,journal/iccv/xia2013,journal/pami/babenko15}, filter files~\cite{conf/vldb/Weber1998,conf/cikm/Hakan2000,journal/vldb/Zoumpatianos2016}, hash tables~\cite{conf/stoc/indyk1998,conf/poccs/broder1997,conf/sigcg/datar2004,conf/stoc/charikar02,journal/nips/liu2004,conf/soda/panigrahy2006,journal/siamdm/motwani2007,conf/vldb/lv2007,conf/sigmod/gan2012,journal/atct/odonnell2014,qalsh} and graphs~\cite{conf/siam/arya1993,journal/apr/chavez2010,conf/sigkdd/aoyama2011,conf/iccv/wang2013,journal/is/malkov2014,conf/sisap/chavez2015,conf/icmm/jiang2016,journal/corr/malkov16}. 
%
%We note that all exact indexing methods depend on lower-bounding, since it allows indexes to prune the search space with the guarantee of no false dismissals~\cite{conf/sigmod/Faloutsos1994} (the DSTree index~\cite{conf/vldb/Wang2013} also supports an upper-bounding distance, but does not use it for similarity search).
%Metric indexes (such as the M-tree~\cite{conf/vldb/Ciaccia1997}) additionally require the distance measure triangle inequality to hold.
%Though, there exist (non-metric) indexes for data series that are based on distance measures that are not metrics~\cite{journal/kis/Keogh2005}. 
%
%we should say that some indexes index the whole data while others only a subset.
%
%In addition to indexing and sequential methods, there also exist alternative 
There also exist multi-step approaches, e.g., Stepwise~\cite{conf/kdd/Karras2011}, that transform and organize data according to a hierarchy of resolutions. % with search including multiple intermediate filtering steps.
% as levels are sequentially read one at a time.{ 	
%Stepwise is such a method~\cite{conf/kdd/Karras2011}, relying on Euclidean distance, and lower- and upper-bounding distances. 

Next, we outline the \emph{approximate} similarity search methods (refer also to Table~\ref{tab:multiprogram}) and their summarization techniques. 
(\emph{Exact} methods are detailed in~\cite{journal/pvldb/echihabi2018}).
%Table~\ref{tab:multiprogram} summarizes the properties of these algorithms. %, and Figure~\ref{fig:taxonomy} classifies them in a taxonomy according to our definitions of Section~\ref{sec:problem}.

\begin{table*}[tb]
		\caption{{\color{black}Similarity search methods used in this study} % (a comprehensive list of exact similarity search methods is included in our previous work~\cite{journal/pvldb/echihabi2018}). 
		("$\bullet$" indicates our modifications to original methods). {\color{black}All methods support in-memory data, but only methods ticked in last column support disk-resident data.}}
	{\small
		\centering
		%\hspace*{0.5cm}
		\begin{tabular*}{\linewidth}{|*{11}{c|}} 
			%    	\begin{tabular*}{\linewidth}{|*cc|c|c|c|c|c|c|c|c|c|c|}
			\cline{3-11} 
			\multicolumn{1}{c}{}& & \multicolumn{4}{c|}{Matching Accuracy}  & %\multicolumn{2}{c|}{Matching Type} &
			 \multicolumn{2}{|c|}{Representation} & \multicolumn{3}{c|}{Implementation}\\    		
			\cline{3-11} 
			\multicolumn{1}{c}{}& & exact & ng-appr. & $\epsilon$-appr. & $\delta$-$\epsilon$-appr. & Raw & Reduced & Original  & New & Disk-resident Data \\    		
			\cline{1-11}			 		 
			%\multicolumn{1}{|c|}{\multirow{6}{*}{\rotatebox[origin=c]{90}{Indexes}}}
			\cline{2-11}			 
			\multicolumn{1}{|c|}{\multirow{2}{*}{{Graphs}}}
			& \multicolumn{1}{c|}{HNSW} & 
			& \cite{journal/corr/malkov16} &   &  & \checkmark & & C++ &  &\\	
			\cline{2-11}			 	 
			& \multicolumn{1}{c|}{NSG} & 
			& \cite{nsg} &   &  & \checkmark & & C++ &  &\\	
			\cline{1-11}			 
			\multicolumn{1}{|c|}{\multirow{1}{*}{{Inv. Indexes}}}
			& \multicolumn{1}{c|}{IMI} &  & \cite{journal/pami/babenko15,journal/tpami/ge2014}  &  &  &  & OPQ & C++  & &\checkmark\\	
			\cline{1-11}			 	 
			\multicolumn{1}{|c|}{\multirow{2}{*}{{LSH}}}
			& \multicolumn{1}{c|}{QALSH} & &  & & \cite{qalsh} &   & Signatures & C++ & &\\	
			\cline{2-11}			 		 			
			& \multicolumn{1}{c|}{SRS} & & &  & \cite{conf/vldb/sun14} &  & Signatures & C++ & &\\				
			\cline{1-11}			 
			\multicolumn{1}{|c|}{\multirow{1}{*}{{Scans}}}
			& \multicolumn{1}{c|}{VA+file} & {\cite{conf/cikm/Hakan2000}} & $\bullet$&$\bullet$&$\bullet$ & & {DFT} & {MATLAB} & {C} & \checkmark\\	
			\cline{2-11}			 		 			
			\cline{1-11}			 
			\multicolumn{1}{|c|}{\multirow{4}{*}{{Trees}}}
			& \multicolumn{1}{c|}{Flann} & & \cite{flann} & &  & \checkmark  &  & C++ & &\\	
			\cline{2-11}			 
			& \multicolumn{1}{c|}{DSTree} &\cite{conf/vldb/Wang2013} & \cite{conf/vldb/Wang2013} &$\bullet$&$\bullet$&  & EAPCA  & Java & C & \checkmark\\	
			\cline{2-11}			 
			& \multicolumn{1}{c|}{HD-index} & & \cite{hdindex} & &  &   & Hilbert keys & C++ & & \checkmark\\	
			\cline{2-11}			 
			& \multicolumn{1}{c|}{iSAX2+} & \cite{journal/kais/Camerra2014} & \cite{journal/kais/Camerra2014} &$\bullet$&$\bullet$&  &  iSAX &  C\# & C & \checkmark\\	
			\cline{1-11}			 
			%& \multicolumn{1}{|c|}{M-tree} & \cite{conf/vldb/Ciaccia1997}
			%&  & \cite{conf/icde/Ciaccia2000} & \cite{conf/icde/Ciaccia2000} & \checkmark&  & \checkmark & & C++ &  \\	
			%\cline{2-12}			 	 
			%& \multicolumn{1}{|c|}{\color{red}{Faiss-IVF}} &  & \cite{journal/tpami/jegou2011,journal/tpami/ge2014}  &  &  & \checkmark&  &  & OPQ & C++  &\\	
			%& \multicolumn{1}{|c|}{R*-tree} & \cite{conf/icmd/Beckmann1990} &  &  &  & \checkmark&  &  & PAA & C++  &\\	
			%\cline{2-12}			 		 
			%& \multicolumn{1}{|c|}{SFA trie} &\cite{journal/edbt/Schafer2012} & \cite{journal/edbt/Schafer2012} &  &  & \checkmark & \checkmark & & SFA & Java & C\\	
			%\cline{2-12}			 		 
			%& \multicolumn{1}{|c|}{LSH} & &  &  & \cite{conf/vldb/Gionis1999} & \checkmark&   &  \checkmark&  & C++& \\	
			\cline{1-11}			 		 
			%\multicolumn{1}{|c|}{\multirow{3}{*}{\rotatebox[origin=c]{90}{ Other }}}
			%& \multicolumn{1}{|c|}{UCR Suite} & \cite{conf/kdd/Mueen2012} &  &  &  & $\bullet$& \checkmark &  \checkmark &  & C &\\	
			%\cline{2-12}			 		 
			%& \multicolumn{1}{|c|}{MASS} &\cite{journal/dmkd/Yeh2017} &  &   &  &$\bullet$& \checkmark &  & DFT &  C &\\	
			%\cline{2-12}			 		 
			%& \multicolumn{1}{|c|}{Stepwise} &\cite{conf/kdd/Karras2011} &  &  &  & \checkmark&  &   & DHWT & C & \\	
			%\cline{1-12}			 		 
		\end{tabular*}
	} % font size
	\label{tab:multiprogram}
\end{table*}













\subsection{Summarization Techniques}

\noindent\textbf{Random projections} (used by SRS~\cite{conf/vldb/sun14}) reduce the original high dimensional data into a lower dimensional space by multiplying it with a random matrix. 
The Johnson-Lindenstrauss (JL) Lemma~\cite{conf/map/johnson84} guarantees that if the projected space has a large enough number of dimensions, there is a high probability that the pairwise distances are preserved, with a distortion not exceeding $(1+\epsilon)$.

%The similarity search methods surveyed in this paper use the following summarization techniques: {\it Discrete Haar Wavelet Transforms} (DHWT)~\cite{conf/icde/Chan1999}, {\it Discrete Fourier Transforms} (DFT)~\cite{conf/fodo/Agrawal1993, conf/sigmod/Faloutsos1994, conf/sigmod/Rafiei1997, journal/corr/Rafiei1998}, {\it Symbolic Fourier Approximation} (SFA) \cite{journal/edbt/Schafer2012},  {\it Piecewise Aggregate Approximation} (PAA)~\cite{journal/kais/Keogh2001}, {\it Symbolic Aggregate approXimation} (SAX)~\cite{journal/dmkd/Lin2007}, {\it indexable Symbolic Aggregate Approximation} (iSAX)~\cite{conf/kdd/Shieh2008},  {\it Adaptive Piecewise Constant Approximation} (APCA)~\cite{journal/acds/Chakrabarti2002} and {\it Extended Adaptive Piecewise Approximation} (EAPCA)~\cite{conf/vldb/Wang2013}. 
%Due to limited space, we will not describe these techniques in detail. 

%Recall that a raw data series $DS$ of length $n$ consists of a sequence of $n$ real values, aka $n$ dimensions. To alleviate the curse of dimensionality, most methods rely on summarization techniques to reduce the high-dimensional raw data series $DS$ into a data series $DSS$ of lower dimension $l$. 

%In order to get a deeper insight into the differences and similarities of the surveyed methods, we use an intuitive formulation to describe the dimensionality reduction techniques. Given a data series $DS$ of length $n$, and its summarization $DSS$ of length $l$, each dimension $C_i$ in $DSS$ such that $1 \leq i \leq l$ can be represented using a pair $(x_i, y_i)$ in $\mathbb{R}^2$. 

%The {\it Discrete Haar Wavelet Transform} (DHWT)~\cite{conf/icde/Chan1999} uses the Haar wavelet decomposition to transform each data series $S$ into a multi-level hierarchical structure.
%% called the Haar or error tree. The length of the input data series has to be an integer power of 2. of a length Each node in the tree contains a wavelet coefficient $c$. The coefficient at the root is the average of the values of $DS$. The average at each node of the tree is obtained by adding the node's coefficient $c$ to the average at the parent node $avg$ if the node is to the left or subtracting $c$ from $avg$ if the node is to the right. The original values of $DS$ are reconstructed by adding all the signed coefficients from root to leaf, where each value corresponds to one leaf. 
%Resulting summarizations are composed of the first $l$ coefficients. 

\noindent{\bf Piecewise Aggregate Approximation} (PAA)~\cite{journal/kais/Keogh2001} and {\it Adaptive Piecewise Constant Approximation} (APCA)~\cite{journal/acds/Chakrabarti2002} are segmentation techniques that approximate a data series $S$ using $l$ segments (of equal/arbitrary length, respectively). The approximation represents each segment with the mean value of its points. 
%Each dimension $C_i$ of $DSS$ corresponds to the mean of the values of the $i$th segment in $DS$. The difference between PAA and APCA is that the former divides $DS$ into equi-length segments, whereas the segments in the latter can be of different lengths. Given two data series $DS$ and $DS'$, their approximations $DSS$ and $DSS'$ will have the same number of segments, so segments $C_i$ and $C'_i$ may have a different values for $x_i$ but will keep the same value for $y_i$.
%For the PAA and APCA techniques, $x_i$ is the right endpoint of the $i$th segment of $DS$ and $y_i$ is the mean of the values in the $i$th segment of $DS$. 
The {\it Extended APCA} (EAPCA)~\cite{conf/vldb/Wang2013} technique extends APCA by representing each segment with both the mean and the standard deviation.
%$y_i$ and right endpoint of the $i$th segment $x_i$, it adds 
%$y'_i$ of the $i$th segment. So, instead of the pair $(x_i, y_i)$, a triplet $(x_i,y_i,y'_i)$ is used to summarize each segment. The EAPCA segmentation is dynamic because the number, length and right endpoints of the segments can vary for different data series. We will detail this further in the next section when we describe the DSTree. 
%It was pointed out in that the PAA and Haar wavelets
\begin{comment}
In order to get a deeper insight into the differences and similarities of the surveyed methods, in particular SFA, SAX and EAPCA, we devise an intuitive formulation which we will use to describe the DFT, PAA and APCA techniques. Given a data series $DS$ of length $n$, and its summarization $DSS$ of length $l$, each dimension $C_i$ in $DSS$ such that $1 \leq i \leq l$ can be mapped to a point $P_i(x_i, y_i)$ in $\mathbb{R}^2$. For DFT, $x_i$ is the endpoint of the $i$th unit segment on the x-axis and $y_i$ is the value of the $i$th fourier coefficient. For the PAA and APCA techniques, $x_i$ is the right endpoint of the $i$th segment of $DS$ and $y_i$ is the mean of the values in the $i$th segment of $DS$. 
\end{comment}
%the next definition extends the APCA definitions in 
% into $DSS$, 

\noindent\textbf{Quantization} 
%The three summarization techniques below are all based on quantization, which 
is a lossy compression process that maps a set of infinite numbers to a finite set of codewords that together constitute the codebook. 
%We identify three main types of quantization: scalar, vector and product quantization. 
A \emph{scalar} quantizer operates on the individual dimensions of a vector independently, whereas a \emph{vector} quantizer considers the vector as a whole (leveraging the correlation between dimensions~\cite{journal/tit/gray1998}). 
The size $k$ of a codebook increases exponentially with the number of bits allocated for each code. 
%A larger $k$ reduces information loss but requires larger storage and more processing time. 
A \emph{product} quantizer~\cite{journal/tpami/jegou2011} 
%has been proposed to address these limitations by splitting 
splits
the original vector of dimension $d$ into $m$ smaller subvectors, on which a lower-complexity vector quantization is performed. 
The codebook then consists of the cartesian product of the codebooks of the $m$ subquantizers. 
Scalar and vector quantization are special cases of product quantization, where $m$ is equal to $d$ and 1, respectively.

%\begin{compactitem}
%\item 
\noindent(i) 
{\it Optimized Product Quantization} (OPQ) (used by IMI~\cite{journal/tpami/ge2014}) improves the accuracy of the original product quantizer~\cite{journal/tpami/jegou2011} by adding a preprocessing step consisting of a linear transformation of the original vectors, which decorrelates the dimensions and optimizes space decomposition. {\color{black} A similar quantization technique, CK-Means, was proposed in~\cite{ck-means} but OPQ is considered the state-of-the-art~\cite{conf/CVPR/kalantidis2014,journal/ite/matsui2018}}. 

%\item 
\noindent(ii) 
The {\it Symbolic Aggregate Approximation} (SAX)~\cite{conf/dmkd/LinKLC03} technique starts by transforming the data series into $l$ real values using PAA, and then applies a \emph{scalar} quantization technique to represent the PAA values 
%$y_i$ as a discrete value instead of a real value for a smaller footprint. 
using discrete symbols forming an alphabet of size $a$, called the cardinality of SAX. 
The $l$ symbols form the SAX representation. %, which consists of only a few bits.
%The discretization uses a heuristic to place $b$ breakpoints on the y-axis. If we draw lines parallel to the x-axis and passing through each of the breakpoints, we can see that the space is divided into $a$ regions where $a = b+1$. Each region is labeled using a discrete alphabet of size $a$, $a$ is also called the cardinality of SAX. The $y_i$ of each dimension $C_i$ is then mapped to a discrete value from this alphabet. 
%The $i$SAX (indexable SAX)~\cite{conf/kdd/Shieh2008} representation extends SAX can have an arbitrary alphabet size for each segment.
%. and each symbol is represented using the same cardinality, that is the alphabet size is fixed.
The $i$SAX~\cite{conf/kdd/Shieh2008} technique %extends SAX to 
allows comparisons of SAX representations of different cardinalities, which makes SAX indexable. 
%An iSAX word is represented using a higher resolution by adding a breakpoint on the y-axis. The higher resolution word will also consist of $l$ symbols but the alphabet size available for a given symbol will be larger.  The maximum resolution of a SAX word uses the maximum alphabet size which is 256.

%Similarly to SAX, the {\it Symbolic Fourier Approximation} (SFA) \cite{journal/edbt/Schafer2012} is also a symbolic approach. 
%However, instead of PAA, it first transforms $S$ %into $DSS$ 
%into $l$ DFT coefficients using FFT (or MFT for subsequence matching), then extends the discretization principle of SAX to support both equi-depth and equi-width binning, and to allow each dimension to have its own breakpoints. 
%An SFA 
%%word 
%summary consists of $l$ symbols.

%\item 
\noindent(iii) \textit{The Karhunen-Lo\`{e}ve transform (KLT).} 
The original VA+file method~\cite{conf/cikm/Hakan2000} first converts a data series $S$ of length $n$ using KLT into $n$ real values to de-correlate the data, then applies a \emph{scalar} quantizer to encode the real values as discrete symbols.
%Each dimension is allocated a certain number of bits and decision intervals for each dimension are determined using the k-means algorithm. 
As {\color{black}we} will explain in the next subsection, for efficiency considerations, we altered the VA+file to use the {\it Discrete Fourier Transform} (DFT) instead of KLT. DFT~\cite{conf/fodo/Agrawal1993,conf/sigmod/Faloutsos1994,conf/sigmod/Rafiei1997,journal/corr/Rafiei1998} 
approximates a data series using $l$ frequency coefficients, 
%transforms a data series $S$ into frequency coefficients.
%%parts, without keeping information about time or space. Each part corresponds to one DFT coefficient. 
%The data series $S$ is then approximated using a subset of $l$ coefficients. 
%There exist different implementation algorithms for the DFT, we chose the Fast Fourier Transform (FFT) algorithm, since it is 
% to compute DFT transforms since it is the most 
and can be efficiently implemented with Fast Fourier Transform (FFT), which is 
%which is 
optimal for whole matching (alternatively, the MFT algorithm~\cite{conf/icdsp/Albrecht1997} is adapted to subsequence matching %as it computes the DFT transform using sliding windows).
since it uses sliding windows).
%In the case of both DHWT and DFT, for each dimension of $DSS$, $x_i$ is equal to $i$ and $y_i$ is the value of the $i$th coefficient. 
%\end{compactitem}


\begin{comment}
\begin{figure}[tb]
	\captionsetup{justification=centering}
	\includegraphics[scale =0.70]{{summarizations}}
	\caption{Summarizations}
	%	}
	\label{fig:summarizations}
\end{figure}
\end{comment}

\begin{comment}
    \begin{table*}[tb]
    {\small
    	\centering
    	\hspace*{1cm}
    	\begin{tabular*}{\linewidth}{|*{12}{c|}} 
    		%    	\begin{tabular*}{\linewidth}{|*cc|c|c|c|c|c|c|c|c|c|c|}
    		\cline{3-12} 
    		\multicolumn{1}{c}{}& & \multicolumn{4}{c|}{Matching Accuracy}  & \multicolumn{2}{c|}{Matching Type} & \multicolumn{2}{|c|}{Representation} & \multicolumn{2}{c|}{Implementation}\\    		
    		\cline{3-12} 
    		\multicolumn{1}{c}{}& & exact & ng-appr. & $\epsilon$-appr. &$\delta$-$\epsilon$-appr. & Whole & Subseq. & Raw & Reduced & Original  & New \\    		
    		\cline{1-12}			 		 
    		%			\multirow{2}{*}{\rotatebox[origin=c]{90}{Indexes}} 
    		\multicolumn{1}{|c|}{\multirow{5}{*}{\rotatebox[origin=c]{90}{Indexes}}}
    		& \multicolumn{1}{|c|}{ADS+} &\cite{journal/vldb/Zoumpatianos2016} &\cite{journal/vldb/Zoumpatianos2016} &  &  &  \checkmark &  &  & iSAX &  C &\\	
    		\cline{2-12}			 		 
    		& \multicolumn{1}{|c|}{DSTree} &\cite{conf/vldb/Wang2013} & \cite{conf/vldb/Wang2013} &  &  &  \checkmark &   & & EAPCA & Java & C\\	
    		\cline{2-12}			 		 
    		& \multicolumn{1}{|c|}{iSAX2+} & \cite{journal/kais/Camerra2014} & \cite{journal/kais/Camerra2014} &  &  & \checkmark & &  &  iSAX &  C\# & C\\	
    		\cline{2-12}			 	 
    		& \multicolumn{1}{|c|}{M-tree} & \cite{conf/vldb/Ciaccia1997}
    		 &  & \cite{conf/icde/Ciaccia2000} & \cite{conf/icde/Ciaccia2000} & \checkmark&  & \checkmark & & C++ &  \\	
    		\cline{2-12}			 		 
    		& \multicolumn{1}{|c|}{R*-tree} & \cite{conf/icmd/Beckmann1990} &  &  &  & \checkmark&  &  & PAA & C++  &\\	
    		\cline{2-12}			 		 
    		& \multicolumn{1}{|c|}{SFA trie} &\cite{journal/edbt/Schafer2012} & \cite{journal/edbt/Schafer2012} &  &  & \checkmark & \checkmark & & SFA & Java & C\\	
    		\cline{2-12}			 		 
    		& \multicolumn{1}{|c|}{LSH} & &  &  & \cite{conf/vldb/Gionis1999} & \checkmark&   &  \checkmark&  & C++& \\	
    		\cline{1-12}			 		 
    		\multicolumn{1}{|c|}{\multirow{3}{*}{\rotatebox[origin=c]{90}{ Other }}}
    		& \multicolumn{1}{|c|}{UCR Suite} & \cite{conf/kdd/Mueen2012} &  &  &  &  & \checkmark &  \checkmark &  & C &\\	
    		\cline{2-12}			 		 
    		& \multicolumn{1}{|c|}{MASS} &\cite{journal/dmkd/Yeh2017} &  &   &  & \checkmark & \checkmark &  & DFT &  C &\\	
    		\cline{2-12}			 		 
    		& \multicolumn{1}{|c|}{Stepwise} &\cite{conf/kdd/Karras2011} &  &  &  & \checkmark&  &   & DHWT & C & \\	
    		\cline{2-12}			 		 
    		& \multicolumn{1}{|c|}{CSZ} & &  &  & \cite{conf/kdd/ColeSZ05}  & \checkmark &  &   & Sketches & Python & \\	
	        \cline{1-12}			 		 
    		\end{tabular*}
    		} % font size
    	\caption{Similarity Search Methods}
    	\label{tab:multiprogram}
    	\end{table*}
    			
\begin{figure}[!htb]
	\captionsetup{justification=centering}
	\includegraphics[width =\columnwidth]{{summarizations}}
	\caption{Summarizations}
	%	}
	\label{fig:summarizations}
\end{figure}

\end{comment}

\begin{comment}
\begin{table*}[tb]
	{\small
		\centering
		\hspace*{1cm}
		\begin{tabular*}{\linewidth}{|*{12}{c|}} 
			%    	\begin{tabular*}{\linewidth}{|*cc|c|c|c|c|c|c|c|c|c|c|}
			\cline{3-12} 
			\multicolumn{1}{c}{}& & \multicolumn{4}{c|}{Matching Accuracy}  & \multicolumn{2}{c|}{Matching Type} & \multicolumn{2}{|c|}{Representation} & \multicolumn{2}{c|}{Implementation}\\    		
			\cline{3-12} 
			\multicolumn{1}{c}{}& & exact & ng-appr. & $\epsilon$-appr. & $\delta$-$\epsilon$-appr. & Whole & Subseq. & Raw & Reduced & Original  & New \\    		
			\cline{1-12}			 		 
			%			\multirow{2}{*}{\rotateTbox[origin=c]{90}{Indexes}} 
			\multicolumn{1}{|c|}{\multirow{6}{*}{\rotatebox[origin=c]{90}{Indexes}}}
			& \multicolumn{1}{|c|}{ADS+} &\cite{journal/vldb/Zoumpatianos2016} &\cite{journal/vldb/Zoumpatianos2016} & $\bullet$ & $\bullet$ &  \checkmark &  &  & iSAX &  C &\\	
			%\cline{2-12}			 		 
		 	%& \multicolumn{1}{|c|}{Coconut} &\cite{journal/pvldb/kondylakis18} & \cite{journal/pvldb/kondylakis18} &  &  &  \checkmar &   & & EAPCA & C++ & \\	
		 	\cline{2-12}			 		 		& \multicolumn{1}{|c|}{DSTree} &\cite{conf/vldb/Wang2013} & \cite{conf/vldb/Wang2013} &$\bullet$&$\bullet$&  \checkmark &   & & EAPCA & Java & C\\	
			\cline{2-12}			 & \multicolumn{1}{|c|}{iSAX2+} & \cite{journal/kais/Camerra2014} & \cite{journal/kais/Camerra2014} &$\bullet$&$\bullet$& \checkmark & &  &  iSAX &  C\# & C\\	
			\cline{2-12}			 	 
			& \multicolumn{1}{|c|}{HNSW} & 
			& \cite{journal/corr/malkov16} &   &  & \checkmark&  & \checkmark & & C++ &  \\	
			\cline{2-12}			 	 
			& \multicolumn{1}{|c|}{M-tree} & \cite{conf/vldb/Ciaccia1997}
			&  & \cite{conf/icde/Ciaccia2000} & \cite{conf/icde/Ciaccia2000} & \checkmark&  & \checkmark & & C++ &  \\	
			\cline{2-12}			 	 
			%& \multicolumn{1}{|c|}{\color{red}{Faiss-IVF}} &  & \cite{journal/tpami/jegou2011,journal/tpami/ge2014}  &  &  & \checkmark&  &  & OPQ & C++  &\\	
			& \multicolumn{1}{|c|}{IMI} &  & \cite{journal/pami/babenko15,journal/tpami/ge2014}  &  &  & \checkmark&  &  & OPQ & C++  &\\	
			\cline{2-12}			 		 
			& \multicolumn{1}{|c|}{R*-tree} & \cite{conf/icmd/Beckmann1990} &  &  &  & \checkmark&  &  & PAA & C++  &\\	
			\cline{2-12}			 		 
			& \multicolumn{1}{|c|}{SFA trie} &\cite{journal/edbt/Schafer2012} & \cite{journal/edbt/Schafer2012} &  &  & \checkmark & \checkmark & & SFA & Java & C\\	
			\cline{2-12}			 		 
			& \multicolumn{1}{|c|}{SRS} & & &\cite{conf/vldb/sun14}  & \cite{conf/vldb/sun14} & \checkmark &  &  & Signatures & C++ & \\				
			\cline{2-12}			 		 
			& \multicolumn{1}{|c|}{VA+file} & {\cite{conf/cikm/Hakan2000}} & $\bullet$&$\bullet$&$\bullet$& {\checkmark} &  & & {DFT} & {MATLAB} & {C}\\	
			\cline{2-12}			 		 
			%& \multicolumn{1}{|c|}{LSH} & &  &  & \cite{conf/vldb/Gionis1999} & \checkmark&   &  \checkmark&  & C++& \\	
			\cline{1-12}			 		 
			\multicolumn{1}{|c|}{\multirow{3}{*}{\rotatebox[origin=c]{90}{ Other }}}
			& \multicolumn{1}{|c|}{UCR Suite} & \cite{conf/kdd/Mueen2012} &  &  &  & $\bullet$& \checkmark &  \checkmark &  & C &\\	
			\cline{2-12}			 		 
			& \multicolumn{1}{|c|}{MASS} &\cite{journal/dmkd/Yeh2017} &  &   &  &$\bullet$& \checkmark &  & DFT &  C &\\	
			\cline{2-12}			 		 
			& \multicolumn{1}{|c|}{Stepwise} &\cite{conf/kdd/Karras2011} &  &  &  & \checkmark&  &   & DHWT & C & \\	
			\cline{1-12}			 		 
		\end{tabular*}
	} % font size
	\caption{Similarity search methods used in this study (a comprehensive list of exact similarity search methods is included in our previous work~\cite{journal/pvldb/echihabi2018}). The "$\bullet$" indicates our modifications to original methods.}
	\label{tab:multiprogram}
\end{table*}
\end{comment}
















\vspace*{-0.3cm}

\subsection{Approximate Similarity Search Methods}

There exist several techniques for approximate similarity search~\cite{conf/stoc/indyk1998,conf/vldb/Gionis1999, journal/jda/bustos2004,conf/icde/houle2005,conf/kdd/ColeSZ05,journal/tpami/chavez2008,conf/icsis/amato2008,conf/sisap/tellez2011,conf/vldb/sun14,journal/tpami/ge2014,journal/corr/malkov16,conf/cvpr/yandex16}  {\color{black}\cite{conf/sigmod/berchtold1998,conf/pods/ooi2000,conf/vldb/yu2001}}. 
We focus on the {\color{black}7} most prominent techniques designed for multidimensional data, and we also describe the approximate search algorithms designed specifically for data series. 
We also propose a new set of techniques that can answer $\delta$-$\epsilon$-approximate queries based on modifications to existing exact similarity methods for data series.

%In Table~\ref{tab:multiprogram}, we categorize LSH, a state of the art approximate algorithm.
%A thorough evaluation of all approximate methods deserves a study on its own, and we defer it to future work.

%A thorough evaluation of all approximate methods deserves a study on its own, and we defer it to future work.
















\subsubsection{State-of-the-Art for Multidimensional Vectors}

\begin{comment}

{\color{red}
	$\epsilon$-approximate
	S. Arya, D. Mount, Approximate nearest neighbor queries in fixed
	dimensions, in: Proceedings of the fourth annual ACM-SIAM Sym-
	posium on Discrete algorithms, SODA'93, Philadelphia, PA, USA,
	1993.
	[20] P. Indyk, R. Motwani, Approximate nearest neighbors: towards
	removing the curse of dimensionality, in: Proceedings of STOC'98,
	New York, USA, 1998.
	[21] E., Kushilevitz, R. Ostrovsky, Y. Rabani. Efficient search for approx-
	imate nearest neighbor in high dimensional spaces, in: Proceedings
	of STOC'98, New York, USA 1998.
	[22] P., Haghani, S. Michel, K. Aberer, Distributed similarity search in high
	dimensions using locality sensitive hashing, in: Proceedings of the
	12th International Conference on Extending Database Technology:
	Advances in Database Technology, New York, USA, 2009, pp. 744
	–
	755.
	[23] S. Arya, D.M. Mount, N.S. Netanyahu, R. Silverman, A. Wu. An
	optimal algorithm for approximate nearest neighbor searching. in:
	Proceedings of the Fifth Annual ACM-SIAM Symposium on Discrete
	Algorithms, Society for Industrial and Applied Mathematics, 19
	
	
	$\delta-0$-approximate
	
	
	E. Chávez, K. Figueroa, G. Navarro, Effective proximity retrieval by
	ordering permutations, IEEE Trans. Pattern Anal. Mach. Intell. 30 (9)
	(2008) 1647
	–
	1658
	.
	[25] E.S. Tellez, E. Chávez, G. Navarro, Succinct nearest neighbor search,
	in: Proceedings of SISAP, 2011.
	[26] M.E., Houle, J. Sakuma. Fast approximate similarity search in
	extremely high-dimensional data sets, in: Proceedings of ICDE 2005.
	[27]
	E. Chávez, E. Sadit Tellez, Navigating k-nearest neighbor graphs to
	solve nearest neighbor searches, Adv. Pattern Recog. (2010) 270
	–
	280
	.
	[28]
	K.L. Clarkson, Nearest neighbor queries in metric spaces, Discrete
	Comput. Geometry 22 (1) (1999) 63
	–
	93
	.
	[29]
	B. Bustos, G. Navarro, Probabilistic proximity searching algorithms
	based on compact partitions, J. Discrete Algorithms 2 (1) (2004)
	115
	–
	134
	.
	[30] E. Chávez, G. Navarro, A probabilistic spell for the curse of dimen-
	sionality, in: Proceedings of the Algorithm Engineering and Experi-
	mentation, Springer, 2001 pp. 147
	–
	160.
	[31]
	E. Chávez, G. Navarro, Probabilistic proximity search: fighting the
	curse of dimensionality in metric spaces, Inf. Process. Lett. 85 (1)
	(2003) 39
	–
	46
	
	
	
	Reviews:
	
	Approximate similarity search: A multi-faceted problem
	Author links open overlay panelMarcoPatella
	PaoloCiaccia
	
	On nonmetric similarity search problems in complex domains.
	
}
\end{comment}

{\color {black} \noindent{\bf Flann}~\cite{flann} is an in-memory ensemble technique for $ng$-approximate nearest neighbor search in high-dimensional spaces. Given a dataset and a desired search accuracy, Flann selects and auto-tunes the most appropriate algorithm among randomized kd-trees~\cite{random-kd-trees} and a new proposed approach based on hierarchical k-means trees~\cite{flann}.}

{\color {black} \noindent{\bf HD-index}~\cite{hdindex} is an $ng$-approximate nearest neighbor technique that partitions the original space into disjoint partitions of lower dimensionality, then represents each partition by an RBD tree (modified B+tree with leaves containing distances of data objects to reference objects) built on the Hilbert keys of data objects. A query $Q$ is partitioned according to the same scheme, % as the dataset, 
searching the hilbert key of $Q$ in the RDB tree of each partition, then refining the candidates first using approximate distances based on triangular and Ptolemaic inequalities then using the real distances.}

\noindent{\bf HNSW}. 
HNSW~\cite{journal/corr/malkov16} is an {\color{black} in-memory} $ng$-approximate method that belongs to the class of proximity graphs that exploit two fundamental geometric structures: the Voronoi Diagram (VD) and the Delaunay Triangulation (DT). 
A VD is obtained when a given space is decomposed using a finite number of points, called \emph{sites}, into regions such that each site is associated with a region consisting of all points that are closer to it than to any other site. 
The DT is the dual of the VD.
It is constructed by connecting sites with an edge if their regions share a side. %, and thus guarantees that given a query $S_Q$, and any starting vertex, a greedy traversal of the graph, selecting at each step the next adjacent vertex closest to $S_Q$, finds the nearest neighbor of $S_Q$~\cite{journal/corr/malkov16}. 
Since constructing a DT for a generic metric space is not always possible (except if the DT is the complete graph)~\cite{journal/vldbj/navarro2002}, proximity graphs, which approximate the DT by conserving only certain edges, have been proposed~\cite{conf/siam/arya1993,journal/apr/chavez2010,conf/sigkdd/aoyama2011,conf/iccv/wang2013,journal/is/malkov2014,conf/sisap/chavez2015,conf/icmm/jiang2016,journal/corr/malkov16}. 
A k-NN graph is a proximity graph, where only the links to the closest neighbors are preserved. 
Such graphs suffer from two limitations: (i) the curse of dimensionality; and (ii) the poor performance on clustered data (the graph has a high probability of being disconnected). 
To address these limitations, the Navigable Small World (NSW) method~\cite{journal/is/malkov2014} proposed to heuristically augment the approximation of the DT with long range links to satisfy 
%(in a generic metric space), 
the small world navigation properties~\cite{conf/stoc/kleinberg2000}.  
%for the Euclidean space, to ensure global graph connectivity and poly-logarithmic scalability of the greedy search. 
The HNSW graph~\cite{journal/corr/malkov16} improves the search efficiency of NSW 
%to logarithmic scale 
by organizing the links in hierarchical layers according to their lengths. 
Search starts at the top layer, which contains only the longest links, and proceeds down the hierarchy. % once the local minimum is found at each layer. 
%We select the 
HNSW %graph to represent graph-based methods since it 
is considered the state-of-the-art~\cite{conf/sisap/martin17}. 

{\color {black} \noindent{\bf NSG}~\cite{nsg} is a recent in-memory proximity graph approach that approximates a graph structure called MRNG~\cite{nsg} which belongs to the class of Monotonic Search Networks (MSNET). 
%The MSNET family guarantees that there exists a monotonic path between any two nodes and includes the DT structure discussed earlier. However, 
Building an MRNG graph for large datasets becomes impractical; that is why the state-of-the-art techniques approximate it. NSG approximates the MRNG graph by relaxing the monotonicity requirement and edge selection strategy, and dropping the longest edges in the graph.}


%\paragraph*{\textbf{LSH}}


%\noindent{\bf SRS.} SRS belongs to the LSH family~\cite{journal/corr/andoni2018}. 
%LSH encompasses a class of randomized algorithms that solve the $\delta$-$\epsilon$-approximate nearest neighbor problem in sub-linear time, for $\delta < 1 $. 
%The main intuition 
%is that two points that are nearby in a high dimensional space, will remain nearby when projected to a lower dimensional space~\cite{conf/stoc/indyk1998}. 
%LSH techniques project points using hash functions, which guarantee that only nearby points are likely to be mapped to the same bucket. 
%Given a dataset $\mathbb{S}$ and a query $S_Q$, $L$ hash functions are applied to all points in $\mathbb{S}$ and to the query $S_Q$. Only points that fall at least once in the same bucket as $S_Q$, in each of the $L$ hash tables, are further processed in a linear scan to find the $\delta$-$\epsilon$-approximate nearest-neighbor. 
%There exist many variants of LSH, either proposing different hash functions to support particular similarity measures~\cite{conf/poccs/broder1997,conf/sigcg/datar2004,conf/stoc/charikar02,conf/sigmod/gan2012}, 
%%, for instance, min-hash for Jaccard coefficient~\cite{conf/poccs/broder1997}, p-stable distributions for $L_p$ distances~\cite{conf/sigcg/datar2004}, sim-hash~\cite{conf/stoc/charikar02}, or cosine similarity~\cite{conf/sigmod/gan2012}, 
%or improving the theoretical bounds on query accuracy (i.e., $\delta$ or $\epsilon$), query efficiency or the index size~\cite{journal/nips/liu2004,conf/soda/panigrahy2006,journal/siamdm/motwani2007,conf/vldb/lv2007,conf/sigmod/gan2012,journal/atct/odonnell2014,conf/vldb/sun14,qalsh}. 
%%While LSH-based methods enjoy sound theoretical guarantees, they are not scalable since the size of the index grows beyond the dataset size~\cite{conf/poccs/broder1997}. 
%In this work, we use the state-of-the-art LSH-based index,  SRS~\cite{conf/vldb/sun14}, which answers $\delta$-$\epsilon$-approximate queries using size linear to the dataset size, while empirically outperforming other LSH methods (with size super-linear to the dataset size~\cite{conf/poccs/broder1997}). 
%%A comprehensive survey on LSH methods can be found in~\cite{journal/corr/andoni2018}.

%% 
%%SRS: We are using SRS with the following settings:
%%-SRS-2, use first stop condition based on probability threshold and not the number of data points processed. We also modified it to compile with -O2 instead of O3.
%%Mention the JL lemma: requirement (m >= 4*log(n)/...)
%%(LSH)
%%maybe C2LSH (more efficient but high memory footprint)

%%It supports randomized exact and randomized approximate near-neighbor queries. 
%%Note that a near-neighbor query is different from a nearest-neighbor query. 
%%Given $r \in \mathbb{R}_{>0}$ and a query $S_Q$, the point $S_C$ is an $r$-near-neighbor of $S_Q$ if the distance between the two points is less than $r$. 
%%Given $r > 0$, $c > 1$ and $ 0 \leq \delta \leq 1$, $S_C$ is the $c$-approximate $r$-near-neighbor of $S_Q$ if their distance is at most $c$ times $r$, with probability at least $\delta$. Although different, nearest neighbor search algorithms are related as follows: 
%%1) the nearest-neighbor algorithm answers the near-neighbor search simply by verifying that the reported nearest neighbor is within distance $r$ from $S_Q$; 2) 
%%Near-neighbor search can also answer nearest-neighbor queries, 
%%within theoretical bounds, if run 
%%by running the search several times with increasing values of $r$~\cite{journal/cacm/andoni2008}.


\noindent{\bf IMI}. Among the different quantization-based inverted indexes proposed in the literature~\cite{journal/tpami/jegou2011,conf/icassp/jegou2011,journal/iccv/xia2013,journal/pami/babenko15}, IMI~\cite{journal/tpami/ge2014,journal/pami/babenko15} is considered the state-of-the-art~\cite{journal/ite/matsui2018}. 
This class of techniques builds an inverted index storing the list of data points that lie in the proximity of each codeword. 
The codebook
%, consisting of all the codewords, 
is the set of representative points obtained by performing clustering on the original data. When a query arrives, the $ng$-approximate search algorithm returns the list of all points corresponding to the closest codeword (or list of codewords).

{\color{black} \noindent{\bf LSH.} The LSH family}~\cite{journal/corr/andoni2018} encompasses a class of randomized algorithms that solve the $\delta$-$\epsilon$-approximate nearest neighbor problem in sub-linear time, for $\delta < 1 $. 
The main intuition 
is that two points that are nearby in a high dimensional space, will remain nearby when projected to a lower dimensional space~\cite{conf/stoc/indyk1998}. 
LSH techniques {\color{black}partition points into buckets} using hash functions, which guarantee that only nearby points are likely to be mapped to the same bucket. 
Given a dataset $\mathbb{S}$ and a query $S_Q$, $L$ hash functions are applied to all points in $\mathbb{S}$ and to the query $S_Q$. Only points that fall at least once in the same bucket as $S_Q$, in each of the $L$ hash tables, are further processed in a linear scan to find the $\delta$-$\epsilon$-approximate nearest-neighbor. 
There exist many variants of LSH, either proposing different hash functions to support particular similarity measures~\cite{conf/poccs/broder1997,conf/sigcg/datar2004,conf/stoc/charikar02,conf/sigmod/gan2012}, 
%, for instance, min-hash for Jaccard coefficient~\cite{conf/poccs/broder1997}, p-stable distributions for $L_p$ distances~\cite{conf/sigcg/datar2004}, sim-hash~\cite{conf/stoc/charikar02}, or cosine similarity~\cite{conf/sigmod/gan2012}, 
or improving the theoretical bounds on query accuracy (i.e., $\delta$ or $\epsilon$), query efficiency or the index size~\cite{journal/nips/liu2004,conf/soda/panigrahy2006,journal/siamdm/motwani2007,conf/vldb/lv2007,conf/sigmod/gan2012,journal/atct/odonnell2014,conf/vldb/sun14,qalsh} {\color{black}~\cite{sk-lsh}}. 
%While LSH-based methods enjoy sound theoretical guarantees, they are not scalable since the size of the index grows beyond the dataset size~\cite{conf/poccs/broder1997}. 
{\color{black}In this work, we select SRS~\cite{conf/vldb/sun14} and QALSH~\cite{qalsh} to represent the class of LSH techniques because they are considered the state-of-the-art in terms of footprint and accuracy, respectively~\cite{hdindex}}. SRS answers $\delta$-$\epsilon$-approximate queries using size linear to the dataset size, while empirically outperforming other LSH methods (with size super-linear to the dataset size~\cite{conf/poccs/broder1997}). 
%A comprehensive survey on LSH methods can be found in~\cite{journal/corr/andoni2018}.
{\color{black} QALSH is a query-aware LSH technique that partitions points into buckets using the query as anchor. Other LSH methods typically partition data points before a query arrives, using a random projection followed by a random shift. QALSH, does not perform the second step until a query arrives, thus improving the likelihood that points similar to the query are mapped to the same bucket.}


 \begin{comment}
%\paragraph*{\textbf{CSZ}}
\noindent{\bf CSZ.} The CSZ method\footnote{Since this method was not explicitly named in the original paper, we will refer to it as \emph{CSZ} (from the authors' initials).}~\cite{conf/kdd/ColeSZ05} is a $\delta$-$\epsilon$-approximate  similarity search method based on sketches that finds the most correlated equi-length windows in a set of uncooperative, i.e., noisy time series. Given two windows of equal length, $w_1$ and $w_2$, CSZ computes the similarity between the entire windows $w_1$ and $w_2$ (whole matching). It approximates $w_1$ and $w_2$ with $2b+1$ random projections using $d$ random vectors. The median of the Euclidean distances between these projections approximates the exact distance between $w_1$ and $w_2$. In particular, the approximate distance is $\delta$-$\epsilon$ approximate, where $\epsilon$ is a function of $d$ and $\delta = 1-(1/2)^b$~\cite{conf/kdd/ColeSZ05}.
\end{comment}



















\subsubsection{State-of-the-Art for Data Series}
While a number of data series methods support approximate similarity search ~\cite{conf/icde/shatkay1996,conf/kdd/Keogh1997,conf/ssdm/Wang2000,conf/kdd/ColeSZ05,conf/icdm/Camerra2010,journal/edbt/Schafer2012,conf/vldb/Wang2013,journal/kais/Camerra2014,journal/vldb/Zoumpatianos2016}, we focus on those that fit the scope of this study, i.e., methods that support out-of-core k-NN queries with Euclidean distance. 
In particular, we examine DSTree~\cite{conf/vldb/Wang2013}, iSAX2+~\cite{journal/kais/Camerra2014}, and VA+file~\cite{conf/cikm/Hakan2000}, the three data series methods that perform the best in terms of exact search~\cite{journal/pvldb/echihabi2018}, and also inherently support ng-approximate search.
 
%some are not applicable to our study since they either use correlation~\cite{conf/ssdm/Wang2000,conf/kdd/ColeSZ05}, are designed for subsequence matching~\cite{conf/kdd/Keogh1997} or in-memory data~\cite{conf/icde/shatkay1996}. Since the focus of our study is out-of-core k-NN queries supporting the Euclidean distance, the only methods that we are aware of that support this scenario are exact data series indexes that include ng-approximate search, so we chose the best in this class of techniques ~\cite{conf/vldb/Wang2013,journal/kais/Camerra2014,journal/vldb/Zoumpatianos2016} per the results in~\cite{journal/pvldb/echihabi2018}. 

%In this subsection, we describe exact methods designed for data series. 
%We selected the best methods based on an extensive experimental evaluation~\cite{journal/pvldb/echihabi2018}. 
%In this section, we describe, in chronological order, algorithms that can produce exact results.
%The properties of these algorithms are also summarized in Table~\ref{tab:multiprogram}.
%: ADS+~\cite{journal/vldb/Zoumpatianos2016}, DSTree~\cite{conf/vldb/Wang2013}, iSAX2+~\cite{journal/kais/Camerra2014}, MASS~\cite{journal/dmkd/Yeh2017}, M-tree~\cite{conf/vldb/ciaccia1997}, R*-tree~\cite{conf/icmd/Beckmann1990}, SFA trie~\cite{journal/edbt/Schafer2012}, UCR Suite~\cite{conf/kdd/Mueen2012}, and Stepwise~\cite{conf/kdd/Karras2011}. 
%Below, we succinctly describe, in chronological order of publication, the main intuition behind each similarity search method and any modifications or optimizations we added to the original implementations.
%
%Figure~\ref{fig:taxonomy} shows the taxonomy of the methods evaluated in this study.
%


%\paragraph*{\textbf{R*-tree}}
%1990

\begin{comment}
\noindent{\bf R*-tree.}
The R*-tree~\cite{conf/icmd/Beckmann1990} is a height-balanced spatial access method 
%for rectangles which optimizes the R-tree~\cite{conf/sigmod/Guttman1984} by reducing the coverage (dead space) and margin of individual rectangles and minimizes their overlap. The family of R-trees are spatial access methods 
that partitions the data space into a hierarchy of nested overlapping rectangles.
% , and organizes them into a height-balanced index. 
Each leaf can contain either the raw data objects or pointers to those, along with the enclosing rectangle. 
Each intermediate node contains the minimum bounding rectangle that encompasses the rectangles of its children. 
%The root node contains one rectangle bounding the whole data entries. 
Given a query $S_Q$, the R*-tree query answering algorithm visits all nodes whose rectangle intersects $S_Q$, starting from the root. Once a leaf is reached, all its data entries are returned. 
%Since the rectangles overlap, the exact search is not guaranteed to follow one single path to the leaf.
We tried multiple implementations of the R*-tree, and opted for the fastest~\cite{code/Marios2014}. 
We modified this code by adding support for PAA summaries.

%\paragraph*{\textbf{M-tree}}
%1997
\noindent{\bf M-tree.}
The M-tree~\cite{conf/vldb/Ciaccia1997} is a multidimensional, metric-space access method 
%but it partitions the data differently from the R-trees. Whereas the R-tree partitions the data entries according to their positions in the vector space using rectangles, the M-tree 
that uses hyper-spheres to divide the data entries according to their relative distances.
% using hyper-spheres. The M-tree is designed for metric spaces, which encompass vector spaces, and thus only requires the distance function to be a metric, i.e., it satisfies positivity, symmetry and triangular inequality. 
The leaves store data objects, and the internal nodes store routing objects; both store distances from each object to its parent. 
%A routing object $O_r$ is a data object which was selected to serve a routing role. It points to its subtree which includes all data objects within distance $r(O_r)$ from $O_r$. The subtree is called the covering tree and the distance $r(O_r)$ is called the covering radius. 
During query answering, the M-tree uses these distances to prune the search space. 
The triangle inequality that holds for metric distance functions guarantees correctness.
Apart from exact queries, it also supports $\epsilon$-approximate and $\delta$-$\epsilon$-approximate 
%($r$-range and $k$-NN) 
queries. 
We experimented with four different code bases:
% for the M-tree: 
two implementations that support bulk-loading~\cite{conf/ads/Ciaccia1998,journal/vldb/Dallachiesa2014}, the disk-aware mvptree~\cite{conf/sigmod/Bozkaya1997},
% for the mvptree~\cite{conf/sigmod/Bozkaya1997}. 
%Although the implementation proposed in
and a memory-resident implementation~\cite{journal/vldb/Dallachiesa2014}.
% is memory-persistent, we picked it since it is 
We report the results with the memory-resident version, because (despite our laborious efforts) it was the only one that scaled to datasets larger than 1GB. 
We modified it slightly to use the same sampling technique as the original implementation of the MTree~\cite{conf/ads/Ciaccia1998} that chooses the number of initial samples based on the leaf size, minimum utilization, and dataset size. 
%We set the minimum utilization to 0.2 as suggested by the author.
\end{comment}

\noindent{\bf DSTree}~\cite{conf/vldb/Wang2013} is a tree index based on the EAPCA summarization technique and supports ng-approximate and exact query answering. 
Its dynamic segmentation algorithm %helps it summarize a data series accurately and allows it to increase the resolution of a segmentation by splitting 
allows tree nodes to split vertically and horizontally, unlike the other data series indexes which allow either one or the other. 
%In particular, SAX-based indexes allow horizontal splitting by adding a breakpoint to the y-axis, and SFA allows vertical splitting by adding a new DFT coefficient.) 
The DSTree supports a lower and upper bounding distance and uses them to calculate a QoS measure that determines the optimal way to split any given node. 
%Each nodes has its own segmentation policy which may result in a different number of segments or in segments of different length. The data series in a given node are all segmented using the same policy but each node has its own segmentation policy which may result in nodes having a different number of segments or segments of different lengths.
We significantly improved the efficiency of the original DSTree Java implementation by developing it from scratch in C and optimizing its buffering and memory management, 
%significantly improving the performance of the original implementation (in Java). 
%In fact, using our default workload of indexing and answering 100 exact queries, the C version is 
making it 4 times faster across datasets ranging between 25-250GB.


%2014
%\noindent{\bf iSAX2+.}
\noindent{\bf SAX-based indexes} include different flavors of tree indexes based on SAX summarization. 
The original iSAX index~\cite{conf/kdd/shieh1998} was enhanced with a better spliting policy and bulk-loading support in iSAX 2.0~\cite{conf/icdm/Camerra2010}, while
%and to improve iSAX's node splitting by using equi-depth of gaussian distribution. 
iSAX2+~\cite{journal/kais/Camerra2014} further optimized bulk-loading. 
%In the literature, competing approaches have either compared to iSAX, or iSAX 2.0. 
%This is the first time that iSAX2+ is compared to other exact data series indexes. 
ADS+~\cite{journal/vldb/Zoumpatianos2016} then improved upon iSAX2+ by making it adaptive, %initially building a tree structure containing only the iSAX summarizations of the raw data, and then using a query-adaptive algorithm to add the raw data to the leaves. 
Coconut~\cite{journal/pvldb/kondylakis18,DBLP:conf/sigmod/KondylakisDZP19,coconutjournal} by constructing a compact and contiguous data layout, and DPiSAX~\cite{dpisax,dpisaxjournal}, ParIS~\cite{conf/bigdata/peng18} and MESSI~\cite{conf/icde/peng20} by exploiting parallelization.
Here, we use iSAX2+, because of its excellent performance~\cite{journal/pvldb/echihabi2018} and the fact that the SIMS query answering strategy~\cite{journal/vldb/Zoumpatianos2016} of ADS+, Coconut, and ParIS is not immediately amenable to approximate search with guarantees (we plan to extend these methods in our future work). 
%In contrast, iSAX2+ supports efficient ng-approximate %and exact 
%query answering. 
We do not include DPiSAX and MESSI, because they are distributed, and in-memory only, algorithms, respectively. %, and therefore out of the scope of this work.
%We significantly improved the efficiency of iSAX2+ by reimplementing it from scratch in C (instead of C\#), and optimizing its memory management. 

%It tailors to diverse workloads by proposing different variants: ADS, ADS+, PADS+, ADS FULL and ADS+ (SIMS). ADS+ improves the performance of ADS by dynamically using an adaptive leaf size: a large leaf size is chosen during index building and a smaller one during query answering. PADS+ tailors to approximate answering by only building the root node and the buffers of the root children nodes. 
%The SIMS algorithm performs an ng-approximate search in the index tree and uses the answer to prune the search space, performing a skip-sequential search on the raw data.
%In all our experiments involving ADS+ we use the SIMS algorithsm for exact similarity search.
%ADS-FULL is a non-adaptive version of ADS, that builds a full index using a double pass on the data.
%but instead of adding the raw data series to the leaves, it adds their summarizations. 


%summarizes the raw data series into SAX summarizations . A SAX summarization first divides the raw data into a smaller number of horizontal segments of equal length using PAA \cite{journal/kais/Keogh2001}. Then instead of representing each segment with a real value as in the DSTree, it uses a discretization technique on the Stepwise 

%The difference between isax2+ and sfa in terms of the tree structure:
%isax2+ fanout at the root level is equal to the length of the reduced dim (number of PAA segments)
%isax2+ fanout at each internal node is always 2
%isax2+ depth is at most the alphabet size * num_PAA_segments: default 8 * 16. 

%\paragraph*{\textbf{ADS+}}
%2016
%\noindent{\bf ADS+.}
%ADS+~\cite{journal/vldb/Zoumpatianos2016} is the first adaptive data series index. 
%based on the iSAX representation and further improves upon iSAX2+ by initially building the 
%It builds a tree structure containing only the iSAX summarizations of the raw data, and then exploits a query-adaptive algorithm to add the raw data to the leaves. It tailors to diverse workloads by proposing different variants: ADS, ADS+, PADS+, ADS FULL and ADS+ (SIMS). ADS+ improves the performance of ADS by dynamically using an adaptive leaf size: a large leaf size is chosen during index building and a smaller one during query answering. PADS+ tailors to approximate answering by only building the root node and the buffers of the root children nodes. 
%The SIMS algorithm performs an ng-approximate search in the index tree and uses the answer to prune the search space, performing a skip-sequential search on the raw data.
%In all our experiments involving ADS+ we use the SIMS algorithm for exact similarity search.
%ADS-FULL is a non-adaptive version of ADS, that builds a full index using a double pass on the data.
%but instead of adding the raw data series to the leaves, it adds their summarizations. 

%compared to isax 2, serial, rtree, xtree


%\paragraph*{\textbf{DSTree}}
%2013

% (Figure~\ref{fig:dstree:orig:new}).
\ifJournal
\begin{figure}[t]
	\captionsetup{justification=centering}
	\captionsetup[subfigure]{justification=centering}
	%\hspace{3mm}
	\begin{subfigure}{0.49\columnwidth}
		\centering
		\includegraphics[width=\columnwidth]{dstree_orig_new}
		\caption{Total Time (Indexing and Answering 100 Exact Queries)}
		\label{fig:dstree:orig:new:combined}
	\end{subfigure}u
	\begin{subfigure}{0.49\columnwidth}
		\centering
		\includegraphics[width=\columnwidth]{dstree_orig_new_detailed}
		\caption{Detailed Times for Indexing and Answering 100 Exact Queries}
		\label{fig:dstree:orig:new:detailed}
	\end{subfigure}
	\caption{DSTree Implementation Optimization}
	\label{fig:dstree:orig:new}
}
\end{figure}
\fi




\begin{comment}

%\paragraph*{\textbf{MASS}} 
%2017
\noindent{\bf MASS.}
MASS~\cite{journal/dmkd/Yeh2017} is an exact subsequence matching algorithm, which computes 
%the distance profile of a query $Q$ to a long data series $DS$, i.e. the distance between $Q$ and every subsequence of $DS$. 
the distance between a query, $S_Q$, and every subsequence in the series, using 
%It calculates the distance based on 
the dot product of the DFT transforms of the series and the reverse of $S_Q$.
We adapted it to perform exact whole matching queries. 
%Since the query and candidate series are all of equal length, only one dot product operation is performed between a single query and a single candidate data series. However, the number of dot products calculated for each query is equal to the size of the dataset. 

\end{comment}


%\paragraph*{\textbf{VA+file}}
%2000


{\color {black} \noindent{\bf TARDIS}~\cite{conf/icde/zhang2019} is a distributed indexing method that supports exact and $ng$-approximate kNN queries. It improves the efficiency and accuracy of iSAX by building a more compact, k-ary tree index, exploiting word-level (instead of character-level) cardinality, and using a novel conversion scheme between SAX representations. We do not include TARDIS in the experimental evaluation since it is a distributed algorithm (built in Scala for Spark). 
%All the evaluated approaches are single-node approaches built with C/C++.
}

\noindent{\bf VA+file}~\cite{conf/cikm/Hakan2000} is a skip-sequential method that improves the accuracy and efficiency of the VA-file~\cite{conf/vldb/Weber1998}. 
Both techniques create a file that contains quantization-based summarizations of the original multidimensional data. 
Search proceeds by sequentially reading each summarization, % from the filter file, 
calculating its lower bounding distance to the query, and accessing the original multidimensional vector only if the lower bounding distance is less than the current \emph{best-so-far (bsf)} answer. 
%The VA+file improves the approximations by: 1) decorrelating the data using KLT; 2) allocating a different number of bits per dimension depending on its energy level; 3) and using a k-means instead of an equi-depth approach to select the centroids for each dimension. 
We greatly improved the performance of the original VA+file by approximating KLT with DFT~\cite{conf/cikm/Hakan2000,journal/acta/maccone2007} and implementing it in C instead of Matlab. 
In the rest of the text, whenever we mention the VA+file, we refer to the modified version.



 %helps it summarize a data series 
%significantly improved the efficiency of the original VA+file by implementing it in C and modifying it to use DFT instead of KLT, . %In the rest of the text, whenever we mention the VA+file, we refer to the modified version.



%does not assume that neighboring points (dimensions) in the sequence are uncorrelated. It thus improves the accuracy of the approximations by 
%1) transforming the original data using KLT for optimal energy compaction; 2) 
%allocating bits per dimension in a non-uniform fashion, and partitioning each dimension using a k-means (instead of an equi-depth approach). 
%In the rest of the text, whenever we mention the VA+file, we refer to the modified version.

\begin{comment}
%\paragraph*{\textbf{Stepwise}}
%2011
\noindent{\bf Stepwise.}
The Stepwise method~\cite{conf/kdd/Karras2011} differentiates itself from indexing methods by storing DHWT summarizations vertically across multiple levels. 
%That is different levels of filtering are performed before the raw data in the high dimensional space is accessed. 
%A preprocessing step consists of transforming the raw data into a multi-level representation using DHWT. 
This process happens in a pre-processing step.
%The raw data is stored physically on disk in this new representation level by level. The number of levels is determined by the length of the query. So the longer the query, the more levels it takes to represent it, which also means the more intermediate steps there are before needing to access the raw data. 
When a query $S_Q$ arrives, the algorithm converts it to DWHT, and computes the distance between $S_Q$ and the DHWT of each candidate data series 
%in a stepwise fashion 
one level at a time,
%. The precomputed distances are stored for faster processsing. Both a 
using lower and upper bounding distances it filters out non-promising candidates.
When leaves are reached, the final refinement step consists of calculating the Euclidean distance between the raw representations of $S_Q$ and the candidate series.
We modified the original implementation to load the pre-computed sums in memory and answer one query at a time (instead of the batch query answering of the original implementation). 
We also slightly improved memory management to address swapping issues that occurred with the out-of-memory datasets. 

%compared to isax (not isax 2), and sequential access.

%they show Stepwise being better than sequential scan and sequential scan better than isax across the board.


%\paragraph*{\textbf{SFA trie}}
%2012
\noindent{\bf SFA trie.}
%Given an input dataset $D$ of raw data series $DS$,  
The SFA approach~\cite{journal/edbt/Schafer2012} first summarizes the series using SFA of length 1 and builds a trie with a fanout equal to the alphabet size on top of them. 
%The threshold of a leaf is the maximum number of raw data series that it can hold. Once this threshold is reached, the leaf is split, 
As leaves reach their capacity and split, the length of the SFA word for each series in the leaf is increased by one, and the series are redistributed among the new nodes. 
The maximum resolution is the number of DFT coefficients given as a parameter. 
%A bulk-loading algorithm is supported to allow faster insertions and a lower-bounding distance function is supplied to prune the search space. 
SFA implements lower-bounding to prune the search space, as well as a bulk-loading algorithm.
We re-implemented SFA in C, optimized its memory management, and improved the sampling and buffering schemes. 
This resulted in a significantly faster implementation than the original one in Java.
%The original JAVA version allowed the max leaf size to be exceeded, so we modified this as well to stop execution and request the user to increase the leaf threshold.

%\paragraph*{\textbf{UCR Suite}}
%2012
\noindent{\bf UCR Suite.}
The UCR Suite~\cite{conf/kdd/Mueen2012} is an optimized sequential scan algorithm 
%supporting both DTW and Euclidean distances 
for exact subsequence matching. 
%The algorithm uses the following optimizations: squared distances, lower bounding of DTW, early abandoning of ED and DTW, early abandoning of z-normalization, reordering early abandoning and cascading lower bounds in the case of DTW.
We adapted the original algorithm to support exact whole matching. 
%We used the optimizations relevant to the Euclidean distance.
\end{comment}























\vspace*{-0.2cm}

\subsubsection{Extensions of Data Series Methods}
\label{sec:dataseriesextensions}

We now propose extensions to the data series methods described above, that will allow them to support $\epsilon$-approximate and $\delta$-$\epsilon$-approximate search (in addition to ng-approximate that they already support).
Due to space limitations, we only discuss the tree-based methods (such as iSAX2+ and DSTree); skip-sequential techniques (such as VA+file) can be modified following the same ideas.

The exact 1-NN search algorithms of DSTree and iSAX2+ are based on an optimal exact NN algorithm first proposed for PMR-Quadtree~\cite{conf/isasd/samet1995}, which was then generalized for any hierarchical index structure that is constructed using a conservative and recursive partitioning of the data~\cite{conf/pods/berchtold1997}. 
%This generalized algorithm forms the backbone of exact 1-NN search for most of the indexing methods surveyed in this study~\cite{conf/icmd/Beckmann1990,conf/vldb/Ciaccia1997,journal/edbt/Schafer2012,conf/vldb/Wang2013,journal/kais/Camerra2014,journal/vldb/Zoumpatianos2016,ulisse}. 

\begin{algorithm}[tb]
	{\scriptsize
		\caption{exactNN({$\bm{S_Q}$},{$\bm{idx}$})}
		\begin{algorithmic}[1]
			%\Comment{query $q$ and index $idx$}	
			\\{$\bm{bsf.dist}$} $\gets$ $\infty$ ; {$\bm{bsf.node}$} $\gets$ $NULL$;		
			\For {each {{$\bm{rootNode}$} in {$\bm{idx}$} }}         
			\State{$\bm{result.node}$} $\gets$ {$\bm{rootNode}$};
			\State{$\bm{result.dist}$} $\gets$ calcMinDist({$\bm{S_Q}$},{$\bm{rootNode}$});			
			\State{push $\bm{result}$ to $\bm{pqueue}$} 
			\EndFor
			\\{$\bm{bsf}$} $\gets$ {\color{mygreen}\dashuline{ng-approxNN}}({{$\bm{S_Q}$},$\bm{idx}$}); 		
			\\add {$\bm{bsf}$} to {$\bm{pqueue}$};
			\While{ {$\bm{result}$} $\gets$ pop next node from {$\bm{pqueue}$ } } 
			\State{$\bm{n}$} $\gets$ {$\bm{result.node}$};
			\If { {$\bm{n.dist}$} $>$ {$\bm{bsf.dist}$}} 
			break;
			\EndIf                 
			\If {{$\bm{n}$} is a leaf}    \Comment{a leaf node}     
			\For {each {{$\bm{S_C}$} in {$\bm{n}$} }}         
			\State {$\bm{realDist}$} $\gets$ calcRealDist({$\bm{S_Q}$},{$\bm{S_C}$});
			\If { {$\bm{realDist}$} $<$ {$\bm{bsf.dist}$}} 
			\State {$\bm{bsf.dist}$} $\gets$ {$\bm{realDist}$} ;
			\State {$\bm{bsf.node}$} $\gets$ {$\bm{n}$};		               
			\EndIf                  
			\EndFor        
			\Else  \Comment{an internal node}
			\For {each {{$\bm{childNode}$} in {$\bm{n}$} }}         
			\State {$\bm{minDist}$} $\gets$ calcMinDist({$\bm{S_Q}$},{$\bm{childNode}$});
			\If { {$\bm{minDist}$} $<$ {$\bm{bsf.dist}$}} add {$\bm{childNode}$} to
			\State {$\bm{pqueue}$ } with priority {$\bm{minDist}$}; 
			\EndIf                  
			\EndFor        
			\EndIf 
			\EndWhile\label{euclidendwhile}
			\State \Return {$\bm{bsf}$}%\Comment{The gcd is b}
		\end{algorithmic}
		\label{alg:exactNN}
	} % font size
\end{algorithm}

Algorithm~\ref{alg:exactNN} describes an index-invariant algorithm for exact 1-NN search. It takes as arguments a query $S_Q$ and an index $idx$.
%Line 4 only applies to~\cite{journal/edbt/Schafer2012,conf/vldb/Wang2013,journal/kais/Camerra2014,journal/vldb/Zoumpatianos2016,ulisse}.
Lines 1-5 initialize the \emph{best-so-far (bsf)} answer and a priority queue with the root node(s) of the index in increasing order of lower bounding ($lb$) distances (the $lb$ distance is calculated by the function $calcMinDist$). 
In line 6, the $ng$-approxNN function traverses one path of the index tree visiting one leaf to return an $ng$-approximate bsf answer, {\color{black} which} is added to the queue (line 7). 
In line 8, the algorithm pops nodes from the queue, terminating in line 10 if the $lb$ distance of the current node is greater than the current \emph{bsf} distance (the $lb$ distances of all remaining nodes in the queue are also greater than the \emph{bsf}). 
Otherwise, if the node is a leaf, the \emph{bsf} is updated if a better answer is found (lines 11-16); if the node is an internal node, its children are added to the queue provided their $lb$ distances are greater than the \emph{bsf} distance (lines 18-21).

\begin{algorithm}[tb]
	{\scriptsize
		\caption{{\color{myred}\underline{\underline{delta}}}{\color{myblue}\underline{Epsilon}}NN({$\bm{S_Q}$},{$\bm{idx}$},{$\bm{\delta}$},{$\bm{\epsilon}$}, {$\bm{F_Q(.)}$})}
		\begin{algorithmic}[1]
			%\Comment{query $q$ and index $idx$}	
			\\{$\bm{bsf.dist}$} $\gets$ $\infty$ ; {$\bm{bsf.node}$} $\gets$ $NULL$;			
			\Statex	 {\color{myred}\underline{\underline{${\bm {r_\delta(Q)}}$  $\gets$ calcDeltaRadius({$\bm{S_Q}$},{$\bm{\delta}$}, {$\bm{F_Q(.)}$})}}}; 
			\\{$\bm{bsf}$} $\gets$ {\color{mygreen}\dashuline{ng-approxNN}}({{$\bm{S_Q}$},$\bm{idx}$}); 		
			\\add {$\bm{bsf}$} to {$\bm{pqueue}$};
			\For {each {{$\bm{rootNode}$} in {$\bm{idx}$} }}         
			\State{$\bm{result.node}$} $\gets$ {$\bm{rootNode}$};
			\State{$\bm{result.dist}$} $\gets$ calcMinDist({$\bm{S_Q}$},{$\bm{rootNode}$});			
			\State{push $\bm{result}$ to $\bm{pqueue}$} 
			\EndFor
			\While{ {$\bm{result}$} $\gets$ pop next node from {$\bm{pqueue}$ } } 
			\State{$\bm{n}$} $\gets$ {$\bm{result.node}$};
			\If { {$\bm{n.dist}$} $>$ {$\bm{bsf.dist}{\color{myblue}\underline{/(1+\epsilon)}}$}} 
			break;
			\EndIf                 
			\If {{$\bm{n}$} is a leaf}    \Comment{a leaf node}     
			\For {each {{$\bm{S_C}$} in {$\bm{n}$} }}         
			\State {$\bm{realDist}$} $\gets$ calcRealDist({$\bm{S_Q}$},{$\bm{S_C}$});
			\If { {$\bm{realDist}$} $<$ {$\bm{bsf.dist}$}} 
			\State {$\bm{bsf.dist}$} $\gets$ {$\bm{realDist}$} ;
			\State {$\bm{bsf.node}$} $\gets$ {$\bm{n}$};		               
			\Statex\hspace{2cm}{{\color{myred} \underline{\underline{if { {$\bm{bsf.dist}$} $\leq$ $(1+\epsilon)$ {$\bm {r_\delta(Q)}$}} then exit;}}}}
			\EndIf                  
			\EndFor        
			\Else  \Comment{an internal node}
			\For {each {{$\bm{childNode}$} in {$\bm{n}$} }}         
			\State {$\bm{minDist}$} $\gets$ calcMinDist({$\bm{S_Q}$},{$\bm{childNode}$});
			\If { {$\bm{minDist}$} $<$ {$\bm{bsf.dist}{\color{myblue}/\underline{(1+\epsilon)}}$}} add
			\State {$\bm{childNode}$} to {$\bm{pqueue}$ } with priority {$\bm{minDist}$}; 
			\EndIf                  
			\EndFor        
			\EndIf 
			\EndWhile\label{euclidendwhile}
			\State \Return {$\bm{bsf}$}%\Comment{The gcd is b}
		\end{algorithmic}
		\label{alg:deltaepsilonNN}
	} % font size
\end{algorithm}


\begin{comment}
\begin{algorithm}
\caption{{\color{myred}\underline{\underline{delta}}}{\color{myblue}\underline{Epsilon}}NN({$\bm{S_Q}$},{$\bm{idx}$},{$\bm{\delta}$},{$\bm{\epsilon}$}, {$\bm{F_Q(.)}$})}
\begin{algorithmic}[1]
%\Comment{query $q$ and index $idx$}
\\{$\bm{pqueue}$} $\gets$ initialize a priority queue with the root node(s) of {$\bm{idx}$}; {\color{myred}\underline{\underline{${\bm {d^Q_\delta}}$  $\gets$ calcDQDelta({$\bm{S_Q}$},{$\bm{\delta}$}, {$\bm{F_Q(.)}$})}}}; 
\\{$\bm{bsf.dist}$} $\gets$ $\infty$ ; 
\\{$\bm{bsf.node}$} $\gets$ $NULL$;		
\\{$\bm{bsf}$} $\gets$ {\color{mygreen}\dashuline{ng-approxNN}}({{$\bm{S_Q}$},$\bm{idx}$}); 		
\\add {$\bm{bsf}$} to {$\bm{pqueue}$};
\While{ {$\bm{node}$} $\gets$ pop next node from {$\bm{pqueue}$ } } 
\If { calcMinDist({$\bm{S_Q}$},{$\bm{node}$}) $>$ {$\bm{bsf.dist} {\color{myblue}\underline{/(1+\epsilon)}}$ }}  
break;
\EndIf                 
\If {{$\bm{node}$} is a leaf}    \Comment{a leaf node}     
\For {each {{$\bm{S_C}$} in {$\bm{node}$} }}         
\State {$\bm{realDist}$} $\gets$ calcRealDist({$\bm{S_Q}$},{$\bm{S_C}$});
\If { {$\bm{realdist}$} $<$ {$\bm{bsf.dist}$}} 
\State {$\bm{bsf.dist}$} $\gets$ {$\bm{realDist}$} ;
\State {$\bm{bsf.node}$} $\gets$ {$\bm{node}$};
\Statex\hspace{2cm}{{\color{myred} \underline{\underline{if { {$\bm{bsf.dist}$} $\leq$ $(1+\epsilon)$ {$\bm {d^Q_\delta}$}} then break;}}}}
\EndIf                  
\EndFor        
\Else  \Comment{an internal node}
\For {each {{$\bm{childNode}$} in {$\bm{node}$} }}         
\State {$\bm{minDist}$} $\gets$ calcMinDist({$\bm{S_Q}$},{$\bm{childNode}$});
\If { {$\bm{minDist}$} $<$ {$\bm{bsf.dist}{\color{myblue}/\underline{(1+\epsilon)}}$} } 
\State add {$\bm{childNode}$} to {$\bm{pqueue}$ } with priority \State {$\bm{minDist}$}; 
\EndIf                  
\EndFor        
\EndIf 
\EndWhile\label{euclidendwhile}
\State \Return {$\bm{bsf}$}%\Comment{The gcd is b}
\end{algorithmic}
\label{alg:deltaepsilonNN}
\end{algorithm}
\end{comment}

We can use Algorithm~\ref{alg:exactNN} for $ng$-approximate search, by visiting one leaf and returning the first \emph{bsf}. 
%This $ng$-approximate answer is depicted in Figure~\ref{fig:approxNN} with $S_{ng}$: it can be anywhere between the inner and outer spheres. 
This $ng$-approximate answer can be anywhere in the data space
% between the inner and outer spheres. 
%The inner sphere has radius $d_x$, which is equal to the distance between the query $S_Q$ and its exact NN $S_x$. 
%The outer sphere represents the full data space. 

We extend approximate search in Algorithm~\ref{alg:exactNN} by introducing two changes: (i) allow the index to visit up to $nprobe$ leaves (user parameter); and (ii) apply the modifications suggested in~\cite{conf/icde/Ciaccia2000} to support $\delta$-$\epsilon$-approximate NN search. 
The first change is straightforward, so we only describe the second change in Algorithm~\ref{alg:deltaepsilonNN}. 
%
%To return the $\epsilon$-approximate NN of $S_Q$, $S_\epsilon$ (falls between the inner and dashed spheres in Figure~\ref{fig:approxNN}), {\bf\emph{bsf.dist}} is replaced with {\bf\emph{bsf.dist$/(1+\epsilon)$}} in lines 10 and 20.  
%
To return the $\epsilon$-approximate NN of $S_Q$, $S_\epsilon$, {\bf\emph{bsf.dist}} is replaced with {\bf\emph{bsf.dist$/(1+\epsilon)$}} in lines 10 and 20.  
%To return the $\delta$-$\epsilon$-approximate NN of $S_Q$, $S_{\delta\epsilon}$ (falls between the inner and dotted spheres in Figure~\ref{fig:approxNN}), we also modify lines 1 and 16.
%To return the $\delta$-$\epsilon$-approximate NN of $S_Q$, $S_{\delta\epsilon}$ (falls with probability $\delta$ between the inner and dashed spheres in Figure~\ref{fig:approxNN}), we also modify lines 1 and 16.
To return the $\delta$-$\epsilon$-approximate NN of $S_Q$, $S_{\delta\epsilon}$, we also modify lines 1 and 16.

The distance $r_\delta(Q)$ is initialized in line 1 using $F_Q(\cdot)$, $S_Q$ and $\delta$. $F_Q(\cdot)$ represents the relative distance distribution of $S_Q$. 
Intuitively, $r_\delta(Q)$ is the maximum distance from $S_Q$, such that the sphere with center $S_Q$ and radius $r_\delta(Q)$ is empty with probability $\delta$. 
As proposed in~\cite{conf/pods/Ciaccia1998}, we use $F(\cdot)$, the overall distance distribution, instead of $F_Q(\cdot)$ to estimate $r_\delta(Q)$. 
The delta radius $r_\delta(Q)$ is then used in line 16 as a stopping condition.
%Although ${\epsilon}$-approximate nearest neighbor algorithms improve efficiency at the expense of accuracy. The actual loss of accuracy is much smaller than the user-defined tolerance $\epsilon$~\cite{journal/acm/Arya1998}. To answer herefore, a probabilistic algorithm called PAC-NN~\cite{conf/icde/Ciaccia2000} has been proposed to return $\epsilon$-approximate answers whose accuracy is close to the tolerated threshold, by introducing a stopping condition based on distance distributions. 
When  $\delta = 1$, Algorithm~\ref{alg:deltaepsilonNN} returns $S_{\delta\epsilon}$, the $\epsilon$-approximate NN of $S_Q$, and when $\delta = 1$ and $\epsilon=0$, Algorithm~\ref{alg:deltaepsilonNN} becomes equivalent to Algorithm~\ref{alg:exactNN}, i.e., it returns $S_x$, the exact NN of $S_Q$. 
Our implementations generalize Algorithm~\ref{alg:deltaepsilonNN} to the case of $k \ge 1$. 
These modifications are straightforward and omitted for the sake of brevity. 
A proof of correctness for Algorithm~\ref{alg:deltaepsilonNN} can be found in~\cite{conf/icde/Ciaccia2000,conf/sisap/ciaccia17} for $k = 1$ and $k \ge 1$, respectively.
 


\begin{comment}

\begin{lem} \label{lem:appmatch}
	Given an index $\bm{idx}$, constructed using a conservative and recursive partitioning technique, a query $\bm{S_Q}$, and $\bm{\epsilon \geq 0}$, Algorithm~\ref{alg:exactNN} returns $\bm{S_\epsilon}$, the $\epsilon$-approximate nearest neighbor of $S_Q$, if lines 10 and 20 are modified by replacing {$\bm{bsf.dist}$} with {$\bm{bsf.dist/(1+\epsilon)}$}.
\end{lem}


Below, we propose a formal proof of correctness (Lemma \ref{lem:appmatch}) for Algorithm\label{alg:deltaepsilonNN}. We provide a proof for $epsilon$-approximate search, i.e., $delta = 1$ and $r_\delta(Q) = \infty$ since a proof for $delta$-$epsilon$-approximate search appears in\cite{conf/icde/Ciaccia2000}

\begin{proof}
We will prove Lemma~\ref{lem:appmatch} by contradiction. First, consider the query $S_Q$, the index $idx$, the series $S_C$, which is the exact nearest neighbor of $S_Q$, and the series $S_C'$ returned by the modified Algorithm~\ref{alg:exactNN}. Distances $d$ and $d'$ denote $d(S_Q,S_C)$ and $d(S_Q,S_C')$ respectively. Suppose that $S_C'$ is not the ${\epsilon}$-approximate nearest neighbor of $S_Q$, therefore $d' > (1+\epsilon)d$, according to Definition~\ref{def:epsmatch}. Since Algorithm~\ref{alg:exactNN} returns $S_C'$ and not $S_C$, there exist two scenarios: 1) $S_C'$ was processed before $S_C$ or 2) $S_C$ was processed before $S_C'$. In scenario 1, $S_C$ was filtered out by the algorithm when bsf.dist was equal to $d'$. This could only have happened in either line 7 or 17 so $minDist(S_Q, N_C) >= d' / (1+\epsilon)$, $N_C$ being the node containing $S_C$. Since $d >= minDist(S_Q, N_C)$ and both $d$ and $\epsilon$ are positive, then $d >= d'/(1+\epsilon)$, so $d' \leq (1+\epsilon) d$ and $d' > (1+\epsilon)d$, so $d > (1+\epsilon)d$. Since $d$ is positive, then $1 > 1+\epsilon$ but this contradicts with $\epsilon \geq 0$. In scenario 2, there exist another series $S_C''$ which was the $bsf$ when $S_C$ was filtered out ($bsf.dist = d(S_Q, S_C'') = d'')$. Again, this would have happened in lines 7 or 17, thus $minDist(S_Q, N_C) >= d''$

\end{proof}

\begin{proof}
	First, consider the query $S_Q$, the index $idx$, the series $S_C$, which is the exact nearest neighbor of $S_Q$, and the series $S_\epsilon$ returned by Algorithm~\ref{alg:deltaepsilonNN}. $S_\epsilon$ can be located anywhere in the blue region of Figure~\ref{fig:approxNN} (including the boundaries). Distances $d$ and $d'$ denote $d(S_Q,S_C)$ and $d(S_Q,S_C_\epsilon)$ respectively. Since Algorithm~\ref{alg:deltaepsilonNN} returns $S_\epsilon$ and not $S_C$, there exist two scenarios: 1) $S_\epsilon$ was processed before $S_C$; or 2) $S_C$ was processed before $S_\epsilon$. In scenario 1, $S_C$ was filtered out by the algorithm when $bsf.dist$ was equal to $d'$. This could only have happened in either line 10 or 20 so $minDist(S_Q, N_C) >= d' / (1+\epsilon)$, $N_C$ being the node containing $S_C$. Since $d >= minDist(S_Q, N_C)$ and both $d$ and $\epsilon$ are positive, then $d >= d'/(1+\epsilon)$, so $d' \leq (1+\epsilon) d$. 
	Therefore, according to Definition~\ref{def:epsmatch}, $S_\epsilon$ is  the ${\epsilon}$-approximate nearest neighbor of $S_Q$. In scenario 2, the
	re exist another series $S''_\epsilon$ which was the $bsf$ when $S_C$ was filtered out ($bsf.dist = d(S_Q, S''_\epsilon) = d'')$. Again, this would have happened in lines 10 or 20, thus $minDist(S_Q, N_C) >= d'' / (1+\epsilon)$, so $d >= d'' / (1+\epsilon)$ so $d'' \leq (1+\epsilon)d$. Since, the algorithm returns $S'_\epsilon$ and not $S''_\epsilon$, then $d' < d''$, so $d' < (1+\epsilon)d$. Again, according to Definition~\ref{def:epsmatch}, it follows that $S'_\epsilon$ is  the ${\epsilon}$-approximate nearest neighbor of $S_Q$.
	
\end{proof}

%{\color{red}
%Based on~\cite{conf/isasd/samet1995}, we extend Algorithm~\ref{alg:exactNN} to make it incremental (Algorithm~\ref{alg:exactIncrNN}). Add pseudocode and proof for approx k-NN.

% Do we want to make it incremental since it relies on mindist? 
% Maybe we should add experiments comparing scheduling policies: mindist vs maxdist since maxdist has been shown to be better for approx-kNN? Can we use maxdist for the exact indexes?
% Discuss the results in these papers: Maxdist node scheduling is more optimal than mindist for approx-kNN  ~\cite{journal/jda/bustos04,conf/sisap/ciaccia17}
% while mindist is more optimal for exact-kNN ~\cite{conf/pods/berchtold1997,conf/isasd/samet1995}

\end{comment}













\begin{figure}[tb]
	\centering
	\captionsetup{justification=centering}
	%\includegraphics[scale =0.46]{deltaEpsilonNN_Proof_v7.pdf}
	%\caption{$\delta$-$\epsilon$-Approximate NN search.}
	%\label{fig:approxNN}
	%\vspace*{0.4cm}
	\includegraphics[width=\columnwidth]{taxonomy_journal_v10.pdf}
	\caption{{\color{black} Taxonomy of similarity search methods.}}
	\vspace*{-0.2cm}
	\label{fig:taxonomy}
\end{figure}






\subsection{Taxonomy of Similarity Search Methods}
Figure~\ref{fig:taxonomy} presents a taxonomy of similarity search methods based on the type of guarantees they provide (methods with multiple types of guarantees are included in more than one leaf of the taxonomy).
We call probabilistic the general $\delta$-$\epsilon$-approximate methods. 
%include both probabilistic approximate methods with $\delta$-approximate guarantees,
% ADS+, DSTree, iSAX2+, MTree, SRS, VA+file, 
When $\delta =1$ we have the $\epsilon$-approximate methods.
%: ADS+, DSTree, iSAX2+, MTree, VA+file.
Setting $\delta=1$ and $\epsilon=0$, we get the exact methods. %ADS+, DSTree, iSAX2+, MTree, MASS, RTree, Stepwise, UCR-Suite, VA+file. 
Finally, methods that provide no guarantees are categorized under ng-approximate. 
%Those include: ADS+, DSTree, HNSW, IMI, iSAX2+, VA+file.
Here, {\color{black} we cover 7 state-of-the-art methods from the high-dimensional literature, Flann, HD-index, HNSW, IMI, NSG, QALSH and SRS, as well as the 3 best methods from the data series community~\cite{journal/pvldb/echihabi2018}, iSAX2+, DSTree and VA+file. %\sout{we only concentrate on the highest performing disk-based algorithms, skipping methods like: MTree~\cite{conf/vldb/Ciaccia1997}, MASS~\cite{code/Mueen2017}, and Stepwise~\cite{conf/kdd/Karras2011}. Moreover, we also skip UCR-Suite~\cite{conf/kdd/Mueen2012} as it is designed for a different problem, and ADS+ as it belongs to the iSAX family, covered by iSAX2+.}
}











\section{Experimental Evaluation}
\label{sec:experiments}
{\color{black}
We assessed all methods on the same framework. 
Source code, datasets, queries, and all results are available in~\cite{url/DSSeval2}.










\vspace{1cm}



\subsection{Experimental Setup}
\label{subsec:framework}
\label{subsec:environment}

\noindent{\bf Environment.} 
All methods were compiled with GCC 6.2.0 under Ubuntu Linux 16.04.2 with their default compilation flags; optimization level was set to 2. 
Experiments were run on a server with two Intel Xeon E5-2650 v4 2.2GHz CPUs,
%(30MB cache, 12 cores, 24 hyper-threads),
%384GB of RAM (12 x 32GB RDIMM, 2400 MT/s)
75GB\footnote{We used GRUB to limit the amount of RAM, so that all methods are forced to use the disk. Note that GRUB prevents the operating system from using the rest of the RAM as a file cache, which is what we wanted for our experiments.} of RAM, 
%(RDIMM 2400 MT/s), 
and 10.8TB (6 x 1.8TB) 10K RPM SAS hard drives 
%(12Gbps) 
in RAID0 with a throughput of 1290 MB/sec.
%The second machine, called \emph{SSD}, is a server with two Intel Xeon E5-2650 v4 2.2Ghz CPUs, 
%(30MB cache, 12 cores, 24 hyper-threads) {\bf ??? do we need the parenthesis? ???}, 
%256 (8 x 32GB) RDIMM 2400 MT/s RAM, 
%75GB of RAM, 
%(RDIMM 2400 MT/s), 
%and 3.2TB (2 x 1.6TB) SATA2 SSD in RAID0.
%The throughput of the RAID0 array is 330 MB/sec.
%All our algorithms are single-core implementations. 
%, while we deliberately limited the available RAM for all experiments to 75GB in order to force methods use the hard disk.
%We used GRUB settings to limit RAM, in order prevent the operating system from using the rest of it as a file cache, thus influencing our experiments.
% We will explain this phenomenon in detail later in this section.

%\noindent{\textbf{Scope.}}
%This work concentrates on approximate whole-matching (WM) k-NN queries, including $ng$-approximate, $\epsilon$-approximate, and $\delta$-$\epsilon$-approximate queries. 
%%Extending our experimental framework to cover subsequence matching and range queries is straight-forward, and part of our future work. 

\noindent{\textbf{Algorithms.}}
We use the most efficient C/C++ implementation available for each method: %~\cite{url/hnsw,url/srs,url/faiss} and re-implemented in C (from scratch) methods that were available in other languages~\cite{url/DSSeval2}. 
%On all tested datasets, our new implementations achieve higher time and space efficiency than the original ones (see Appendix). 
%We study both exact data series methods %(DSTree~\cite{conf/vldb/Wang2013}, iSAX2+~\cite{journal/kais/Camerra2014}, VA+file~\cite{conf/cikm/Hakan2000}) and approximate methods designed for vectors (HNSW~\cite{journal/corr/malkov16}, the Inverted Multi-Index (IMI), iSAX2+, SRS ). These methods all support the Euclidean distance and are described in detail in Section~\ref{sec:approaches}.
iSAX2+~\cite{url/DSSeval}, DSTree~\cite{url/DSSeval} and VA+file~\cite{url/DSSeval} representing exact data series methods with support for approximate queries; and HNSW~\cite{url/hnsw}, Faiss IMI~\cite{url/faiss}, SRS~\cite{url/srs}, {\color{black} FLANN~\cite{flann}, and QALSH~\cite{qalsh} representing strictly approximate methods for vectors. We ran experiments with the HD-index~\cite{hdindex} and NSG~\cite{nsg}, but since they could not scale for our smallest 25GB dataset, we do not report results for them.}
%All these methods support Euclidean distance and are described in Section~\ref{sec:approaches}. 
%We used the most efficient implementations that we are aware of for each method,
% our C implementations for DSTree, iSAX2+ and VA+file~\cite{url/DSSeval2}, the original code bases for HNSW~\cite{url/hnsw} and SRS~\cite{url/srs}, and the Faiss version of IMI~\cite{url/faiss}. 
We extended DSTree, iSAX2+ and VA+file with Algorithm~\ref{alg:deltaepsilonNN}, approximating $r_{\delta}$ with density histograms on a 100K data series sample, following the C++ implementation of~\cite{conf/icde/Ciaccia2000}. 
All methods are single core implementations, except for HNSW and IMI that make use of multi-threading and SIMD vectorization. 
%IMI additionally exploits the BLAS library and popcount. 
% bring back for camera ready!: We allow each method to leverage its full functionalities. % and we use various metrics, including implementation-independent ones, to guard against bias.
%Our baseline is the Euclidean distance version of the UCR Suite~\cite{conf/kdd/Mueen2012}.
%This is a set of techniques for performing very fast similarity computation scans.
%These optimizations include: a) avoiding the computation of square root on Euclidean distance, b) early abandoning of Euclidean distance calculations, and c) reordering early abandoning on normalized data\footnote{Early abandoning of Z-normalization is not used since all datasets were normalized in advance.}.
%We used these optimizations on all the methods that we examined.
Data series points are %always 
represented using single precision values and methods based on fixed summarizations use 16 dimensions. 
%The same set of known optimizations for data series processing are applied to all methods.

\noindent{\textbf{Datasets.}}
We use synthetic and real datasets. Synthetic datasets, called $Rand$, were generated as random-walks using a summing process with steps following a Gaussian distribution (0,1). 
Such data model financial time series~\cite{conf/sigmod/Faloutsos1994} and have been widely used in the literature~\cite{conf/sigmod/Faloutsos1994,journal/kais/Camerra2014,conf/kdd/Zoumpatianos2015}. 
Our four real datasets cover domains as varied as deep learning, computer vision, seismology, and neuroscience. %The astronomy dataset, \emph{Astro100GB}, contains 100 million data series of size 256 representing celestial objects~\cite{journal/aa/soldi2014}. 
\emph{Deep1B}~\cite{url/data/deep1b} comprises 1 billion vectors of size 96 extracted from the last layers of a convolutional neural network. 
\emph{Sift1B}~\cite{conf/icassp/jegou2011,url/data/sift} consists of 1 billion SIFT vectors of size 128 representing image feature descriptions. 
To the best of our knowledge, these two vector datasets are the largest publicly available real datasets. 
\emph{Seismic100GB}~\cite{url/data/seismic}, contains 100 million data series of size 256 representing earthquake recordings at seismic stations worldwide. 
\emph{Sald100GB}~\cite{url/data/eeg} contains neuroscience MRI data and includes 200 million data series of size 128. 
%We, thereafter, refer to the size of each dataset in GB instead of the number of data series. 
In our experiments, we vary the size of the datasets from 25GB to 250GB. 
The name of each dataset is suffixed with its size. 
%For real datasets, whenever we use the full original size, we refer to the dataset in its original name, for example, Deep1B, otherwise we state the subset size in GB in the suffix. 
%for example Deep25GB is the top 25GB subset of Deep1B.
We do not use other real datasets that have appeared in the literature~\cite{UCRArchive,conf/sisap/martin17}, because they are very small, not exceeding 1GB in size. 


\noindent{\textbf{Queries.}}
%unless otherwise stated, 
All our query workloads consist of 100 query series run asynchronously, i.e., not in batch mode. 
Synthetic queries were generated using the same random-walk generator as the $Rand$ dataset (with a different seed, reported in~\cite{url/DSSeval2}). 
%while $Synth$-$Ctrl$ queries are created by extracting data series from the input data set and adding progressively larger amounts of noise, in order to control the difficulty of each query (more difficult queries tend to be less similar to their nearest neighbor~\cite{johannesjoural2018}).
For the Deep1B and Sift1B datasets, we randomly select 100 queries from the real workloads that come with the datasets archives. For the other real datasets, query workloads were generated by adding progressively larger amounts of noise to data series extracted from the raw data, so as to produce queries having different levels of difficulty, following the ideas in~\cite{johannesjoural2018}. 
%(the less similar a query to its nearest neighbor, the harder it is~\cite{johannesjoural2018}). 
Our experiments cover $ng$-approximate and $\delta$-$\epsilon$-approximate k-NN queries, where k $\in [1,100]$. We also include results for exact queries to serve as a yardstick. 
% to be included in the camera ready version!: {\color{black} All datasets and queries are z-normalized to allow efficient similarity search~\cite{journal/dmkd/Keogh2003}}. 

\noindent{\textbf{Scenarios.}}
{\color{black}Our experimental evaluation proceeds in four main steps: 
%parametrization, evaluation, comparison of all methods and a more detailed comparison of the best disk-based methods. 
(i) we tune methods to their optimal parameters (\S\ref{ssec:parametrization}); (ii) we evaluate the indexing scalability of the methods
% against datasets of various sizes
(\S\ref{ssec:indexing_efficiency}); (iii) we compare in-memory and out-of-core scalability and  accuracy of all methods (\S\ref{ssec:query_efficiency_mem}-\S\ref{ssec:query_efficiency_disk}); and (iv) we perform additional experiments on the best performing methods for disk-resident data (\S\ref{ssec:query_efficiency_disk})}. 


\noindent{\textbf{Measures.}} We assess methods using the following criteria:

\noindent(1) Scalability and search efficiency using: \emph{wall clock time} (input, output, CPU  and total time), \emph{throughput} (\# of queries answered per minute), and two implementation-independent measures: the \emph{number of random disk accesses} (\# of disk seeks) and the \emph{percentage of data accessed}. 
%\emph{Wall clock time} measures input, output and total execution times (CPU time is calculated as by subtracting I/O time from the total time). 
%The \emph{throughput} is the number of queries that can be answered in one minute. 
%The \emph{number of random disk accesses} represents the number of leaf accesses for disk-based indexes, and the number of skips for skip-sequential access methods. 
%To assess the number of sequential I/O accesses, we also report the \emph{percent of data accessed} which corresponds to the fraction of the dataset that is accessed by an algorithm to answer a given query. 
%As will be evident in the results, our measure of random disk accesses provides a good insight into the actual performance of indexes, even though we do not account for details such as caching, the number of disk pages occupied by a leaf and the numbers of leaves in contiguous disk blocks.

\noindent(2) Search accuracy is assessed using: \emph{Avg\_Recall}, \emph{Mean Average Precision (MAP)}, and \emph{Mean Relative Error (MRE)}. Recall is the most commonly used accuracy metric in the approximate similarity search literature. However, since it does not consider rank accuracy, we also use MAP~\cite{conf/sigir/turpin2006} that is popular in information retrieval~\cite{book/manning2008,conf/sigir/buckley2000} {\color{black} and has been proposed recently in the high-dimensional community~\cite{hdindex} as an alternative accuracy measure to recall}. 
For a workload of queries $S_{Q_i} : i \in [1, N_Q]$, these are defined as follows.
\begin{compactitem}
\item $Avg\_Recall(workload) = \sum_{i=1}^{N_Q} Recall(S_{Q_i}) / N_Q $ 
{\color{black}\item $MAP(workload) = \sum_{i=1}^{N_Q} AP(S_{Q_i}) / N_Q $}
\item $MRE(workload) = \sum_{i=1}^{N_Q} RE(S_{Q_i}) / N_Q $
\end{compactitem}
where:  

\noindent$\bullet$  
$Recall(S_{Q_i}) = \frac{\textit{\# true neighbors returned by }{Q_i}}{k}$

\noindent$\bullet$  
$AP(S_{Q_i}) = \frac {\sum_{r=1}^{k} (P(S_{Q_i,r}) \times rel(r))} {k}, \forall i \in [1,N_Q]$ 

$-$  
$P({S_{Q_i}},r) = \frac {\text{\# true neighbors among the first $r$ elements}} {r}$.
%Intuitively, $P(S_{Q_i},r)$ is the precision of query $S_{Q_i}$ at rank $r$ and 

$-$ $rel(r)$ is equal 1 if the neighbor returned at position $r$ 

is one of the $k$ exact neighbors of $S_{Q_i}$ and 0 otherwise.
	
\noindent$\bullet$  
$RE(S_{Q_i}) = \frac{1}{k} \times \sum_{r=1}^{k} \frac {d(S_{Q_i},S_{C_{r}}) - d(S_{Q_i},S_{C_i})} {d(S_{Q_i},S_{C_i})}$. 	
$S_{C_i}$ is the exact nearest neighbor of $S_{Q_i}$ and $S_{C_{r}}$ is the $r$-th NN retrieved\footnote{Note that in Definition~\ref{def:epsmatch}, $\epsilon$ is an upper bound on $RE(S_{Q_i})$.}.
Without loss of generality, we do not consider the case where $d(S_{Q_i},S_{C_i}) = 0$. 
(i.e., range queries with radius zero, or kNN queries where the 1-NN is the query itself\footnote{In these cases, the MRE definition can be extended to use the symmetric mean absolute percentage error~\cite{journal/omega/Flores1986}.}.) 
%Also, we do not consider the absolute value since the retrieved neighbor cannot be closer to the query than its exact neighbor.



\noindent(3) Size, using the \emph{main memory} footprint of the algorithm.

\noindent{\textbf{Procedure.}}
Experiments involve two steps: index building and query answering. Caches are fully cleared before each step, and stay warm between consecutive queries.
For large datasets that do not fit in memory, the effect of caching is minimized for all methods. 
All experiments use workloads of 100 queries. 
Results reported for workloads of 10K queries are extrapolated: we discard the $5$ best and $5$ worst queries of the original 100 (in terms of total execution time), and multiply the average of the 90 remaining queries by 10K. 

\begin{comment}
of In order to evaluate scalability and search efficiency, we use \emph{wall clock time}, and two implementation-independent measures: the \emph{number of random disk accesses} -For kNN: Report the incremental times for answering the NNs, i.e. time to answer 1st NN, then the additional time to answer the 2nd NN and so on. \\
-Verify if delta-epsilon holds for kNN, if so, add proof \\
-add MAP@k measure to compare the order of the NNs (as in HD-index \cite{journal/pvldb/arora2018}:
-Report the percentage of neighbors colocated in the same leaf for the different indexes: it should be higher for indexes with better clustering.
-Report Recall@R = the rate of queries for which true NN is present in a short list of length R. 
-compare recall and throughput vs. maxpoints accessed. (For SRS, use early termination test)
-For accuracy, use the average relative error  and contrat to the overall ratio defined in used in \cite{journal/tods/tao2010,journal/pvldb/arora2018}
-Add a footnote saying we use recall@1. Mention literature usage of recall@R\cite{journal/tpami/jegou2011} and precise that recall@1 is a more sensitive measure. \\
-Do not vary index parameters in plots\\
-Discuss that ann-benchmakrs do this, but for us, large datasets, it is not feasible.\\
-Also add extra time on top of approximate to get exact answer.

\end{comment}


%\noindent3. We also consider the pruning ratio $P$, which has been widely used in the data series literature \cite{journal/kais/Keogh2001,journal/edbt/Schafer2012,conf/vldb/Wang2013,conf/vldb/Ding2008,conf/kdd/Karras2011} as an implementation-independent measure to compare the effectiveness of an index. It is defined as follows: 
%\[P \ = \ 1-\frac{\# \ of \ Raw \ Data \ Series \ Examined}{\# \ of \ Data \ Series \ In \ Dataset} \]
%\[P_{node} \ = \ 1-\frac{\# \ of \ Leaf \ Nodes \ Examined}{\# \ of \ Leaf \ Nodes \ In \ Index} \]
%Pruning ratio is a good indicator of the number of sequential I/Os incurred. However, since relevant data series are usually spread out on disk, it should be considered along with the number of random disk accesses (seeks) performed.

%\noindent4. The \emph{tightness of the lower bound}, $TLB$ has been used in the literature as an implementation independent measure in various different forms~\cite{conf/kdd/shieh1998,journal/edbt/Schafer2012,journal/dmkd/Wang2013}.
    %    the $TLB$ was calculated as follows:
    % 	   	\[TLB \ = \ \frac{Lower \ Bounding \ Distance (Q\prime, C\prime)}{ True \ Distance(Q, C)}  \]
		% Such that $Q$ and $C$ are two data series randomly sampled from the dataset 1000 times (with replacement), $Q\prime$ is the PAA or DFT representation of the query $Q$ and $C\prime$ is the SAX or SFA representation of $C$. It was not explicitly stated whether the minimum, maximum or average value of the $TLB$ is reported.
    % 	 	Whereas in \cite{journal/dmkd/Wang2013}, the $TLB$ was calculated using the formula:
    % 	   	\[TLB \ = \ \frac{Lower \ Bounding \ Distance (Q\prime, N)}{ Minimum \ True \ Distance(Q, N)}  \]
    % 	   Such that $Q$ is the query, $Q\prime$ is the representation of $Q$ using the segmentation of a given leaf node $N$ and the minimum true distance between the query $Q$ and the node $N$ is the smallest Euclidean distance between $Q$ and any data series in $N$. Please note that for the DSTree, the first $TLB$ calculation method is not applicable since the segmentation is node-dependent.
   %      In this work we use the following version of the $TLB$ measure that better captures the performance of indexes:
   % 	   	\[TLB \ = \ \frac{Lower \ Bounding \ Distance (Q\prime, N)}{ Average \ True \ Distance(Q, N)}  \]
   % 	   Where $Q$ is the query, $Q\prime$ is the representation of $Q$ using the segmentation of a given leaf node $N$, and the average true distance between the query $Q$ and node $N$ is the average Euclidean distance between $Q$ and all data series in $N$. We report the average over all leaf nodes for all 100 queries.


%\ifJournal

%\noindent5. The \emph{accuracy of the approximate search} is measured by $\epsilon_{eff}$ defined as follows.
%% \begin{defn} \label{def:effepsilon}
%Given a query data series $S_Q$, an exact match $S_C$ and an approximate match $S_{C_{approx}}$, the \emph{effective error, $\epsilon_{\text{eff}}$} of $S_{C_{approx}}$ is:
%\[\epsilon_{\text{eff}} = \frac {d(S_Q,S_{C_{approx}}) - d(S_Q, S_C)} {d(S_Q,S_C)} \]
%{\color{black} For the special case when $d(Q, C_{approx}) - d(Q,C) =0$, the effective error is defined as: \[\epsilon_eff = d(Q, C_{approx}) - d(Q,C) \]}
% \end{defn}
% It is noteworthy to point out that the $\epsilon$ in Definition~\ref{def:epsmatch} constitutes an upper bound on the actual approximation error $\epsilon_{eff}$ in Definition~\ref{def:effepsilon}.
% Without loss of generality, we do not consider the case where $d(Q,C) = 0$.
% This can happen in range queries with radius zero, or kNN queries where the nearest neighbor is the query point itself.
% In these cases, the definition for effective error can be modified to use the absolute error instead of the relative error.
% Also, we do not consider the absolute value since the difference between the approximate and actual distances is always positive.

%Note that $\epsilon_{eff}$ and $TLB$ are different: $TLB$ measures how close the lower bounding distance $d_{lb}$ is to the real distance $d_{exact}$, whereas $\epsilon_{eff}$ measures how close the approximate distance $d_{approx}$ is to $d_{exact}$. 
%The approximate and lower bounding distances are related as such: 
%The following inequality holds: 
%$d_{lb} \ \leq  \ d_{exact} \leq d_{approx}$.
%\fi

\begin{comment}
\begin{figure*}[tb]
	\captionsetup{justification=centering}
 \begin{subfigure}{\textwidth}
	\centering
  	\hspace{0.5cm}
  	\includegraphics[width=0.5\textwidth]{{exact_datasize_time_idxproc_cache_legend.png}}
  \end{subfigure}	\captionsetup[subfigure]{justification=centering}
%\begin{subfigure}{0.31\textwidth}
\hspace*{\fill} % separation between the subfigures
\hspace*{\fill} % separation between the subfigures

\begin{subfigure}{0.16\textwidth}
	\centering
	\includegraphics[width=\textwidth] {exact_leafsize_time_idxproc_ads+}
	\caption{ADS+\\ Dataset = 100GB}
	\label{fig:exact:leafsize:time:idxproc:ADS+}
\end{subfigure}
%\hspace*{\fill} % separation between the subfigures
%\begin{subfigure}{0.31\textwidth}
\begin{subfigure}{0.16\textwidth}
	\centering
	\includegraphics[width=\textwidth]{exact_leafsize_time_idxproc_dstree}
	\caption{DSTree\\
	Dataset = 100GB}	
	\label{fig:exact:leafsize:time:idxproc:dstree}
\end{subfigure}
%\hspace*{\fill} % separation between the subfigures
%\begin{subfigure}{0.31\textwidth}
\begin{subfigure}{0.16\textwidth}
	\centering
	\includegraphics[width=\textwidth] {{exact_leafsize_time_idxproc_isax2+}}
	\caption{iSAX2+\\
	Dataset = 100GB}	
	\label{fig:exact:leafsize:time:idxproc:iSAX2+}
\end{subfigure}
%\begin{subfigure}{0.31\textwidth}
\begin{subfigure}{0.16\textwidth}
	\centering
	\includegraphics[width=\textwidth] {{exact_leafsize_time_idxproc_m-tree}}
	\caption{M-tree \\
	Dataset = 50GB}			
	\label{fig:exact:leafsize:time:idxproc:mtree}
\end{subfigure}
\begin{subfigure}{0.16\textwidth}
	\centering
	\includegraphics[width=\textwidth] {{exact_leafsize_time_idxproc_r-tree}}
	\caption{R*-tree\\
	Dataset = 50GB}			
	\label{fig:exact:leafsize:time:idxproc:rstree}
\end{subfigure}
\begin{subfigure}{0.16\textwidth}
	\centering
	\includegraphics[width=\textwidth] {{exact_leafsize_time_idxproc_sfa}}
	\caption{SFA trie \\
		Dataset = 100GB}			
	\label{fig:exact:leafsize:time:idxproc:sfa}
\end{subfigure}

\caption{Leaf size parametrization}
%\\(Data Series Length = 256, 100 Exact Queries)
\label{fig:exact:leafsize:time:idxproc}
\end{figure*}
\end{comment}



\subsection{Results}
\label{subsec:results}

%This section presents the results of the experimental evaluation for all scenarios defined in the previous subsection.
%Parametrization, efficiency and scalability test and best methods comparison.












\subsubsection{\textbf{Parametrization}}
\label{ssec:parametrization}

%%%%%%%%%%%%%%%%%%%%%%%%%%%%%%%%%%%%%%%%%%%
%  Figure: Exact Methods  Parameterization%
%%%%%%%%%%%%%%%%%%%%%%%%%%%%%%%%%%%%%%%%%%%
\begin{comment}
\begin{figure*}[tb]
	\captionsetup{justification=centering}
		\begin{subfigure}\textwidth}
		\centering
			%\begin{subfigure}{0.31\textwidth}
			%\vspace{-20pt}
			\centering
			%\includegraphics[scale=0.3]{exact_leafsize_time_idxproc_r-tree}
			\includegraphics[scale=0.6] {{exact_leafsize_time_idxproc_cache_legend}}
			%			{\color{black} x-axis:} Node size (M = (5K,10K,20K,40K) and minimal fillfactor(m=0,0.2,0.3,0.5) \\
			%			{\color{black} y-axis:} Time in Hours (range TBD) \\
			%			{\color{black} curves:} Indexing and Query Processing Detailed Times}
      \label{fig:exact:leafsize:time:idxproc:rstree}
		\end{subfigure}	\captionsetup[subfigure]{justification=centering}
	%\begin{subfigure}{0.31\textwidth}
	\begin{subfigure}{0.05\textwidth}
		\centering
		%	\includegraphics[scale=0.90,trim={0 0 0 0},clip]  {{leafsize_y_axis}}					
		\includegraphics[width=\columnwidth, trim={0 4.2em 0 2.2em},clip] {{leafsize_y_axis}}
					
    %\vspace{4.3em}
	\end{subfigure}
	\begin{subfigure}{0.15\textwidth}
		\centering
		%\includegraphics[scale=0.3] {{exact_leafsize_time_idxproc_ads+}}
		%\vspace{-20pt}
		%\hspace{-25pt}
		\includegraphics[width=\textwidth,trim={0 0 0 2.2em},clip] {{exact_leafsize_time_idxproc_noyaxis_ads+}}
		\caption{ADS+}
		\label{fig:exact:leafsize:time:idxproc:ADS+}
	\end{subfigure}
	%\hspace*{\fill} % separation between the subfigures
	%\begin{subfigure}{0.31\textwidth}
	\begin{subfigure}{0.15\textwidth}
		\centering
		%\vspace{-20pt}
		%\hspace{-50pt}
		\includegraphics[width=\textwidth,trim={0 0 0 2.2em},clip]{exact_leafsize_time_idxproc_noyaxis_dstree}
		\caption{DSTree}
		\label{fig:exact:leafsize:time:idxproc:dstree}
	\end{subfigure}
	%\hspace*{\fill} % separation between the subfigures
	%\begin{subfigure}{0.31\textwidth}
	\begin{subfigure}{0.15\textwidth}
		\centering
		%\vspace{-20pt}
		%\hspace{-30pt}
		%\includegraphics[scale=0.3]{{exact_leafsize_time_idxproc_isax2+}}
		\includegraphics[width=\textwidth,trim={0 0 0 2.2em},clip] {{exact_leafsize_time_idxproc_noyaxis_isax2+}}
		\caption{iSAX2+}
		\label{fig:exact:leafsize:time:idxproc:iSAX2+}
	\end{subfigure}
	%\begin{subfigure}{0.31\textwidth}
	\begin{subfigure}{0.15\textwidth}
				%\vspace{-20pt}
		%		\hspace{-30pt}
		\centering
		%\includegraphics[scale=0.3]{exact_leafsize_time_idxproc_sfa}
		\includegraphics[width=\columnwidth,trim={0 0 0 2.2em},clip] {{exact_leafsize_time_idxproc_noyaxis_sfa}}
		\caption{SFA trie}
		\label{fig:exact:leafsize:time:idxproc:sfa}
	\end{subfigure}
	%\hspace*{\fill}
	%\begin{subfigure}{0.31\textwidth}
	\begin{subfigure}{0.15\textwidth}
				%\vspace{-20pt}
		%\hspace{-30pt}
		\centering
		%\includegraphics[scale=0.3]{exact_leafsize_time_idxproc_m-tree}
		\includegraphics[width=\columnwidth,trim={0 0 0 2.2em},clip] {{exact_leafsize_time_idxproc_noyaxis_m-tree}}
		\caption{M-tree}
		\label{fig:exact:leafsize:time:idxproc:mtree}
	\end{subfigure}
	%\hspace*{\fill}
	\begin{subfigure}{0.15\textwidth}
	%\begin{subfigure}{0.31\textwidth}
		%\vspace{-20pt}
		%\hspace{-30pt}
		\centering
		%\includegraphics[scale=0.3]{exact_leafsize_time_idxproc_r-tree}
		\includegraphics[width=\columnwidth,trim={0 0 0 2.2em},clip] {{exact_leafsize_time_idxproc_noyaxis_r-tree}}
		\caption{R*-tree}
		%			{\color{black} x-axis:} Node size (M = (5K,10K,20K,40K) and minimal fillfactor(m=0,0.2,0.3,0.5) \\
		%			{\color{black} y-axis:} Time in Hours (range TBD) \\
		%			{\color{black} curves:} Indexing and Query Processing Detailed Times}
		\label{fig:exact:leafsize:time:idxproc:rstree}
	\end{subfigure}

	%\hspace*{\fill}

	\begin{comment}
	\begin{subfigure}{0.31\textwidth}
	\centering
	\includegraphics[scale=0.3,trim={0 0 0 2.2em},clip]{exact_leafsize_time_idxproc_mass}
	\caption{MASS}
	%			{\color{black} x-axis:} Min Length of Join Segment\\
	%			{\color{black} y-axis:} Time in Hours\\
	%			{\color{black} curves:} The Query and Pre-Query Times
%	}
	\label{fig:exact:leafsize:time:idxproc:mass}
	\end{subfigure}
	\hspace*{\fill}
	\begin{subfigure}{0.31\textwidth}
	\centering
	\includegraphics[scale=0.3,trim={0 0 0 2.2em},clip]{{exact_leafsize_time_idxproc_ucr-suite}}
	\caption{UCR Suite}
	%			{\color{black} x-axis:} NO PARAMETERS!!\\
	%			{\color{black} y-axis:} Time in Hours\\
	%			{\color{black} curves:} The Query and Pre-Query Times		}
	\label{fig:exact:leafsize:time:idxproc:ucr}
	\end{subfigure}
	\hspace*{\fill}
	\begin{subfigure}{0.31\textwidth}
	\centering
	\includegraphics[scale=0.3,trim={0 0 0 2.2em},clip]{exact_leafsize_time_idxproc_Stepwise}
	\caption{Stepwise}
	%			{\color{black} x-axis:} \# of TS per File per Level \\
	%			{\color{black} y-axis:} Time in Hours\\
	%			{\color{black} curves:} The Query and Pre-Query Times		}
	\label{fig:exact:leafsize:time:idxproc:Stepwise}
	\end{subfigure}
	%\caption{Leaf Size Parameterization}
  %\\%(Dataset Size = 100 GB, Data Series Length = 256, Buffer Size = 60GB, 100 Exact Queries)
	%\label{fig:exact:leafsize:time:idxproc}
\end{figure*}
\end{comment}

{\color{black}
%For  ADS+, the leaf size is 100K, the buffer size is 20GB and the number of segments is 16.
We start by fine tuning each method (graphs omitted for brevity). 
In order to understand the speed/accuracy tradeoffs, we fix the total memory size available to 75GB. 
The optimal parameters for DSTree, iSAX2+ and VA+file are set according to~\cite{journal/pvldb/echihabi2018}. 
For indexing, the buffer and leaf sizes are set to 60GB and 100K, respectively, for both DSTree and iSAX2+. 
iSAX2+ is set to use 16 segments. 
VA+file uses a 20GB buffer and 16 DFT symbols. 
For SRS, we set M (the projected space dimensionality) to 16 so that the representations of all datasets fit in memory. 
The settings were the same for all datasets. 
The fine tuning for HNSW and IMI proved more tricky and involved many testing iterations since the index building parameters strongly affect the speed/accuracy of query answering and differ greatly across datasets. 
For this reason, different parameters were chosen for different datasets. 
For the in-memory method HNSW, we set efConstruction (the number of neighbors considered during index construction) to 500, and M (the number of bi-directional edges created for every new node during indexing) to 4 for the Rand25GB dataset. 
{\color{black} For Deep25GB and Sift25GB, we set efConstruction to 500 and M to 16}.
To tune the Faiss implementation of IMI, we followed the guidelines in~\cite{url/faiss}. 
For the in-memory datasets, we set the index factory key to PQ32\_128,IMI2x12,PQ32 and the training size to 1048576, while for disk based datasets, the index key is PQ32\_128,IMI2x14,PQ32 and the training size 4194304. 
To tune $\delta$-$\epsilon$-approximate search performance and accuracy, we vary $\delta$ and $\epsilon$ for SRS and $\epsilon$ for DSTree, iSAX2+ and VA+file (except in one experiment where we also vary $\delta$). 
For $ng$-approximate search, we vary the $nprobe$ parameter for DSTree/iSAX2+/IMI/VA+file ($nprobe$ represents the number of visited leaves for DSTree/iSAX2+, the number of visited raw series for VA+file, and the number of inverted lists for IMI), and the \emph{efs} parameter for HNSW (which represents the number of non-pruned candidates). %during search).

%We use a synthetic dataset of 100GB with data series of length 256.

%The most critical parameter for these methods is the leaf threshold, i.e., the maximum number of data series that an index leaf can hold.
%We thus vary the leaf size and study the tradeoffs of index construction and query answering for each method.
%Figure~\ref{fig:exact:leafsize:time:idxproc} reports indexing and querying execution times for each method, normalized by the largest total cost.
%The ratio is broken down into CPU and I/O times.
%Figure~\ref{fig:exact:leafsize:time:idxproc:ADS+} shows that the performance of ADS+ is the same across leaf sizes.
%The leaf size affects indexing time, but not query answering.
%This is not visible in the figure, because index construction time is minimal compared to query answering time.  
%This behavior is expected, since ADS+ is an adaptive index,
%A distinctive feature of ADS+ is adaptive node splitting.
%which during querying splits the nodes until a minimal leaf size is reached.
% For all other methods, the leaf size affects both indexing and querying costs.
%For M-tree, larger leaves cause both indexing and querying times to deteriorate.
%For all other methods, increasing the leaf size improves indexing time (because trees are smaller) and querying time (because several series are read together), but once the leaf size goes beyond the optimal leaf size, querying slows down (because some series are unnecessarily read and processed).
%For DSTree,
% Although Figure \ref{fig:exact:leafsize:time:idxproc:dstree} implies that the combined indexing and querying cost is lower for the 150K leaf size,
%the experiments execution logs indicate that querying is faster with the 100K leaf size.
%The optimal leaf size for iSAX2+ is also 100K, for SFA is 1M, and for M-tree and R*-tree are 1 and 50, respectively.

%SFA takes two other parameters: the alphabet size and the binning method. We ran experiments with both equi-depth and equi-width binning, and alphabet sizes from 8 (default value), to 256 (default alphabet size of iSAX2+ and ADS+).
%Alphabet size 8 and equi-depth binning provided the best performance and were thus used for subsequent experiments.

%SFA can also be parametrized by varying the alphabet size and the binning method (equi-width or equi-depth). We ran experiments using both binning methods and alphabet sizes from 8 (default value), to 256 (default alphabet size of iSAX2+ and ADS+).
%Alphabet size 8 and equi-depth binning provided the best performance and were thus used for subsequent experiments.

%Some of the evaluated methods also use internal buffers to manage raw data that do not fit in memory during index building and query processing.
%We ran experiments varying these buffer sizes from 5GB to 60GB.
%The maximum was set to 60GB (recall that total RAM was 75GB).
%All methods benefit from a larger buffer size except ADS+.
%This is because a smaller buffer size allows the OS to use extra memory for file caching during query processing, since ADS+ accesses the raw data file directly.


\subsubsection{\textbf{\color{black}Indexing Efficiency}} %of Individual Methods}}
\label{ssec:indexing_efficiency}
In this section, we evaluate the indexing scalability of each method by varying the dataset size. We used four synthetic datasets of sizes 25GB, 50GB, 100GB and 250GB, two of which fit in memory (total RAM was 75GB).
%The datasets for this experiment were generated using the same synthetic generator described in subsection \ref{subsec:framework}.

\begin{comment}
\begin{figure*}[tb]
	\captionsetup{justification=centering}
	\captionsetup[subfigure]{justification=centering}
	%\begin{subfigure}{0.31\columnwidth}
	%	\centering
	%	\includegraphics[width=\columnwidth]{{exact_datasize_time_indexing_cache_ads+}}
	%	\caption{ADS+}
	%	\label{fig:exact:datasize:time:indexing:cache:ads+}
	%\end{subfigure}
	\begin{subfigure}{0.12\textwidth}
		\centering
		\includegraphics[width=\textwidth ]{exact_datasize_time_indexing_cache_dstree_legend}
		\caption{DSTree}
		\label{fig:exact:datasize:time:indexing:cache:dstree}
	\end{subfigure}
	\begin{subfigure}{0.12\textwidth }
	\centering
	\includegraphics[width=\textwidth ]{{exact_datasize_time_indexing_cache_flann}}
	\caption{FLANN}
	\label{fig:exact:datasize:time:indexing:cache:flann}
	\end{subfigure}
	%\hspace*{\fill} % separation between the subfigures
	%\hspace*{\fill}
	\begin{subfigure}{0.12\textwidth }
		\centering
		\includegraphics[width=\textwidth ]{{exact_datasize_time_indexing_cache_hnsw}}
		\caption{HSNW}
		\label{fig:exact:datasize:time:indexing:cache:hnsw}
	\end{subfigure}
	\begin{subfigure}{0.12\textwidth }
		\centering
		\includegraphics[width=\textwidth ]{{exact_datasize_time_indexing_cache_opq-imi}}
		\caption{IMI}
		\label{fig:exact:datasize:time:indexing:cache:imi}
	\end{subfigure}
	\begin{subfigure}{0.12\textwidth }
		\centering
		\includegraphics[width=\textwidth ]{exact_datasize_time_indexing_cache_isax2+}
		\caption{iSAX2+}
		\label{fig:exact:datasize:time:indexing:cache:iSAX2+}
	\end{subfigure}
	\begin{subfigure}{0.12\textwidth }
	\centering
	\includegraphics[width=\textwidth ]{{exact_datasize_time_indexing_cache_qalsh}}
	\caption{QALSH}
	\label{fig:exact:datasize:time:indexing:cache:qalsh}
	\end{subfigure}	
	\begin{subfigure}{0.12\textwidth }
		\centering
		\includegraphics[width=\textwidth ]{{exact_datasize_time_indexing_cache_lsh-srs}}
		\caption{SRS}
		\label{fig:exact:datasize:time:indexing:cache:srs}
	\end{subfigure}
	%\hspace*{\fill}
	\begin{subfigure}{0.12\textwidth }
		\centering
		\includegraphics[width=\textwidth ]{{exact_datasize_time_indexing_cache_va+file}}
		\caption{VA+file}
		\label{fig:exact:datasize:time:indexing:cache:va+file}
	\end{subfigure}
	%\caption{Indexing scalability with increasing dataset sizes \\
	%	{\color{black} photoshop OPQ and HNSW to indicate parallel}}
	\caption{{\color{black} ({\bf MetaRev-2 and Rev2-W1})Indexing scalability}}	
	\label{fig:exact:datasize:time:indexing:cache}
\end{figure*}
\end{comment}

\begin{comment}
\begin{figure}[tb]
	\captionsetup{justification=centering}
	\captionsetup[subfigure]{justification=centering}
	%\begin{subfigure}{0.31\columnwidth}
	%	\centering
	%	\includegraphics[width=\columnwidth]{{exact_datasize_time_indexing_cache_ads+}}
	%	\caption{ADS+}
	%	\label{fig:exact:datasize:time:indexing:cache:ads+}
	%\end{subfigure}
	\begin{subfigure}{0.005\columnwidth }
		\centering
	%\hspace*{4cm}
	%\vspace*{-1cm}
	%		\includegraphics[width=\columnwidth ]{{exact_datasize_time_indexing_cache_vertical_legend.png}}
		%\includegraphics[scale=0.3]{{exact_datasize_time_indexing_cache_vertical_legend_new.png}}
	\end{subfigure}	\\	
	\begin{subfigure}{0.31\columnwidth}
		\centering
		\includegraphics[width=\columnwidth ]{exact_datasize_time_indexing_cache_dstree_legend}
		\caption{DSTree}
		\label{fig:exact:datasize:time:indexing:cache:dstree}
	\end{subfigure}
	%\hspace*{\fill} % separation between the subfigures
	%\hspace*{\fill}
	\begin{subfigure}{0.31\columnwidth }
		\centering
		\includegraphics[width=\columnwidth ]{{exact_datasize_time_indexing_cache_hnsw}}
		\caption{HSNW}
		\label{fig:exact:datasize:time:indexing:cache:hnsw}
	\end{subfigure}
	\begin{subfigure}{0.31\columnwidth }
	\centering
	\includegraphics[width=\columnwidth ]{{exact_datasize_time_indexing_cache_opq-imi}}
	\caption{IMI}
	\label{fig:exact:datasize:time:indexing:cache:imi}
	\end{subfigure}
	\begin{subfigure}{0.31\columnwidth }
	\centering
	\includegraphics[width=\columnwidth ]{exact_datasize_time_indexing_cache_isax2+}
	\caption{iSAX2+}
	\label{fig:exact:datasize:time:indexing:cache:iSAX2+}
	\end{subfigure}
	\begin{subfigure}{0.31\columnwidth }
		\centering
		\includegraphics[width=\columnwidth ]{{exact_datasize_time_indexing_cache_lsh-srs}}
		\caption{SRS}
		\label{fig:exact:datasize:time:indexing:cache:srs}
	\end{subfigure}
	%\hspace*{\fill}
	\begin{subfigure}{0.31\columnwidth }
		\centering
		\includegraphics[width=\columnwidth ]{{exact_datasize_time_indexing_cache_va+file}}
		\caption{VA+file}
		\label{fig:exact:datasize:time:indexing:cache:va+file}
	\end{subfigure}
	\begin{subfigure}{0.31\columnwidth }
		\centering
		\includegraphics[width=\columnwidth ]{{exact_datasize_time_indexing_cache_qalsh}}
		\caption{QALSH}
		\label{fig:exact:datasize:time:indexing:cache:qalsh}
	\end{subfigure}
	\begin{subfigure}{0.31\columnwidth }
		\centering
		\includegraphics[width=\columnwidth ]{{exact_datasize_time_indexing_cache_flann}}
		\caption{FLANN}
		\label{fig:exact:datasize:time:indexing:cache:flann}
	\end{subfigure}
	%\caption{Indexing scalability with increasing dataset sizes \\
	%	{\color{black} photoshop OPQ and HNSW to indicate parallel}}
	\caption{{\color{black} ({\bf MetaRev-2 and Rev2-W1})Indexing scalability}}	
	\label{fig:exact:datasize:time:indexing:cache}
\end{figure}
\end{comment}
%\noindent\textbf{ADS+.} Figure~\ref{fig:exact:datasize:time:indexing:cache:ads+} shows that index building with ADS+ is extremely fast taking less than 10 minutes for the 250GB dataset. This is because ADS+ is an adaptive index that builds the index tree using only summarizations thus reducing I/O cost significantly.

\begin{comment}
{\color{black}\noindent\textbf{FLANN.} FLANN builds the index for the 25GB dataset quickly but does not scale for the 50GB dataset (Figure~\ref{fig:exact:datasize:time:indexing:cache:flann}). We think the reason is there is a memory management issue with the code since it swaps heavily although the index size on disk is only 8GB and the data is 25GB (FLANN loads the full dataset in-memory).}

\noindent\textbf{DSTree.} DSTree's index building algorithm is more expensive, requiring about 4 hours for the largest dataset (Figure~\ref{fig:exact:datasize:time:indexing:cache:dstree}). Nevertheless, the cost is dominated by CPU time. Thus, parallelization can be used to improve this method.

\noindent\textbf{HNSW.} HNSW is an in-memory method that requires both the dataset and the graph structure to fit in memory. Therefore, given that the total RAM was 75GB, we were able to build the HNSW index only for the 25GB dataset. Despite the use of multi-threading, index building is extremely slow, taking over 10 hours for the smallest dataset (Figure~\ref{fig:exact:datasize:time:indexing:cache:hnsw}).

\noindent\textbf{IMI.} Although our Faiss IMI implementation exploits multi-threading %, popcount, 
and SIMD vectorization %and the OpenBLAS libary, 
training and building the IMI index was very slow, requiring over 15 hours on the 250GB dataset (Figure~\ref{fig:exact:datasize:time:indexing:cache:imi}).
%We used the recommended max\_code for each dataset size and varied nprobe to adjust the speed-accuracy tradeoff.

\noindent\textbf{iSAX2+.} In Figure~\ref{fig:exact:datasize:time:indexing:cache:iSAX2+}, we can see that iSAX2+ is efficient at index building, requiring less than an hour for the largest dataset. The cost of index building is mostly I/O.

%The HD-index only has numbers for 1M SIFT for OPQ and HNSW, they said they could not construct the index due to a crash.

%time consuming to parameterize because index construction is slow and the index has to be constructed for different parameters: M and efConstruction.

{\color{black}\noindent\textbf{QALSH.} In Figure~\ref{fig:exact:datasize:time:indexing:cache:qualsh}, we can see that the indexing time for QALSH doubles between 25GB and 50GB and is mainly I/O cost.} 

\noindent\textbf{SRS.} 
%We used M=16 to fit the index in memory and varied delta and epsilon to adjust speed-accuracy
The results for SRS are depicted in Figure~\ref{fig:exact:datasize:time:indexing:cache:srs}. 
We used the most efficient implementation provided by the authors and the largest value of M that ensured best accuracy and minimum index size.
Even tough index building in memory is efficient, swapping issues occur for the larger datasets. 

\noindent\textbf{VA+file.} Figure~\ref{fig:exact:datasize:time:indexing:cache:va+file} shows that VA+file performs well at index building, with most of the total cost being CPU (optimal bit allocation and interval decision for each dimension.) This method also offers good opportunities for parallelization.

%In PAC queries paper Patella claims that Unfortunately, AC-NN algorithms are still plagued by the dimensionality curse and become unpractical when D is
%intrinsically high, regardless of epsilon. We show that this is not always the case. The results obtained with epsilon are competitive with the SOTA in vecor indexing and we further use delta to improve the numbers.


%\noindent\textbf{Summary.}
%Overall, Figure~\ref{fig:exact:datasize:time:indexing:cache} shows that it takes Stepwise, MASS, the M-tree and the R*-tree over 12 hours to complete the workload for the 250GB dataset, whereas the other methods need less than 7 hours.
% Moreover, although the performance of Stepwise improves with longer series (see Figure \ref{fig:exact:length:time:idxproc:Stepwise}), it still is not competitive.
% We conclude that Stepwise, MASS, the M-tree and the R*-tree do not scale well with larger datasets on whole matching queries;
%Therefore, in the subsequent experiments, we will only include ADS+, DSTree, iSAX2+, SFA, the UCR suite and the VA+file. %We will choose the UCR suite as the baseline, since it is the best sequential method available.

{\color{black}\noindent\textbf{Discussion.}}
Figure~\ref{fig:exact:datasize:time:indexing:cache} shows that iSAX2+ is the fastest method at index building in and out of memory, followed by VA+file, SRS, DSTree, {\color{black}FLANN}, QALSH, IMI and HNSW. 
Even though IMI and HNSW are the only parallel methods, they are the slowest at index building. {\color{black}Although FLANN is slow at indexing the 50GB dataset, we think this is more due to memory management issues in the code which causes swapping.}
For HNSW, the major cost is building the graph structure, whereas IMI spends most of the time on determining the clusters and computing the product quantizers. 
In terms of footprint, the DSTree is the most memory-efficient, followed by iSAX2+. 
IMI, SRS, VA+file {\color{black}and FLANN} are two orders of magnitude larger, while {\color{black}QALSH} and HNSW are a further order of magnitude bigger (Figure~\ref{fig:exact:datasize:memory:indexing:cache}). 
\end{comment}
{\color{black}
Figure~\ref{fig:exact:datasize:time:indexing:cache} shows that iSAX2+ is the fastest method at index building in and out of memory, followed by VA+file, SRS, DSTree, {\color{black}FLANN}, QALSH, IMI and HNSW. 
Even though IMI and HNSW are the only parallel methods, they are the slowest at index building. {\color{black}Although FLANN is slow at indexing the 50GB dataset, we think this is more due to memory management issues in the code, which cause swapping.}
For HNSW, the major cost is building the graph structure, whereas IMI spends most of the time on determining the clusters and computing the product quantizers. We also measured the breakdown of the indexing time and found out that all methods can be significantly improved by parallelism except iSAX2+ and QALSH that are I/O bound.
In terms of footprint, the DSTree is the most memory-efficient, followed by iSAX2+. 
IMI, SRS, VA+file {\color{black}and FLANN} are two orders of magnitude larger, while {\color{black}QALSH} and HNSW are a further order of magnitude bigger (Figure~\ref{fig:exact:datasize:memory:indexing:cache}).}

\begin{figure}[!htb]
	\captionsetup{justification=centering}
	\captionsetup[subfigure]{justification=centering}
	\begin{subfigure}{\columnwidth }
		\centering
		%\hspace*{4cm}
		%\vspace*{-1cm}
		%		\includegraphics[width=\columnwidth ]{{exact_datasize_time_indexing_cache_vertical_legend.png}}
		\includegraphics[scale=0.14]{{full_epsilon_legend_25GB}}
	\end{subfigure}		
	\begin{subfigure}{0.49\columnwidth }
		\centering
		%\hspace*{1cm}
		\includegraphics[width=\columnwidth ]{{exact_datasize_time_combined}}
		%\includegraphics[scale=0.30]{{exact_datasize_time_combined}}	
		\caption{Indexing time}
		\label{fig:exact:datasize:time:indexing:cache}
	\end{subfigure}
	\begin{subfigure}{0.49\columnwidth }
		\centering
		%\hspace*{1cm}
		\includegraphics[width=\columnwidth ]{{exact_datasize_memory_combined}}
		%\includegraphics[scale=0.30]{{exact_datasize_memory_combined}}		
		\caption{Size in memory}
		\label{fig:exact:datasize:memory:indexing:cache}
	\end{subfigure}
	%\caption{Comparison of indexing scalability with increasing dataset sizes \\
	%	{\color{black} photoshop OPQ and HNSW to indicate parallel}}
	\caption{{\color{black} Comparison of indexing scalability}}	
	\label{fig:exact:datasize:time:memory:indexing:combined:cache}
\end{figure}

\begin{comment}
\begin{figure*}[tb]
	\captionsetup{justification=centering}
	\captionsetup[subfigure]{justification=centering}
	\begin{subfigure}{\textwidth}
		\centering
		%\hspace*{1cm}
		\includegraphics[scale=0.23]{{full_epsilon_legend_new}}\\
	\end{subfigure}	
	\begin{subfigure}{0.16\textwidth}
		\centering
		\includegraphics[width=\textwidth]{{exact_mapk_throughput_synthetic_25GB_256_ng_1NN_100_nocache}}
		\caption{1-NN (ng)} 
		\label{fig:approx:accuracy:qefficiency:synthetic:25GB:256:hdd:ng:1NN:100:nocache}
	\end{subfigure}
	\begin{subfigure}{0.16\textwidth}
		\centering
		\includegraphics[width=\textwidth]{exact_mapk_throughput_synthetic_25GB_256_de_1NN_100_nocache}
		\caption{1-NN ($\bm{\delta\epsilon}$)} 
		\label{fig:approx:accuracy:qefficiency:synthetic:25GB:256:hdd:de:1NN:100:nocache}
	\end{subfigure}
	\begin{subfigure}{0.16\textwidth}
		\centering
		\includegraphics[width=\textwidth]{exact_mapk_throughput_synthetic_25GB_256_ng_10NN_100_nocache}
		\caption{10-NN (ng)} 
		\label{fig:approx:accuracy:qefficiency:synthetic:25GB:256:hdd:ng:10NN:100:nocache}
	\end{subfigure}
	%\hspace*{\fill}
	\begin{subfigure}{0.16\textwidth}
		\centering
		\includegraphics[width=\textwidth]{{exact_mapk_throughput_synthetic_25GB_256_de_10NN_100_nocache}}
		\caption{10-NN ($\bm{\delta\epsilon}$)} 
		\label{fig:approx:accuracy:qefficiency:synthetic:25GB:256:hdd:de:10NN:100:nocache}
	\end{subfigure}
	\begin{subfigure}{0.16\textwidth}
		\centering
		\includegraphics[width=\textwidth]{exact_mapk_throughput_synthetic_25GB_256_ng_100NN_100_nocache}
		\caption{100-NN (ng)} 
		\label{fig:approx:accuracy:qefficiency:synthetic:25GB:256:hdd:ng:100NN:100:nocache}
	\end{subfigure}
	\begin{subfigure}{0.16\textwidth}
		\centering
		\includegraphics[width=\textwidth]{exact_mapk_throughput_synthetic_25GB_256_de_100NN_100_nocache}
		\caption{100-NN ($\bm{\delta\epsilon}$)} 
		\label{fig:approx:accuracy:qefficiency:synthetic:25GB:256:hdd:de:100NN:100:nocache}
	\end{subfigure}
	\begin{subfigure}{0.16\textwidth}
		\centering
		\includegraphics[width=\textwidth]{{exact_mapk_idxproc_synthetic_25GB_256_ng_1NN_100_nocache}}
		\caption{1-NN (ng)} 
		\label{fig:approx:accuracy:efficiency:synthetic:25GB:256:hdd:ng:1NN:100:nocache}
	\end{subfigure}
	\begin{subfigure}{0.16\textwidth}
		\centering
		\includegraphics[width=\textwidth]{exact_mapk_idxproc_synthetic_25GB_256_de_1NN_100_nocache}
		\caption{1-NN ($\bm{\delta\epsilon}$)} 
		\label{fig:approx:accuracy:efficiency:synthetic:25GB:256:hdd:de:1NN:100:nocache}
	\end{subfigure}
	\begin{subfigure}{0.16\textwidth}
		\centering
		\includegraphics[width=\textwidth]{exact_mapk_idxproc_synthetic_25GB_256_ng_10NN_100_nocache}
		\caption{10-NN (ng)} 
		\label{fig:approx:accuracy:efficiency:synthetic:25GB:256:hdd:ng:10NN:100:nocache}
	\end{subfigure}
	\begin{subfigure}{0.16\textwidth}
		\centering
		\includegraphics[width=\textwidth]{{exact_mapk_idxproc_synthetic_25GB_256_de_10NN_100_nocache}}
		\caption{10-NN ($\bm{\delta\epsilon}$)} 
		\label{fig:approx:accuracy:efficiency:synthetic:25GB:256:hdd:de:10NN:100:nocache}
	\end{subfigure}
	\begin{subfigure}{0.16\textwidth}
		\centering
		\includegraphics[width=\textwidth]{{exact_mapk_idxproc_synthetic_25GB_256_ng_100NN_100_nocache}}
		\caption{100-NN (ng)} 
		\label{fig:approx:accuracy:efficiency:synthetic:25GB:256:hdd:ng:100NN:100:nocache}
	\end{subfigure}
	\begin{subfigure}{0.16\textwidth}
		\centering
		\includegraphics[width=\textwidth]{exact_mapk_idxproc_synthetic_25GB_256_de_100NN_100_nocache}
		\caption{100-NN ($\bm{\delta\epsilon}$)} 
		\label{fig:approx:accuracy:efficiency:synthetic:25GB:256:hdd:de:100NN:100:nocache}
	\end{subfigure}
	\begin{subfigure}{0.16\textwidth}
		\centering
		\includegraphics[width=\textwidth]{{exact_mapk_idxproc_synthetic_25GB_256_ng_1NN_10000_nocache}}
		\caption{1-NN (ng)} 
		\label{fig:approx:accuracy:efficiency:synthetic:25GB:256:hdd:ng:1NN:10K:nocache}
	\end{subfigure}
	\begin{subfigure}{0.16\textwidth}
		\centering
		\includegraphics[width=\textwidth]{exact_mapk_idxproc_synthetic_25GB_256_de_1NN_10000_nocache}
		\caption{1-NN ($\bm{\delta\epsilon}$)} 
		\label{fig:approx:accuracy:efficiency:synthetic:25GB:256:hdd:de:1NN:10K:nocache}
	\end{subfigure}
	\begin{subfigure}{0.16\textwidth}
		\centering
		\includegraphics[width=\textwidth]{exact_mapk_idxproc_synthetic_25GB_256_ng_10NN_10000_nocache}
		\caption{10-NN (ng)} 
		\label{fig:approx:accuracy:efficiency:synthetic:25GB:256:hdd:ng:10NN:10K:nocache}
	\end{subfigure}
	\begin{subfigure}{0.16\textwidth}
		\centering
		\includegraphics[width=\textwidth]{{exact_mapk_idxproc_synthetic_25GB_256_de_10NN_10000_nocache}}
		\caption{10-NN ($\bm{\delta\epsilon}$)} 
		\label{fig:approx:accuracy:efficiency:synthetic:25GB:256:hdd:de:10NN:10K:nocache}
	\end{subfigure}
	\begin{subfigure}{0.16\textwidth}
		\centering
		\includegraphics[width=\textwidth]{exact_mapk_idxproc_synthetic_25GB_256_ng_100NN_10000_nocache}
		\caption{100-NN (ng)} 
		\label{fig:approx:accuracy:efficiency:synthetic:25GB:256:hdd:ng:100NN:10K:nocache}
	\end{subfigure}
	\begin{subfigure}{0.16\textwidth}
		\centering
		\includegraphics[width=\textwidth]{exact_mapk_idxproc_synthetic_25GB_256_de_100NN_10000_nocache}
		\caption{100-NN ($\bm{\delta\epsilon}$)} 
		\label{fig:approx:accuracy:efficiency:synthetic:25GB:256:hdd:de:100NN:10K:nocache}
	\end{subfigure}
	\caption{Efficiency vs. accuracy\\
		(Dataset = Rand25GB, Data series length=256)}	
	\label{fig:approx:accuracy:efficiency:synthetic:25GB:256:hdd}
\end{figure*}
\end{comment}

\begin{comment}
\begin{figure*}[tb]
	\captionsetup{justification=centering}
	\captionsetup[subfigure]{justification=centering}
	\begin{subfigure}{\textwidth}
		\centering
		%\hspace*{1cm}
		\includegraphics[scale=0.23]{{full_epsilon_legend_new}}\\
	\end{subfigure}	
	\begin{subfigure}{0.16\textwidth}
		\centering
		\includegraphics[width=\textwidth]{exact_mapk_throughput_synthetic_25GB_256_ng_100NN_100_nocache}
		\caption{100-NN (ng)} 
		\label{fig:approx:accuracy:qefficiency:synthetic:25GB:256:hdd:ng:100NN:100:nocache}
	\end{subfigure}
	\begin{subfigure}{0.16\textwidth}
		\centering
		\includegraphics[width=\textwidth]{exact_mapk_throughput_synthetic_25GB_256_de_100NN_100_nocache}
		\caption{100-NN ($\bm{\delta\epsilon}$)} 
		\label{fig:approx:accuracy:qefficiency:synthetic:25GB:256:hdd:de:100NN:100:nocache}
	\end{subfigure}
	\begin{subfigure}{0.16\textwidth}
		\centering
		\includegraphics[width=\textwidth]{{exact_mapk_idxproc_synthetic_25GB_256_ng_100NN_100_nocache}}
		\caption{100-NN (ng)} 
		\label{fig:approx:accuracy:efficiency:synthetic:25GB:256:hdd:ng:100NN:100:nocache}
	\end{subfigure}
	\begin{subfigure}{0.16\textwidth}
		\centering
		\includegraphics[width=\textwidth]{exact_mapk_idxproc_synthetic_25GB_256_de_100NN_100_nocache}
		\caption{100-NN ($\bm{\delta\epsilon}$)} 
		\label{fig:approx:accuracy:efficiency:synthetic:25GB:256:hdd:de:100NN:100:nocache}
	\end{subfigure}
	\begin{subfigure}{0.16\textwidth}
		\centering
		\includegraphics[width=\textwidth]{exact_mapk_idxproc_synthetic_25GB_256_ng_100NN_10000_nocache}
		\caption{100-NN (ng)} 
		\label{fig:approx:accuracy:efficiency:synthetic:25GB:256:hdd:ng:100NN:10K:nocache}
	\end{subfigure}
	\begin{subfigure}{0.16\textwidth}
		\centering
		\includegraphics[width=\textwidth]{exact_mapk_idxproc_synthetic_25GB_256_de_100NN_10000_nocache}
		\caption{100-NN ($\bm{\delta\epsilon}$)} 
		\label{fig:approx:accuracy:efficiency:synthetic:25GB:256:hdd:de:100NN:10K:nocache}
	\end{subfigure}
	\caption{Efficiency vs. accuracy\\
		(Dataset = Rand25GB, Data series length=256)}	
	\label{fig:approx:accuracy:efficiency:synthetic:25GB:256:hdd}
\end{figure*}
\end{comment}


















\subsubsection{\textbf{Query Answering Efficiency and Accuracy: in-Memory Datasets}}
\label{ssec:query_efficiency_mem}

{\color{black}
	We now compare query answering efficiency and accuracy, in addition to the indexing time, thus, measuring how well each method amortizes index construction time over a large number of queries, and the level of accuracy achieved.
}
%according to different criteria: scalability and search efficiency on more complex workloads and on different hardware platforms, number of random and sequential disk accesses, memory and disk footprint, pruning ratio, tightness of the lower bound and accuracy of the approximate search.

\noindent\textbf{Summary.} 
For our in-memory experiments, we used four datasets of 25GB each: two synthetic (with series of length 256 and 16384, respectively), and two real: Deep25GB and Sift25GB. 
We ran 1NN, 10NN and {\color{black} 100NN} queries on the four datasets and we observed that, while the running times increase with k, the relative performance of the methods stays the same. 
Due to lack of space, Figure~\ref{fig:approx:accuracy:efficiency:synthetic:25GB:inmemory:hdd} shows the 100NN query results only (full results are in~\cite{url/DSSeval2}), which we discuss below. 
{\color{black} Note that  HNSW, QALSH and FLANN store all raw data in-memory, % (this is by design for HNSW and a limitation of the implementation for FLANN and QALSH), 
while all other approaches use the memory to store their data structures, but read the raw data from disk; IMI does not access the raw data at all (it only uses the in-memory summaries).}
%\sout{, but can only answer ng-approximate queries}).}

\noindent\textbf{Short Series}. 
For $ng$-approximate queries of length $256$ on the Rand25GB dataset, HNSW has the largest throughput for any given accuracy, followed by {\color{black}FLANN}, IMI, DSTree and iSAX2+ (Figure~\ref{fig:approx:accuracy:qefficiency:synthetic:25GB:256:hdd:ng:100NN:100:nocache}). 
However, HNSW does not reach a MAP of 1, which is only obtained by the data series indexes (DSTree, iSAX2+, VA+file). The skip-sequential method VA+file performs poorly on approximate search since it prunes per series and not per cluster like the tree-based methods do. 
When indexing time is also considered, iSAX2+ wins for the workload consisting of 100 queries (Figure~\ref{fig:approx:accuracy:efficiency:synthetic:25GB:256:hdd:ng:100NN:100:nocache}), and DSTree for the 10K queries (Figure~\ref{fig:approx:accuracy:efficiency:synthetic:25GB:256:hdd:ng:100NN:10K:nocache}). 

Regarding $\delta$-$\epsilon$-approximate search, DSTree offers the best throughput/accuracy tradeoff, followed by iSAX2+, SRS, {\color{black}VA+file and finally QALSH}. 
SRS does not achieve a MAP higher than 0.5, while DSTree and iSAX2+ are at least 3 times faster than SRS for a similar accuracy (Figure~\ref{fig:approx:accuracy:qefficiency:synthetic:25GB:256:hdd:de:100NN:100:nocache}). 
When we consider the combined indexing and querying times, iSAX2+ wins over all methods for 100 queries (Figure~\ref{fig:approx:accuracy:efficiency:synthetic:25GB:256:hdd:de:100NN:100:nocache}), and DSTree wins for 10K queries (Figure~\ref{fig:approx:accuracy:efficiency:synthetic:25GB:256:hdd:de:100NN:10K:nocache}). 


\begin{figure*}[!htb]
	\captionsetup{justification=centering}
	\captionsetup[subfigure]{justification=centering}
	\begin{subfigure}{\textwidth}
		\centering
		%\hspace*{1cm}
		\includegraphics[scale=0.18]{{full_epsilon_legend_25GB}}\\
	\end{subfigure}	
	\begin{comment}
	\begin{subfigure}{0.16\textwidth}
		\centering
		\includegraphics[width=\textwidth]{exact_mapk_throughput_sift_10GB_128_ng_100NN_100_nocache}
		\scriptsize \caption{Sift10GB (ng)} 
		\label{fig:approx:accuracy:qefficiency:sift:10GB:128:hdd:ng:100NN:100:nocache}
	\end{subfigure}
	\begin{subfigure}{0.16\textwidth}
		\centering
		\includegraphics[width=\textwidth]{exact_mapk_throughput_sift_10GB_128_de_100NN_100_nocache}
		\scriptsize \caption{Sift10GB ($\bm{\delta\epsilon}$)} 
		\label{fig:approx:accuracy:qefficiency:sift:10GB:128:hdd:de:100NN:100:nocache}
	\end{subfigure}
	\begin{subfigure}{0.16\textwidth}
		\centering
		\includegraphics[width=\textwidth]{{exact_mapk_idxproc_sift_10GB_128_ng_100NN_100_nocache}}
		\scriptsize \caption{Sift10GB (ng)} 
		\label{fig:approx:accuracy:efficiency:sift:10GB:128:hdd:ng:100NN:100:nocache}
	\end{subfigure}
	\begin{subfigure}{0.16\textwidth}
		\centering
		\includegraphics[width=\textwidth]{exact_mapk_idxproc_sift_10GB_128_de_100NN_100_nocache}
		\scriptsize \caption{Sift10GB ($\bm{\delta\epsilon}$)} 
		\label{fig:approx:accuracy:efficiency:sift:10GB:128:hdd:de:100NN:100:nocache}
	\end{subfigure}
	\begin{subfigure}{0.16\textwidth}
		\centering
		\includegraphics[width=\textwidth]{exact_mapk_idxproc_sift_10GB_128_ng_100NN_10000_nocache}
		\scriptsize \caption{Sift10GB (ng)} 
		\label{fig:approx:accuracy:efficiency:sift:10GB:128:hdd:ng:100NN:10K:nocache}
	\end{subfigure}
	\begin{subfigure}{0.16\textwidth}
		\centering
		\includegraphics[width=\textwidth]{exact_mapk_idxproc_sift_10GB_256_de_100NN_10000_nocache}
		\scriptsize \caption{Sift10GB ($\bm{\delta\epsilon}$)} 
		\label{fig:approx:accuracy:efficiency:sift:10GB:128:hdd:de:100NN:10K:nocache}
	\end{subfigure}
	\end{comment}	
	\begin{subfigure}{0.16\textwidth}
		\centering
		\includegraphics[width=\textwidth]{exact_mapk_throughput_synthetic_25GB_256_ng_100NN_100_nocache}
		\scriptsize \caption{Rand25GB\\256 (ng)} 
		\label{fig:approx:accuracy:qefficiency:synthetic:25GB:256:hdd:ng:100NN:100:nocache}
	\end{subfigure}
	\begin{subfigure}{0.16\textwidth}
		\centering
		\includegraphics[width=\textwidth]{exact_mapk_throughput_synthetic_25GB_256_de_100NN_100_nocache}
		\scriptsize \caption{Rand25GB\\256 ($\bm{\delta\epsilon}$)} 
		\label{fig:approx:accuracy:qefficiency:synthetic:25GB:256:hdd:de:100NN:100:nocache}
	\end{subfigure}
	\begin{subfigure}{0.16\textwidth}
		\centering
		\includegraphics[width=\textwidth]{{exact_mapk_idxproc_synthetic_25GB_256_ng_100NN_100_nocache}}
		\scriptsize \caption{Rand25GB\\256 (ng)} 
		\label{fig:approx:accuracy:efficiency:synthetic:25GB:256:hdd:ng:100NN:100:nocache}
	\end{subfigure}
	\begin{subfigure}{0.16\textwidth}
		\centering
		\includegraphics[width=\textwidth]{exact_mapk_idxproc_synthetic_25GB_256_de_100NN_100_nocache}
		\scriptsize \caption{Rand25GB\\256 ($\bm{\delta\epsilon}$)} 
		\label{fig:approx:accuracy:efficiency:synthetic:25GB:256:hdd:de:100NN:100:nocache}
	\end{subfigure}
	\begin{subfigure}{0.16\textwidth}
		\centering
		\includegraphics[width=\textwidth]{exact_mapk_idxproc_synthetic_25GB_256_ng_100NN_10000_nocache}
		\scriptsize \caption{Rand25GB\\256 (ng)} 
		\label{fig:approx:accuracy:efficiency:synthetic:25GB:256:hdd:ng:100NN:10K:nocache}
	\end{subfigure}
	\begin{subfigure}{0.16\textwidth}
		\centering
		\includegraphics[width=\textwidth]{exact_mapk_idxproc_synthetic_25GB_256_de_100NN_10000_nocache}
		\scriptsize \caption{Rand25GB\\256 ($\bm{\delta\epsilon}$)} 
		\label{fig:approx:accuracy:efficiency:synthetic:25GB:256:hdd:de:100NN:10K:nocache}
	\end{subfigure}
	\begin{subfigure}{0.16\textwidth}
		\centering
		\includegraphics[width=\textwidth]{exact_mapk_throughput_synthetic_25GB_16384_ng_100NN_100_nocache}
		\scriptsize \caption{Rand25GB\\16384 (ng)} 
		\label{fig:approx:accuracy:qefficiency:synthetic:25GB:16384:ng:hdd:100NN:100:nocache}
	\end{subfigure}
	\begin{subfigure}{0.16\textwidth}
		\centering
		\includegraphics[width=\textwidth]{exact_mapk_throughput_synthetic_25GB_16384_de_100NN_100_nocache}
		\scriptsize \caption{Rand25GB\\16384 ($\bm{\delta\epsilon}$)} 
		\label{fig:approx:accuracy:qefficiency:synthetic:25GB:16384:de:hdd:100NN:100:nocache}
	\end{subfigure}
	\begin{subfigure}{0.16\textwidth}
		\centering
		\includegraphics[width=\textwidth]{exact_mapk_idxproc_synthetic_25GB_16384_ng_100NN_100_nocache}
		\scriptsize \caption{Rand25GB\\16384 (ng)} 
		\label{fig:approx:accuracy:efficiency:synthetic:25GB:16384:ng:hdd:100NN:100:nocache}
	\end{subfigure}
	\begin{subfigure}{0.16\textwidth}
		\centering
		\includegraphics[width=\textwidth]{exact_mapk_idxproc_synthetic_25GB_16384_de_100NN_100_nocache}
		\scriptsize \caption{Rand25GB\\16384 ($\bm{\delta\epsilon}$)} 
		\label{fig:approx:accuracy:efficiency:synthetic:25GB:16384:de:hdd:100NN:100:nocache}
	\end{subfigure}
	\begin{subfigure}{0.16\textwidth}
		\centering
		\includegraphics[width=\textwidth]{exact_mapk_idxproc_synthetic_25GB_16384_ng_100NN_10000_nocache}
		\scriptsize \caption{Rand25GB\\16384 (ng)} 
		\label{fig:approx:accuracy:efficiency:synthetic:25GB:16384:ng:hdd:100NN:10K:nocache}
	\end{subfigure}
	\begin{subfigure}{0.16\textwidth}
		\centering
		\includegraphics[width=\textwidth]{exact_mapk_idxproc_synthetic_25GB_16384_de_100NN_10000_nocache}
		\scriptsize \caption{Rand25GB\\16384($\bm{\delta\epsilon}$)} 
		\label{fig:approx:accuracy:efficiency:synthetic:25GB:16384:de:hdd:100NN:10K:nocache}
	\end{subfigure}
	\begin{subfigure}{0.16\textwidth}
		\centering
		\includegraphics[width=\textwidth]{exact_mapk_throughput_sift_25GB_128_ng_100NN_100_nocache}
		\scriptsize \caption{Sift25GB(ng)} 
		\label{fig:approx:accuracy:qefficiency:sift:25GB:ng:hdd:100NN:100:nocache}
	\end{subfigure}
	\begin{subfigure}{0.16\textwidth}
		\centering
		\includegraphics[width=\textwidth]{exact_mapk_throughput_sift_25GB_128_de_100NN_100_nocache}
		\scriptsize \caption{Sift25GB($\bm{\delta\epsilon}$)} 
		\label{fig:approx:accuracy:qefficiency:sift:25GB:de:hdd:100NN:100:nocache}
	\end{subfigure}
	\begin{subfigure}{0.16\textwidth}
		\centering
		\includegraphics[width=\textwidth]{exact_mapk_idxproc_sift_25GB_128_ng_100NN_100_nocache}
		\scriptsize \caption{Sift25GB(ng)} 
		\label{fig:approx:accuracy:efficiency:sift:25GB:ng:hdd:100NN:100:nocache}
	\end{subfigure}
	\begin{subfigure}{0.16\textwidth}
		\centering
		\includegraphics[width=\textwidth]{exact_mapk_idxproc_sift_25GB_128_de_100NN_100_nocache}
		\scriptsize \caption{Sift25GB($\bm{\delta\epsilon}$)} 
		\label{fig:approx:accuracy:efficiency:sift:25GB:de:hdd:100NN:100:nocache}
	\end{subfigure}
	\begin{subfigure}{0.16\textwidth}
		\centering
		\includegraphics[width=\textwidth]{exact_mapk_idxproc_sift_25GB_128_ng_100NN_10000_nocache}
		\scriptsize \caption{Sift25GB(ng)} 
		\label{fig:approx:accuracy:efficiency:sift:25GB:ng:hdd:100NN:10K:nocache}
	\end{subfigure}
	\begin{subfigure}{0.16\textwidth}
		\centering
		\includegraphics[width=\textwidth]{exact_mapk_idxproc_sift_25GB_128_de_100NN_10000_nocache}
		\scriptsize \caption{Sift25GB($\bm{\delta\epsilon}$)} 
		\label{fig:approx:accuracy:efficiency:sift:25GB:de:hdd:100NN:10K:nocache}
	\end{subfigure}
	\begin{subfigure}{0.16\textwidth}
		\centering
		\includegraphics[width=\textwidth]{exact_mapk_throughput_deep_25GB_96_ng_100NN_100_nocache}
		\scriptsize \caption{Deep25GB(ng)} 
		\label{fig:approx:accuracy:qefficiency:deep:25GB:96:hdd:ng:100NN:100:nocache}
	\end{subfigure}
	\begin{subfigure}{0.16\textwidth}
		\centering
		\includegraphics[width=\textwidth]{exact_mapk_throughput_deep_25GB_96_de_100NN_100_nocache}
		\scriptsize \caption{Deep25GB($\bm{\delta\epsilon}$)} 
		\label{fig:approx:accuracy:qefficiency:deep:25GB:96:hdd:de:100NN:100:nocache}
	\end{subfigure}
	\begin{subfigure}{0.16\textwidth}
		\centering
		\includegraphics[width=\textwidth]{{exact_mapk_idxproc_deep_25GB_96_ng_100NN_100_nocache}}
		\scriptsize \caption{Deep25GB(ng)} 
		\label{fig:approx:accuracy:efficiency:deep:25GB:96:hdd:ng:100NN:100:nocache}
	\end{subfigure}
	\begin{subfigure}{0.16\textwidth}
		\centering
		\includegraphics[width=\textwidth]{exact_mapk_idxproc_deep_25GB_96_de_100NN_100_nocache}
		\scriptsize \caption{Deep25GB($\bm{\delta\epsilon}$)} 
		\label{fig:approx:accuracy:efficiency:deep:25GB:96:hdd:de:100NN:100:nocache}
	\end{subfigure}
	\begin{subfigure}{0.165\textwidth}
		\centering
		\includegraphics[width=\textwidth]{exact_mapk_idxproc_deep_25GB_96_ng_100NN_10000_nocache}
		\scriptsize\caption{Deep25GB(ng)} 
		\label{fig:approx:accuracy:efficiency:deep:25GB:96:hdd:ng:100NN:10K:nocache}
	\end{subfigure}
	\begin{subfigure}{0.16\textwidth}
		\centering
		\includegraphics[width=\textwidth]{exact_mapk_idxproc_deep_25GB_96_de_100NN_10000_nocache}
		\scriptsize \caption{Deep25GB($\bm{\delta\epsilon}$)} 
		\label{fig:approx:accuracy:efficiency:deep:25GB:96:hdd:de:100NN:10K:nocache}
	\end{subfigure}	
	\vspace*{-0.2cm}
	\caption{{\color{black} Efficiency vs. accuracy in memory (100NN queries)}}	
	\vspace*{-0.2cm}
	\label{fig:approx:accuracy:efficiency:synthetic:25GB:inmemory:hdd}
\end{figure*}

\begin{comment}
\begin{figure*}[tb]
	\captionsetup{justification=centering}
	\captionsetup[subfigure]{justification=centering}
	\begin{subfigure}{\textwidth}
		\centering
		%\hspace*{1cm}
		\includegraphics[scale=0.23]{{full_epsilon_legend_new}}\\
	\end{subfigure}	
	\begin{subfigure}{0.16\textwidth}
		\centering
		\includegraphics[width=\textwidth]{exact_mapk_throughput_synthetic_25GB_16384_ng_100NN_100_nocache}
		\caption{100-NN (ng)} 
		\label{fig:approx:accuracy:qefficiency:synthetic:25GB:16384:ng:hdd:100NN:100:nocache}
	\end{subfigure}
	\begin{subfigure}{0.16\textwidth}
		\centering
		\includegraphics[width=\textwidth]{exact_mapk_throughput_synthetic_25GB_16384_de_100NN_100_nocache}
		\caption{100-NN ($\bm{\delta\epsilon}$)} 
		\label{fig:approx:accuracy:qefficiency:synthetic:25GB:16384:de:hdd:100NN:100:cache}
	\end{subfigure}
	\begin{subfigure}{0.16\textwidth}
		\centering
		\includegraphics[width=\textwidth]{exact_mapk_idxproc_synthetic_25GB_16384_ng_100NN_100_nocache}
		\caption{100-NN (ng)} 
		\label{fig:approx:accuracy:efficiency:synthetic:25GB:16384:ng:hdd:100NN:100:nocache}
	\end{subfigure}
	\begin{subfigure}{0.16\textwidth}
		\centering
		\includegraphics[width=\textwidth]{exact_mapk_idxproc_synthetic_25GB_16384_de_100NN_100_nocache}
		\caption{100-NN ($\bm{\delta\epsilon}$)} 
		\label{fig:approx:accuracy:efficiency:synthetic:25GB:16384:de:hdd:100NN:100:cache}
	\end{subfigure}
	\begin{subfigure}{0.16\textwidth}
		\centering
		\includegraphics[width=\textwidth]{exact_mapk_idxproc_synthetic_25GB_16384_ng_100NN_10000_nocache}
		\caption{100-NN (ng)} 
		\label{fig:approx:accuracy:efficiency:synthetic:25GB:16384:ng:hdd:100NN:10K:nocache}
	\end{subfigure}
	\begin{subfigure}{0.16\textwidth}
		\centering
		\includegraphics[width=\textwidth]{exact_mapk_idxproc_synthetic_25GB_16384_de_100NN_10000_nocache}
		\caption{100-NN ($\bm{\delta\epsilon}$)} 
		\label{fig:approx:accuracy:efficiency:synthetic:25GB:16384:de:hdd:100NN:10K:cache}
	\end{subfigure}
	\caption{Efficiency vs. accuracy\\
		(Dataset = Rand25GB, Data series length=16384)}	
	\label{fig:approx:accuracy:efficiency:synthetic:25GB:16384:hdd}
\end{figure*}
\end{comment}
%lowest accuracy possible for iSAX2+, trying fewer leaves does not return 100NN

%{\color{cyan}\noindent\textbf{Search Efficiency for Longer Series.}} 
\noindent\textbf{Long Series}. 
In this experiment, we use dataset sizes of 25GB, and query length of 16384.
% (the number of dimensions in the summarizations remains at 16). 
For $ng$-approximate search, we report the results only for iSAX2+, DSTree and VA+file. We ran several experiments with IMI and HNSW building the indexes using different parameters, but obtained a MAP of 0 for IMI for all index configurations we tried, and ran into a segmentation fault during query answering with HNSW. 
DSTree outperforms both iSAX2+ and VA+file in terms of throughput and combined total cost for the larger workload (Figures~\ref{fig:approx:accuracy:qefficiency:synthetic:25GB:16384:ng:hdd:100NN:100:nocache} and~\ref{fig:approx:accuracy:efficiency:synthetic:25GB:16384:ng:hdd:100NN:10K:nocache}), whereas iSAX2+ wins for the smaller workload when the combined total cost is considered (Figure~\ref{fig:approx:accuracy:efficiency:synthetic:25GB:16384:ng:hdd:100NN:100:nocache}). {\color{black}We note also that the performance of FLANN deteriorates with the increased dimensionality.}

For $\delta$-$\epsilon$-approximate queries, Figure~\ref{fig:approx:accuracy:qefficiency:synthetic:25GB:16384:de:hdd:100NN:100:nocache} shows that DSTree and VA+file outperform all other methods for large MAP values, while DSTree and iSAX2+ have higher throughput for small MAP values. 
Note that the SRS accuracy decreases when compared to series of length 256, with the best MAP value now being 0.25. 
This is due to the increased information loss, as for both series lengths the number of dimensions in the projected space is 16. 
When index building time is considered, VA+file wins for the small workload (Figure~\ref{fig:approx:accuracy:efficiency:synthetic:25GB:16384:de:hdd:100NN:100:nocache}), and iSAX2+ and DSTree win for the large one (Figure~\ref{fig:approx:accuracy:efficiency:synthetic:25GB:16384:de:hdd:100NN:10K:nocache}). {\color{black}We do not report numbers for QALSH because the algorithm ran into a segmentation fault for series of length 16384.}

\noindent\textbf{Real Data}. 
We ran the same set of experiments with real datasets. 
For $ng$-approximate queries, HNSW outperforms the query performance of other methods by a large margin (Figures~\ref{fig:approx:accuracy:qefficiency:sift:25GB:ng:hdd:100NN:100:nocache} and~\ref{fig:approx:accuracy:qefficiency:deep:25GB:96:hdd:ng:100NN:100:nocache}). When indexing time is considered, HNSW loses its edge due to its high indexing cost to iSAX2+ when the query workload consists of 100 queries (Figures~\ref{fig:approx:accuracy:efficiency:sift:25GB:ng:hdd:100NN:100:nocache} and~\ref{fig:approx:accuracy:efficiency:deep:25GB:96:hdd:ng:100NN:100:nocache}) and to DSTree for the 10K workload (Figures~\ref{fig:approx:accuracy:efficiency:sift:25GB:ng:hdd:100NN:10K:nocache} and ~\ref{fig:approx:accuracy:efficiency:deep:25GB:96:hdd:ng:100NN:10K:nocache}). 
HNSW does not achieve a MAP of 1, while DSTree and ISAX2+ both do,  yet at a high cost. 

DSTree clearly wins on Sift25GB and Deep25GB among $\delta$-$\epsilon$-approximate methods (Figures~\ref{fig:approx:accuracy:qefficiency:sift:25GB:de:hdd:100NN:100:nocache}, ~\ref{fig:approx:accuracy:qefficiency:deep:25GB:96:hdd:de:100NN:100:nocache}, ~\ref{fig:approx:accuracy:efficiency:sift:25GB:de:hdd:100NN:10K:nocache}, and ~\ref{fig:approx:accuracy:efficiency:deep:25GB:96:hdd:de:100NN:10K:nocache}), except for the scenario of indexing plus answering 100 queries, where iSAX2+ has the least combined cost (Figures~\ref{fig:approx:accuracy:efficiency:sift:25GB:de:hdd:100NN:100:nocache} and~\ref{fig:approx:accuracy:efficiency:deep:25GB:96:hdd:de:100NN:100:nocache}). 
This is because DSTree's query answering is very fast, but its indexing cost is high, so it is amortized only with a large query workload (Figures~\ref{fig:approx:accuracy:efficiency:sift:25GB:de:hdd:100NN:10K:nocache} and~\ref{fig:approx:accuracy:efficiency:deep:25GB:96:hdd:de:100NN:10K:nocache}). 
We observe a similar trend for both Sift25GB and Deep25GB, except the degradation of the performance of SRS, which achieves a very low accuracy of 0.01 on Deep25GB, despite using the most restrictive parameters ($\delta$ = 0.99 and $\epsilon$ = 0). 
\begin{comment}
\begin{figure*}[!htb]
	\captionsetup{justification=centering}
	\captionsetup[subfigure]{justification=centering}
	\begin{subfigure}{\textwidth}
		\centering
		%\hspace*{1cm}
		\includegraphics[scale=0.23]{{full_epsilon_legend_new}}\\
	\end{subfigure}	
	\begin{subfigure}{0.16\textwidth}
		\centering
		\includegraphics[width=\textwidth]{exact_mapk_throughput_sift_25GB_128_ng_100NN_100_nocache}
		\caption{100-NN (ng)} 
		\label{fig:approx:accuracy:qefficiency:sift:25GB:ng:hdd:100NN:100:nocache}
	\end{subfigure}
	\begin{subfigure}{0.16\textwidth}
		\centering
		\includegraphics[width=\textwidth]{exact_mapk_throughput_sift_25GB_128_de_100NN_100_nocache}
		\caption{100-NN ($\bm{\delta\epsilon}$)} 
		\label{fig:approx:accuracy:qefficiency:sift:25GB:de:hdd:100NN:100:nocache}
	\end{subfigure}
	\begin{subfigure}{0.16\textwidth}
		\centering
		\includegraphics[width=\textwidth]{exact_mapk_idxproc_sift_25GB_128_ng_100NN_100_nocache}
		\caption{100-NN (ng)} 
		\label{fig:approx:accuracy:efficiency:sift:25GB:ng:hdd:100NN:100:nocache}
	\end{subfigure}
	\begin{subfigure}{0.16\textwidth}
		\centering
		\includegraphics[width=\textwidth]{exact_mapk_idxproc_sift_25GB_128_de_100NN_100_nocache}
		\caption{100-NN ($\bm{\delta\epsilon}$)} 
		\label{fig:approx:accuracy:efficiency:sift:25GB:de:hdd:100NN:100:nocache}
	\end{subfigure}
	\begin{subfigure}{0.16\textwidth}
		\centering
		\includegraphics[width=\textwidth]{exact_mapk_idxproc_sift_25GB_128_ng_100NN_10000_nocache}
		\caption{100-NN (ng)} 
		\label{fig:approx:accuracy:efficiency:sift:25GB:ng:hdd:100NN:10K:nocache}
	\end{subfigure}
	\begin{subfigure}{0.16\textwidth}
		\centering
		\includegraphics[width=\textwidth]{exact_mapk_idxproc_sift_25GB_128_de_100NN_10000_nocache}
		\caption{100-NN ($\bm{\delta\epsilon}$)} 
		\label{fig:approx:accuracy:efficiency:sift:25GB:de:hdd:100NN:10K:nocache}
	\end{subfigure}
	\caption{Efficiency vs. accuracy\\
		(Dataset = Sift25GB, Data series length=128)}	
	\label{fig:approx:accuracy:efficiency:sift:25GB:hdd}
\end{figure*}
\end{comment}
%In fact, IMI also 
%drops from 0.5 for Rand25GB to 0.3 and 0.22 for Sift25GB and Deep25GB respectively. 
%achieves its lowest accuracy on the Deep25GB dataset. This is probably due to the fact that the Deep vectors exhibit high interdependence between features, while the two halves of the Sift vectors are independent~\cite{conf/eccv/baranchuk2018}. 
%Note that IMI achieves a better accuracy for Rand25GB than for the real datasets. In fact, the MAP value drops from 0.5 to 0.3 and 0.22 for Sift25GB and Deep25GB respectively.  

\begin{comment}
\begin{figure*}[!htb]
	\captionsetup{justification=centering}
	\captionsetup[subfigure]{justification=centering}
	\begin{subfigure}{\textwidth}
		\centering
		%\hspace*{1cm}
		\includegraphics[scale=0.23]{{full_epsilon_legend_new}}\\
	\end{subfigure}	
	\begin{subfigure}{0.16\textwidth}
		\centering
		\includegraphics[width=\textwidth]{exact_mapk_throughput_deep_25GB_96_ng_100NN_100_nocache}
		\caption{100-NN (ng)} 
		\label{fig:approx:accuracy:qefficiency:deep:25GB:96:hdd:ng:100NN:100:nocache}
	\end{subfigure}
	\begin{subfigure}{0.16\textwidth}
		\centering
		\includegraphics[width=\textwidth]{exact_mapk_throughput_deep_25GB_96_de_100NN_100_nocache}
		\caption{100-NN ($\bm{\delta\epsilon}$)} 
		\label{fig:approx:accuracy:qefficiency:deep:25GB:96:hdd:de:100NN:100:nocache}
	\end{subfigure}
	\begin{subfigure}{0.16\textwidth}
		\centering
		\includegraphics[width=\textwidth]{{exact_mapk_idxproc_deep_25GB_96_ng_100NN_100_nocache}}
		\caption{100-NN (ng)} 
		\label{fig:approx:accuracy:efficiency:deep:25GB:96:hdd:ng:100NN:100:nocache}
	\end{subfigure}
	\begin{subfigure}{0.16\textwidth}
		\centering
		\includegraphics[width=\textwidth]{exact_mapk_idxproc_deep_25GB_96_de_100NN_100_nocache}
		\caption{100-NN ($\bm{\delta\epsilon}$)} 
		\label{fig:approx:accuracy:efficiency:deep:25GB:96:hdd:de:100NN:100:nocache}
	\end{subfigure}
	\begin{subfigure}{0.16\textwidth}
		\centering
		\includegraphics[width=\textwidth]{exact_mapk_idxproc_deep_25GB_96_ng_100NN_10000_nocache}
		\caption{100-NN (ng)} 
		\label{fig:approx:accuracy:efficiency:deep:25GB:96:hdd:ng:100NN:10K:nocache}
	\end{subfigure}
	\begin{subfigure}{0.16\textwidth}
		\centering
		\includegraphics[width=\textwidth]{exact_mapk_idxproc_deep_25GB_96_de_100NN_10000_nocache}
		\caption{100-NN ($\bm{\delta\epsilon}$)} 
		\label{fig:approx:accuracy:efficiency:deep:25GB:96:hdd:de:100NN:10K:nocache}
	\end{subfigure}
	\caption{Efficiency vs. accuracy\\
		(Dataset = Deep25GB, Data series length=96)}	
	\label{fig:approx:accuracy:efficiency:deep:25GB:96:hdd}
\end{figure*}
%DSTree's accuracy cannot go lower since nprobe = 1 
\end{comment}

\begin{comment}
\begin{figure*}[!htb]
	\captionsetup{justification=centering}
	\captionsetup[subfigure]{justification=centering}
	\begin{subfigure}{\textwidth}
		\centering
		%\hspace*{1cm}
		\includegraphics[scale=0.23]{{full_epsilon_legend_new}}\\
	\end{subfigure}	
	\begin{subfigure}{0.16\textwidth}
		\centering
		\includegraphics[width=\textwidth]{{exact_mapk_throughput_synthetic_250GB_256_ng_100NN_100_nocache}}
		\caption{100NN (ng)} 
		\label{fig:approx:accuracy:qefficiency:synthetic:250GB:256:ng:hdd:100NN:100:nocache}
	\end{subfigure}
	\begin{subfigure}{0.16\textwidth}
		\centering
		\includegraphics[width=\textwidth]{{exact_mapk_throughput_synthetic_250GB_256_de_100NN_100_nocache}}
		\caption{100NN ($\bm{\delta\epsilon}$)} 
		\label{fig:approx:accuracy:qefficiency:synthetic:250GB:256:de:hdd:10NN:100:nocache}
	\end{subfigure}
	%\hspace*{\fill}
	\begin{subfigure}{0.16\textwidth}
		\centering
		\includegraphics[width=\textwidth]{{exact_mapk_idxproc_synthetic_250GB_256_ng_100NN_100_nocache}}
		\caption{100NN (ng)} 
		\label{fig:approx:accuracy:efficiency:synthetic:250GB:256:ng:hdd:100NN:100:nocache}
	\end{subfigure}
	\begin{subfigure}{0.16\textwidth}
		\centering
		\includegraphics[width=\textwidth]{{exact_mapk_idxproc_synthetic_250GB_256_de_100NN_100_nocache}}
		\caption{100NN ($\bm{\delta\epsilon}$)} 
		\label{fig:approx:accuracy:efficiency:synthetic:250GB:256:de:hdd:100NN:100:nocache}
	\end{subfigure}
	\begin{subfigure}{0.16\textwidth}
		\centering
		\includegraphics[width=\textwidth]{{exact_mapk_idxproc_synthetic_250GB_256_ng_100NN_10000_nocache}}
		\caption{100NN (ng)} 
		\label{fig:approx:accuracy:efficiency:synthetic:250GB:256:ng:hdd:10NN:10K:nocache}
	\end{subfigure}
	\begin{subfigure}{0.16\textwidth}
		\centering
		\includegraphics[width=\textwidth]{{exact_mapk_idxproc_synthetic_250GB_256_de_100NN_10000_nocache}}
		\caption{100NN ($\bm{\delta\epsilon}$)} 
		\label{fig:approx:accuracy:efficiency:synthetic:250GB:256:de:hdd:100NN:10K:nocache}
	\end{subfigure}
	\caption{Efficiency vs. accuracy\\
		(Dataset = Rand250GB, Data series length=256)}	
	\label{fig:approx:accuracy:efficiency:synthetic:250GB:256:hdd}
\end{figure*}
\end{comment}




\noindent\textbf{Comparison of Accuracy Measures.} In the approximate similarity search literature, the most commonly used accuracy measures are approximation error and recall. The approximation error evaluates how far the approximate neighbors are from the true neighbors, whereas recall assesses how many true neighbors are returned. 
In our study, we refer to the recall and approximation error of a workload as Avg\_Recall and MRE respectively. 
In addition, we use a third measure called MAP because it takes into account the order of the returned candidates and thus is more sensitive than recall. Figures~\ref{fig:approx:map:recall:sift:25GB:128:ng:hdd} and~\ref{fig:approx:map:mre:sift:25GB:128:ng:hdd} compare all three measures for the popular real dataset Sift25GB (we use the 25GB subset to include in-memory methods as well). 
We observe that for any given workload, the Avg\_Recall is equal to MAP for all methods, except for IMI. 
This is because IMI returns the short-listed candidates based on distance calculations on the compressed vectors, while the other methods further refine the candidates by sorting them based on the Euclidean distance of the query to the raw data. %original multidimensional vectors. 
Figure~\ref{fig:approx:map:mre:sift:25GB:128:ng:hdd} illustrates the relationship between MAP and MRE. 
Note that the value of the approximation error is not always indicative of the actual accuracy. 
For instance, an MRE of about 0.5 for iSAX2 sounds acceptable (some popular LSH methods only work with $\epsilon>=3$~\cite{journal/tods/tao2010,conf/sigmod/gan2012}), yet it corresponds to a very low accuracy of 0.03 as measured by MAP (Figures~\ref{fig:approx:map:mre:sift:25GB:128:ng:hdd}). 
Note that MAP can be more useful in practice, since it takes into account the actual ranks of the true neighbors returned, whereas MRE is evaluated only on the distances between the query and its neighbors.
















\subsubsection{\textbf{Query Answering Efficiency and Accuracy: on-Disk Datasets}}
\label{ssec:query_efficiency_disk}

{\color{black}We now report results (Figure~\ref{fig:approx:accuracy:efficiency:synthetic:250GB:ondisk:hdd}) for on-disk experiments, excluding the in-memory only HNSW, {QALSH and FLANN}}. %Figure~\ref{fig:approx:accuracy:efficiency:synthetic:250GB:ondisk:hdd} summarizes all experiments. % with disk-based synthetic and real datasets. 

\noindent\textbf{Synthetic Data}. DSTree and iSAX2+ outperform by far the rest of the techniques on both $ng$-approximate and $\delta$-$\epsilon$-approximate queries. iSAX2+ is particularly competitive when the total cost is considered with the smaller workload (Figures~\ref{fig:approx:accuracy:efficiency:synthetic:250GB:256:ng:hdd:100NN:100:nocache} and~\ref{fig:approx:accuracy:efficiency:synthetic:250GB:256:de:hdd:100NN:100:nocache}). The querying performance of SRS degraded on-disk due to severe swapping issues (Figure~\ref{fig:approx:accuracy:qefficiency:synthetic:250GB:256:de:hdd:10NN:100:nocache}), therefore we do not include this method in further disk-based experiments. Although IMI is much faster than both iSAX2+ and DSTree on $ng$-approximate search, its accuracy is extremely low. In fact, the best MAP accuracy achieved by IMI plummets to 0.05, whereas DSTree and iSAX2+ have much higher MAP values (Figure~\ref{fig:approx:accuracy:qefficiency:synthetic:250GB:256:ng:hdd:100NN:100:nocache}).


\noindent\textbf{Real Data}. DSTree outperforms all methods %for all types of approximate search 
on both Sift250GB and Deep250GB. The only exception is iSAX2+ having an edge when the combined indexing and search costs are considered for the smaller workload (Figures~\ref{fig:approx:accuracy:efficiency:sift:250GB:ng:hdd:100NN:100:nocache},~\ref{fig:approx:accuracy:efficiency:sift:250GB:de:hdd:100NN:100:nocache},~\ref{fig:approx:accuracy:efficiency:deep:250GB:96:hdd:ng:100NN:100:nocache} and~\ref{fig:approx:accuracy:efficiency:deep:250GB:96:hdd:de:100NN:100:nocache}) and being equally competitive on $ng$-approximate query answering (Figures~\ref{fig:approx:accuracy:qefficiency:sift:250GB:ng:hdd:100NN:100:nocache},~\ref{fig:approx:accuracy:qefficiency:sift:250GB:de:hdd:100NN:100:nocache}).
%On the Deep250GB dataset, We can observe a trend similar to the Rand250GB dataset: DSTree achieves the best throughput for ng-approximate and $\delta$-$\epsilon$-approximate search. However when index building time is considered, iSAX2+ becomes the distinct winner on the small workload, and iSAX2+/DSTree having a similar performance on the larger workload.

\begin{figure*}[tb]
	\captionsetup{justification=centering}
	\captionsetup[subfigure]{justification=centering}
	\begin{subfigure}{\textwidth}
		\centering
		%\hspace*{1cm}
		\includegraphics[scale=0.18]{{full_epsilon_legend_250GB}}\\
	\end{subfigure}	
	\begin{subfigure}{0.16\textwidth}
		\centering
		\includegraphics[width=\textwidth]{{exact_mapk_throughput_synthetic_250GB_256_ng_100NN_100_nocache}}
		\scriptsize \caption{Rand250GB(ng)} 
		\label{fig:approx:accuracy:qefficiency:synthetic:250GB:256:ng:hdd:100NN:100:nocache}
	\end{subfigure}
	\begin{subfigure}{0.16\textwidth}
		\centering
		\includegraphics[width=\textwidth]{{exact_mapk_throughput_synthetic_250GB_256_de_100NN_100_nocache}}
		\scriptsize \caption{Rand250GB($\bm{\delta\epsilon}$)} 
		\label{fig:approx:accuracy:qefficiency:synthetic:250GB:256:de:hdd:10NN:100:nocache}
	\end{subfigure}
	%\hspace*{\fill}
	\begin{subfigure}{0.16\textwidth}
		\centering
		\includegraphics[width=\textwidth]{{exact_mapk_idxproc_synthetic_250GB_256_ng_100NN_100_nocache}}
		\scriptsize \caption{Rand250GB(ng)} 
		\label{fig:approx:accuracy:efficiency:synthetic:250GB:256:ng:hdd:100NN:100:nocache}
	\end{subfigure}
	\begin{subfigure}{0.16\textwidth}
		\centering
		\includegraphics[width=\textwidth]{{exact_mapk_idxproc_synthetic_250GB_256_de_100NN_100_nocache}}
		\scriptsize \caption{Rand250GB($\bm{\delta\epsilon}$)} 
		\label{fig:approx:accuracy:efficiency:synthetic:250GB:256:de:hdd:100NN:100:nocache}
	\end{subfigure}
	\begin{subfigure}{0.16\textwidth}
		\centering
		\includegraphics[width=\textwidth]{{exact_mapk_idxproc_synthetic_250GB_256_ng_100NN_10000_nocache}}
		\scriptsize \caption{Rand250GB(ng)} 
		\label{fig:approx:accuracy:efficiency:synthetic:250GB:256:ng:hdd:10NN:10K:nocache}
	\end{subfigure}
	\begin{subfigure}{0.16\textwidth}
		\centering
		\includegraphics[width=\textwidth]{{exact_mapk_idxproc_synthetic_250GB_256_de_100NN_10000_nocache}}
		\scriptsize \caption{Rand250GB($\bm{\delta\epsilon}$)} 
		\label{fig:approx:accuracy:efficiency:synthetic:250GB:256:de:hdd:100NN:10K:nocache}
	\end{subfigure}
	\begin{subfigure}{0.16\textwidth}
		\centering
		\includegraphics[width=\textwidth]{exact_mapk_throughput_sift_250GB_128_ng_100NN_100_nocache}
		\scriptsize \caption{Sift250GB(ng)} 
		\label{fig:approx:accuracy:qefficiency:sift:250GB:ng:hdd:100NN:100:nocache}
	\end{subfigure}
	\begin{subfigure}{0.16\textwidth}
		\centering
		\includegraphics[width=\textwidth]{exact_mapk_throughput_sift_250GB_128_de_100NN_100_nocache}
		\scriptsize \caption{Sift250GB($\bm{\delta\epsilon}$)} 
		\label{fig:approx:accuracy:qefficiency:sift:250GB:de:hdd:100NN:100:nocache}
	\end{subfigure}
	\begin{subfigure}{0.16\textwidth}
		\centering
		\includegraphics[width=\textwidth]{exact_mapk_idxproc_sift_250GB_128_ng_100NN_100_nocache}
		\scriptsize \caption{Sift250GB(ng)} 
		\label{fig:approx:accuracy:efficiency:sift:250GB:ng:hdd:100NN:100:nocache}
	\end{subfigure}
	\begin{subfigure}{0.16\textwidth}
		\centering
		\includegraphics[width=\textwidth]{exact_mapk_idxproc_sift_250GB_128_de_100NN_100_nocache}
		\scriptsize \caption{Sift250GB($\bm{\delta\epsilon}$)} 
		\label{fig:approx:accuracy:efficiency:sift:250GB:de:hdd:100NN:100:nocache}
	\end{subfigure}
	\begin{subfigure}{0.16\textwidth}
		\centering
		\includegraphics[width=\textwidth]{exact_mapk_idxproc_sift_250GB_128_ng_100NN_10000_nocache}
		\scriptsize \caption{Sift250GB(ng)} 
		\label{fig:approx:accuracy:efficiency:sift:250GB:ng:hdd:100NN:10K:nocache}
	\end{subfigure}
	\begin{subfigure}{0.16\textwidth}
		\centering
		\includegraphics[width=\textwidth]{exact_mapk_idxproc_sift_250GB_128_de_100NN_10000_nocache}
		\scriptsize \caption{Sift250GB($\bm{\delta\epsilon}$)} 
		\label{fig:approx:accuracy:efficiency:sift:250GB:de:hdd:100NN:10K:nocache}
	\end{subfigure}	
	\begin{subfigure}{0.16\textwidth}
		\centering
		\includegraphics[width=\textwidth]{exact_mapk_throughput_deep_250GB_96_ng_100NN_100_nocache}
		\scriptsize \caption{Deep250GB(ng)} 
		\label{fig:approx:accuracy:qefficiency:deep:250GB:96:hdd:ng:100NN:100:nocache}
	\end{subfigure}
	\begin{subfigure}{0.16\textwidth}
		\centering
		\includegraphics[width=\textwidth]{exact_mapk_throughput_deep_250GB_96_de_100NN_100_nocache}
		\scriptsize \caption{Deep250GB($\bm{\delta\epsilon}$)} 
		\label{fig:approx:accuracy:qefficiency:deep:250GB:96:hdd:de:100NN:100:nocache}
	\end{subfigure}
	\begin{subfigure}{0.16\textwidth}
		\centering
		\includegraphics[width=\textwidth]{{exact_mapk_idxproc_deep_250GB_96_ng_100NN_100_nocache}}
		\scriptsize \caption{Deep250GB(ng)} 
		\label{fig:approx:accuracy:efficiency:deep:250GB:96:hdd:ng:100NN:100:nocache}
	\end{subfigure}
	\begin{subfigure}{0.16\textwidth}
		\centering
		\includegraphics[width=\textwidth]{exact_mapk_idxproc_deep_250GB_96_de_100NN_100_nocache}
		\scriptsize \caption{Deep250GB($\bm{\delta\epsilon}$)} 
		\label{fig:approx:accuracy:efficiency:deep:250GB:96:hdd:de:100NN:100:nocache}
	\end{subfigure}
	\begin{subfigure}{0.16\textwidth}
		\centering
		\includegraphics[width=\textwidth]{exact_mapk_idxproc_deep_250GB_96_ng_100NN_10000_nocache}
		\scriptsize \caption{Deep250GB(ng)} 
		\label{fig:approx:accuracy:efficiency:deep:250GB:96:hdd:ng:100NN:10K:nocache}
	\end{subfigure}
	\begin{subfigure}{0.16\textwidth}
		\centering
		\includegraphics[width=\textwidth]{exact_mapk_idxproc_deep_250GB_96_de_100NN_10000_nocache}
		\scriptsize \caption{Deep250GB($\bm{\delta\epsilon}$)} 
		\label{fig:approx:accuracy:efficiency:deep:250GB:96:hdd:de:100NN:10K:nocache}
	\end{subfigure}
	\caption{{\color{black} Efficiency vs. accuracy on disk (100NN queries)}}	
	\vspace*{-0.2cm}
	\label{fig:approx:accuracy:efficiency:synthetic:250GB:ondisk:hdd}
\end{figure*}
\begin{comment}
\begin{figure*}[!htb]
	\captionsetup{justification=centering}
	\captionsetup[subfigure]{justification=centering}
	\begin{subfigure}{\textwidth}
		\centering
		%\hspace*{1cm}
		\includegraphics[scale=0.23]{{full_epsilon_legend_new}}\\
	\end{subfigure}	
	\begin{subfigure}{0.16\textwidth}
		\centering
		\includegraphics[width=\textwidth]{exact_mapk_throughput_sift_250GB_128_ng_100NN_100_nocache}
		\caption{100-NN (ng)} 
		\label{fig:approx:accuracy:qefficiency:sift:250GB:ng:hdd:100NN:100:nocache}
	\end{subfigure}
	\begin{subfigure}{0.16\textwidth}
		\centering
		\includegraphics[width=\textwidth]{exact_mapk_throughput_sift_250GB_128_de_100NN_100_nocache}
		\caption{100-NN ($\bm{\delta\epsilon}$)} 
		\label{fig:approx:accuracy:qefficiency:sift:250GB:de:hdd:100NN:100:nocache}
	\end{subfigure}
	\begin{subfigure}{0.16\textwidth}
		\centering
		\includegraphics[width=\textwidth]{exact_mapk_idxproc_sift_250GB_128_ng_100NN_100_nocache}
		\caption{100-NN (ng)} 
		\label{fig:approx:accuracy:efficiency:sift:250GB:ng:hdd:100NN:100:nocache}
	\end{subfigure}
	\begin{subfigure}{0.16\textwidth}
		\centering
		\includegraphics[width=\textwidth]{exact_mapk_idxproc_sift_250GB_128_de_100NN_100_nocache}
		\caption{100-NN ($\bm{\delta\epsilon}$)} 
		\label{fig:approx:accuracy:efficiency:sift:250GB:de:hdd:100NN:100:nocache}
	\end{subfigure}
	\begin{subfigure}{0.16\textwidth}
		\centering
		\includegraphics[width=\textwidth]{exact_mapk_idxproc_sift_250GB_128_ng_100NN_10000_nocache}
		\caption{100-NN (ng)} 
		\label{fig:approx:accuracy:efficiency:sift:250GB:ng:hdd:100NN:10K:nocache}
	\end{subfigure}
	\begin{subfigure}{0.16\textwidth}
		\centering
		\includegraphics[width=\textwidth]{exact_mapk_idxproc_sift_250GB_128_de_100NN_10000_nocache}
		\caption{100-NN ($\bm{\delta\epsilon}$)} 
		\label{fig:approx:accuracy:efficiency:sift:250GB:de:hdd:100NN:10K:nocache}
	\end{subfigure}
	\caption{Efficiency vs. accuracy\\
		(Dataset = Sift250GB, Data series length=128)}	
	\label{fig:approx:accuracy:efficiency:sift:250GB:hdd}
\end{figure*}
\end{comment}

%Figure~\ref{fig:approx:accuracy:efficiency:deep:250GB:96:hdd} summarizes 

\begin{comment}
\begin{figure*}[!htb]
	\captionsetup{justification=centering}
	\captionsetup[subfigure]{justification=centering}
	\begin{subfigure}{\textwidth}
		\centering
		%\hspace*{1cm}
		\includegraphics[scale=0.23]{{full_epsilon_legend_new}}\\
	\end{subfigure}	
	\begin{subfigure}{0.16\textwidth}
		\centering
		\includegraphics[width=\textwidth]{exact_mapk_throughput_deep_250GB_96_ng_100NN_100_nocache}
		\caption{100-NN (ng)} 
		\label{fig:approx:accuracy:qefficiency:deep:250GB:96:hdd:ng:100NN:100:nocache}
	\end{subfigure}
	\begin{subfigure}{0.16\textwidth}
		\centering
		\includegraphics[width=\textwidth]{exact_mapk_throughput_deep_250GB_96_de_100NN_100_nocache}
		\caption{100-NN ($\bm{\delta\epsilon}$)} 
		\label{fig:approx:accuracy:qefficiency:deep:250GB:96:hdd:de:100NN:100:nocache}
	\end{subfigure}
	\begin{subfigure}{0.16\textwidth}
		\centering
		\includegraphics[width=\textwidth]{{exact_mapk_idxproc_deep_250GB_96_ng_100NN_100_nocache}}
		\caption{100-NN (ng)} 
		\label{fig:approx:accuracy:efficiency:deep:250GB:96:hdd:ng:100NN:100:nocache}
	\end{subfigure}
	\begin{subfigure}{0.16\textwidth}
		\centering
		\includegraphics[width=\textwidth]{exact_mapk_idxproc_deep_250GB_96_de_100NN_100_nocache}
		\caption{100-NN ($\bm{\delta\epsilon}$)} 
		\label{fig:approx:accuracy:efficiency:deep:250GB:96:hdd:de:100NN:100:nocache}
	\end{subfigure}
	\begin{subfigure}{0.16\textwidth}
		\centering
		\includegraphics[width=\textwidth]{exact_mapk_idxproc_deep_250GB_96_ng_100NN_10000_nocache}
		\caption{100-NN (ng)} 
		\label{fig:approx:accuracy:efficiency:deep:250GB:96:hdd:ng:100NN:10K:nocache}
	\end{subfigure}
	\begin{subfigure}{0.16\textwidth}
		\centering
		\includegraphics[width=\textwidth]{exact_mapk_idxproc_deep_250GB_96_de_100NN_10000_nocache}
		\caption{100-NN ($\bm{\delta\epsilon}$)} 
		\label{fig:approx:accuracy:efficiency:deep:250GB:96:hdd:de:100NN:10K:nocache}
	\end{subfigure}
	\caption{Efficiency vs. accuracy\\
		(Dataset = Deep250GB, Data series length=96)}	
	\label{fig:approx:accuracy:efficiency:deep:250GB:96:hdd}
\end{figure*}
\end{comment}


 	

{\color{black} \noindent\textbf{Best Performing Methods.}} The earlier results show that VA+file is 
%consistently 
outperformed by DSTree and iSAX2+, and that SRS and IMI have very low accuracy on the large datasets. 
We thus conduct further experiments considering only iSAX2+ and DSTree (recall that HNSW is an in-memory approach only): %: we use two additional real datasets, more comparison criteria like the amount of data accessed and the number of random I/Os, and study in more detail the effect of k, $\delta$ and $\epsilon$. 
%All these results are summarized in 
%and report the results in 
see Figures~\ref{fig:approx:accuracy:data:250GB:hdd:best},~\ref{fig:approx:efficiency:k:hdd} and~\ref{fig:approx:accuracy_efficiency:delta:epsilon:synthetic:250GB:hdd}.%We also examine the actual useful execution time.
In terms of query efficiency/accuracy tradeoff, DSTree outperforms iSAX2+ on all datasets, except for Sald100GB (Figure~\ref{fig:approx:accuracy:throughput:sald:100GB:128:hdd:ng:100NN:100:nocache:best}), and for low MAP values on Seismic100GB (Figure~\ref{fig:approx:accuracy:throughput:seismic:100GB:256:hdd:ng:100NN:100:nocache:best}). 

\begin{comment}
 \begin{figure*}[!htb]
 	\captionsetup{justification=centering}
 	\captionsetup[subfigure]{justification=centering}
 	\begin{subfigure}{\textwidth}
 		\centering
 		%\hspace*{1cm}
 		\includegraphics[scale=0.23]{{full_epsilon_legend_new}}\\
 	\end{subfigure}	
 	\begin{subfigure}{0.24\textwidth}
 		\centering
 		\includegraphics[width=\textwidth]{exact_mapk_throughput_sald_100GB_128_ng_100NN_100_nocache}
 		\caption{Sald100GB\\100-NN (ng)} 
 		\label{fig:approx:accuracy:qefficiency:sald:100GB:ng:hdd:100NN:100:nocache}
 	\end{subfigure}
 	\begin{subfigure}{0.24\textwidth}
 		\centering
 		\includegraphics[width=\textwidth]{exact_mapk_throughput_sald_100GB_128_de_100NN_100_nocache}
 		\caption{Sald100GB\\100-NN ($\bm{\epsilon}$)} 
 		\label{fig:approx:accuracy:qefficiency:sald:100GB:de:hdd:100NN:100:nocache}
 	\end{subfigure}
 	\begin{subfigure}{0.24\textwidth}
 		\centering
 		\includegraphics[width=\textwidth]{exact_mapk_throughput_seismic_100GB_256_ng_100NN_100_nocache}
 		\caption{Seismic100GB\\100-NN (ng)} 
 		\label{fig:approx:accuracy:qefficiency:seismic:100GB:ng:hdd:100NN:100:nocache}
 	\end{subfigure}
 	\begin{subfigure}{0.24\textwidth}
 		\centering
 		\includegraphics[width=\textwidth]{exact_mapk_throughput_seismic_100GB_256_de_100NN_100_nocache}
 		\caption{Seismic100GB\\ 100-NN ($\bm{\epsilon}$)} 
 		\label{fig:approx:accuracy:qefficiency:seismic:100GB:de:hdd:100NN:100:nocache}
 	\end{subfigure}	
 	\begin{subfigure}{0.16\textwidth}
 		\centering
 		%\includegraphics[width=\textwidth]{exact_mapk_throughput_astro_100GB_256_ng_100NN_100_nocache}
 		\caption{Astro100GB\\100-NN (ng)} 
 		\label{fig:approx:accuracy:qefficiency:seismic:100GB:ng:hdd:100NN:100:nocache}
 	\end{subfigure}
 	\begin{subfigure}{0.16\textwidth}
 		\centering
 		%\includegraphics[width=\textwidth]{exact_mapk_throughput_astro_100GB_256_de_100NN_100_nocache}
 		\caption{Astro100GB\\ 100-NN ($\bm{\delta\epsilon}$)} 
 		\label{fig:approx:accuracy:qefficiency:seismic:100GB:de:hdd:100NN:100:nocache}
 	\end{subfigure}	
 	\caption{Efficiency vs. accuracy for other real datasets}	
 	\label{fig:approx:accuracy:efficiency:other:100GB:hdd}
 \end{figure*}
\end{comment}
 
\begin{figure*}[tb]
\begin{minipage}{\textwidth}
	\centering
\includegraphics[scale=0.18]{{full_epsilon_legend_25GB}}
\end{minipage}
\begin{minipage}{0.22\textwidth}
		\captionsetup{justification=centering}
		\captionsetup[subfigure]{justification=centering}
		\begin{subfigure}{\textwidth}
			%\includegraphics[width=\textwidth]{{exact_mapk_recall_sift_25GB_128_ng_100NN_100_nocache}}
			\includegraphics[width=0.95\textwidth]{{exact_mapk_recall_sift_25GB_128_ng_100NN_100_nocache}}
			\caption{Recall vs. MAP}  
			\label{fig:approx:map:recall:sift:25GB:128:ng:hdd}
		\end{subfigure}
		%\begin{subfigure}{0.2\textwidth}
		%	\centering
		%		\includegraphics[width=\textwidth]{exact_mapk_recall_deep_25GB_96_ng_100NN_100_nocache}
		%	\caption{Recall vs. MAP\\
		%		(Deep25GB)} 
		%	\label{fig:approx:map:recall:deep:25GB:96:ng:hdd}
		%\end{subfigure}
		%\\
		\begin{subfigure}{\textwidth}
			%\includegraphics[width=\textwidth]{{exact_mapk_mape_sift_25GB_128_ng_100NN_100_nocache}}
			\includegraphics[width=0.95\textwidth]{{exact_mapk_mape_sift_25GB_128_ng_100NN_100_nocache}}
			\caption{MRE vs. MAP}  
			\label{fig:approx:map:mre:sift:25GB:128:ng:hdd}
		\end{subfigure}
		%\begin{subfigure}{0.2\textwidth}
		%	\centering
		%	\includegraphics[width=\textwidth]{exact_mapk_mape_deep_25GB_96_ng_100NN_100_nocache}f
		%	\caption{MRE vs. MAP\\
		%		(Deep25GB)} 
		%	\label{fig:approx:map:mre:deep:25GB:96:ng:hdd}
		%\end{subfigure}
		\caption{{\color{black} Comparison of measures (Sift25GB)}}	
		\label{fig:approx:map:recall:hdd}
\end{minipage}
\begin{minipage}{0.04\textwidth}
\end{minipage}
\begin{minipage}{0.76\textwidth}
	\captionsetup{justification=centering}
	\captionsetup[subfigure]{justification=centering}
	\begin{subfigure}{0.18\textwidth}
		\centering
		\includegraphics[width=\textwidth]{exact_mapk_throughput_synthetic_250GB_256_de_100NN_100_nocache_best}
		\caption{Rand250GB} 
		\label{fig:approx:accuracy:throughput:synthetic:250GB:256:hdd:de:100NN:100:nocache:best}
	\end{subfigure}
	\begin{subfigure}{0.18\textwidth}
		\centering
		\includegraphics[width=\textwidth]{exact_mapk_throughput_sift_250GB_128_de_100NN_100_nocache_best}
		\caption{Sift250GB} 
		\label{fig:approx:accuracy:throughput:sift:250GB:128:hdd:ng:100NN:100:nocache:best}
	\end{subfigure}
	\begin{subfigure}{0.18\textwidth}
		\centering
		\includegraphics[width=\textwidth]{exact_mapk_throughput_deep_250GB_96_de_100NN_100_nocache_best}
		\caption{Deep250GB} 
		\label{fig:approx:accuracy:throughput:deep:250GB:96:hdd:ng:100NN:100:nocache:best}
	\end{subfigure}
	\begin{subfigure}{0.18\textwidth}
		\centering
		\includegraphics[width=\textwidth]{exact_mapk_throughput_sald_100GB_128_de_100NN_100_nocache}
		\caption{Sald100GB} 
		\label{fig:approx:accuracy:throughput:sald:100GB:128:hdd:ng:100NN:100:nocache:best}
	\end{subfigure}	
	\begin{subfigure}{0.18\textwidth}
		\centering
		\includegraphics[width=\textwidth]{exact_mapk_throughput_seismic_100GB_256_de_100NN_100_nocache}
		\caption{Seismic100GB} 
		\label{fig:approx:accuracy:throughput:seismic:100GB:256:hdd:ng:100NN:100:nocache:best}
	\end{subfigure}	

	\begin{subfigure}{0.18\textwidth}
		\centering
		\includegraphics[width=\textwidth]{exact_mapk_data_synthetic_250GB_256_de_100NN_100_nocache_best}
		\caption{Rand250GB} 
		\label{fig:approx:accuracy:data:synthetic:250GB:256:hdd:de:100NN:100:nocache:best}
	\end{subfigure}
	\begin{subfigure}{0.18\textwidth}
		\centering
		\includegraphics[width=\textwidth]{exact_mapk_data_sift_250GB_128_de_100NN_100_nocache_best}
		\caption{Sift250GB} 
		\label{fig:approx:accuracy:data:sift:250GB:128:hdd:de:100NN:100:nocache:best}
	\end{subfigure}
	\begin{subfigure}{0.18\textwidth}
		\centering
		\includegraphics[width=\textwidth]{exact_mapk_data_deep_250GB_96_de_100NN_100_nocache_best}
		\caption{Deep250GB} 
		\label{fig:approx:accuracy:data:deep:250GB:96:hdd:de:100NN:100:nocache:best}
	\end{subfigure}
	\begin{subfigure}{0.18\textwidth}
		\centering
		\includegraphics[width=\textwidth]{exact_mapk_data_sald_100GB_128_de_100NN_100_nocache}
		\caption{Sald100GB} 
		\label{fig:approx:accuracy:data:sald:100GB:128:hdd:de:100NN:100:nocache:best}
	\end{subfigure}
	\begin{subfigure}{0.18\textwidth}
		\centering
		\includegraphics[width=\textwidth]{exact_mapk_data_seismic_100GB_256_de_100NN_100_nocache}
		\caption{Seismic100GB} 
		\label{fig:approx:accuracy:data:seismic:100GB:256:hdd:de:100NN:100:nocache:best}
	\end{subfigure}

	\begin{subfigure}{0.18\textwidth}
		\centering
		\includegraphics[width=\textwidth]{exact_mapk_random_synthetic_250GB_256_de_100NN_100_nocache_best}
		\caption{Rand250GB} 
		\label{fig:approx:accuracy:random:synthetic:250GB:256:hdd:de:100NN:100:nocache:best}
	\end{subfigure}
	\begin{subfigure}{0.18\textwidth}
		\centering
		\includegraphics[width=\textwidth]{exact_mapk_random_sift_250GB_128_de_100NN_100_nocache_best}
		\caption{Sift250GB} 
		\label{fig:approx:accuracy:random:sift:250GB:128:hdd:de:100NN:100:nocache:best}
	\end{subfigure}
	\begin{subfigure}{0.18\textwidth}
		\centering
		\includegraphics[width=\textwidth]{exact_mapk_random_deep_250GB_96_de_100NN_100_nocache_best}
		\caption{Deep250GB} 
		\label{fig:approx:accuracy:random:deep:250GB:96:hdd:de:100NN:100:nocache:best}
	\end{subfigure}
	\begin{subfigure}{0.18\textwidth}
		\centering
		\includegraphics[width=\textwidth]{exact_mapk_random_sald_100GB_128_de_100NN_100_nocache}
		\caption{Sald100GB} 
		\label{fig:approx:accuracy:random:sald:100GB:128:hdd:de:100NN:100:nocache:best}
	\end{subfigure}
	\begin{subfigure}{0.18\textwidth}
		\centering
		\includegraphics[width=\textwidth]{exact_mapk_random_seismic_100GB_256_de_100NN_100_nocache}
		\caption{Seismic100GB} 
		\label{fig:approx:accuracy:random:seismic:100GB:256:hdd:de:100NN:100:nocache:best}
	\end{subfigure}
	\caption{{\color{black} Efficiency vs. accuracy for the best methods ($\bm{\epsilon}$-approximate)}}
	\label{fig:approx:accuracy:data:250GB:hdd:best}
\end{minipage}
\end{figure*}

\noindent\textbf{Amount of data accessed.} 
As expected, both DSTree and iSAX2+ need to access more data as the accuracy increases. 
Nevertheless, we observe that to achieve accuracies of almost 1, both methods access close to 100\% of the data for Sift250GB  (Figure~\ref{fig:approx:accuracy:data:sift:250GB:128:hdd:de:100NN:100:nocache:best}), Deep250GB (Figure~\ref{fig:approx:accuracy:data:deep:250GB:96:hdd:de:100NN:100:nocache:best}) and Seismic100GB (Figure~\ref{fig:approx:accuracy:data:seismic:100GB:256:hdd:de:100NN:100:nocache:best}), compared to 10\% of data accessed on Sald100GB (Figure~\ref{fig:approx:accuracy:data:sald:100GB:128:hdd:de:100NN:100:nocache:best}) and Rand250GB. (Figure~\ref{fig:approx:accuracy:data:synthetic:250GB:256:hdd:de:100NN:100:nocache:best}). 
The percentage of accessed data also varies among real datasets, Deep250GB and Sift250GB requiring the most. Note that for some datasets, a MAP of 1 is achievable with minimal data access. For instance DSTree needs to access about 1\% of the data to get a MAP of 1 on Sald100GB (Figure~\ref{fig:approx:accuracy:data:sald:100GB:128:hdd:de:100NN:100:nocache:best}).

\begin{comment}
\begin{figure*}[!htb]
	\captionsetup{justification=centering}
	\captionsetup[subfigure]{justification=centering}
	\begin{subfigure}{\textwidth}
		\centering
		%\hspace*{1cm}
		\includegraphics[scale=0.23]{{full_epsilon_legend_new}}\\
	\end{subfigure}	
	\begin{subfigure}{0.19\textwidth}
		\centering
		\includegraphics[width=\textwidth]{exact_mapk_data_synthetic_250GB_256_de_100NN_100_nocache}
		\caption{Rand250GB} 
		\label{fig:approx:accuracy:data:synthetic:250GB:256:hdd:de:100NN:100:nocache}
	\end{subfigure}
	\begin{subfigure}{0.19\textwidth}
		\centering
		\includegraphics[width=\textwidth]{exact_mapk_data_sald_100GB_128_de_100NN_100_nocache}
		\caption{Sald100GB} 
		\label{fig:approx:accuracy:data:sald:100GB:128:hdd:de:100NN:100:nocache}
	\end{subfigure}
	\begin{subfigure}{0.19\textwidth}
		\centering
		\includegraphics[width=\textwidth]{exact_mapk_data_seismic_100GB_256_de_100NN_100_nocache}
		\caption{Seismic100GB} 
		\label{fig:approx:accuracy:data:seismic:100GB:256:hdd:de:100NN:100:nocache}
	\end{subfigure}
	\begin{subfigure}{0.19\textwidth}
		\centering
		\includegraphics[width=\textwidth]{exact_mapk_data_sift_250GB_128_de_100NN_100_nocache}
		\caption{Sift250GB} 
		\label{fig:approx:accuracy:data:sift:250GB:128:hdd:de:100NN:100:nocache}
	\end{subfigure}
	\begin{subfigure}{0.19\textwidth}
		\centering
		%\includegraphics[width=\textwidth]{exact_mapk_data_deep_250GB_96_de_100NN_100_nocache}
		\caption{Deep250GB} 
		\label{fig:approx:accuracy:data:deep:250GB:96:hdd:de:100NN:100:nocache}
	\end{subfigure}
	\caption{Percent of accessed data vs. accuracy for different datasets (100-NN)}	
	\label{fig:approx:accuracy:data:250GB:hdd}
\end{figure*}
\end{comment}

\noindent\textbf{Number of Random I/Os.} 
To understand the nature of the data accesses discussed above, we report the number of random I/Os in Figure~\ref{fig:approx:accuracy:data:250GB:hdd:best} (bottom row). 
Overall, iSAX2+ incurs a higher number of random I/Os for all datasets. 
This is because iSAX2+ has a larger number of leaves, with a smaller fill factor than DSTree~\cite{journal/pvldb/echihabi2018}. 
For instance, the large number of random I/Os incurred by iSAX2+ (Figure~\ref{fig:approx:accuracy:random:seismic:100GB:256:hdd:de:100NN:100:nocache:best}) is what explains the faster runtime of DSTree on the Seismic100GB dataset (Figure~\ref{fig:approx:accuracy:throughput:seismic:100GB:256:hdd:ng:100NN:100:nocache:best}), even if DSTree accesses more data than iSAX2+ for higher MAP values (Figure~\ref{fig:approx:accuracy:data:seismic:100GB:256:hdd:de:100NN:100:nocache:best}). 
The Sald100GB dataset is an exception to this trend 
as iSAX2+ outperforms DSTree on all accuracies except for MAP is 1 (Figure~\ref{fig:approx:accuracy:throughput:sald:100GB:128:hdd:ng:100NN:100:nocache:best}), because it accesses less data incuring almost the same random I/O (Figures~\ref{fig:approx:accuracy:data:sald:100GB:128:hdd:de:100NN:100:nocache:best} and~\ref{fig:approx:accuracy:random:sald:100GB:128:hdd:de:100NN:100:nocache:best}).
\begin{comment}
\begin{figure*}[!htb]
	\captionsetup{justification=centering}
	\captionsetup[subfigure]{justification=centering}
	\begin{subfigure}{\textwidth}
		\centering
		%\hspace*{1cm}
		\includegraphics[scale=0.23]{{full_epsilon_legend_new}}\\
	\end{subfigure}	
	\begin{subfigure}{0.19\textwidth}
		\centering
		%\includegraphics[width=\textwidth]{exact_mapk_random_synthetic_250GB_256_de_100NN_100_nocache}
		\caption{Rand250GB} 
		\label{fig:approx:accuracy:random:synthetic:250GB:256:hdd:de:100NN:100:nocache}
	\end{subfigure}
	\begin{subfigure}{0.19\textwidth}
		\centering
		\includegraphics[width=\textwidth]{exact_mapk_random_sald_100GB_128_de_100NN_100_nocache}
		\caption{Sald100GB} 
		\label{fig:approx:accuracy:random:sald:100GB:128:hdd:de:100NN:100:nocache}
	\end{subfigure}
	\begin{subfigure}{0.19\textwidth}
		\centering
		\includegraphics[width=\textwidth]{exact_mapk_random_seismic_100GB_256_de_100NN_100_nocache}
		\caption{Seismic100GB} 
		\label{fig:approx:accuracy:random:seismic:100GB:256:hdd:de:100NN:100:nocache}
	\end{subfigure}
	\begin{subfigure}{0.19\textwidth}
		\centering
		%\includegraphics[width=\textwidth]{exact_mapk_random_sift_250GB_128_de_100NN_100_nocache}
		\caption{Sift250GB} 
		\label{fig:approx:accuracy:random:sift:250GB:128:hdd:de:100NN:100:nocache}
	\end{subfigure}
	\begin{subfigure}{0.19\textwidth}
		\centering
		%\includegraphics[width=\textwidth]{exact_mapk_random_deep_250GB_96_de_100NN_100_nocache}
		\caption{Deep250GB} 
		\label{fig:approx:accuracy:random:deep:250GB:96:hdd:de:100NN:100:nocache}
	\end{subfigure}
	\caption{Number of random accesses  vs. accuracy (100-NN)}	
	\label{fig:approx:accuracy:random:250GB:hdd}
\end{figure*}
\end{comment}

\noindent\textbf{Effect of k.} Figure~\ref{fig:approx:efficiency:k:hdd} summarizes experiments varying k on different datasets in-memory and on-disk. 
We measure the total time required to complete a workload of 100 queries for each value of k. 
We observe that %for the Deep datasets, 
finding the first neighbor is the most costly operation, while finding the additional neighbors is much cheaper. 
%For the Rand datasets, the total time increases steadily with higher k, while methods behave differently on the iSAX2+ workloads {\bf ??? what is an isax workload? ???}, where iSAX2+ spends most of its time finding the first neighbor and DSTree's cost gradually increases as more neighbors are retrieved. {\bf ??? I don't understand the last sentence: rewrite ???}

\begin{figure}[tb]
	\captionsetup{justification=centering}
	\captionsetup[subfigure]{justification=centering}
	\begin{subfigure}{0.32\columnwidth}
		\centering
		%\includegraphics[width=\columnwidth]{{exact_k_time_synthetic_25GB_256_de_100_nocache}}
		\includegraphics[scale=0.23]{{exact_k_time_synthetic_25GB_256_de_100_nocache}}		\caption{Rand25GB}  
		\label{fig:approx:efficiency:k:hdd:rand:25GB}
	\end{subfigure}
	\begin{subfigure}{0.32\columnwidth}
		\centering
		\includegraphics[scale=0.23]{{exact_k_time_sift_25GB_128_de_100_nocache}}\\
		\caption{Sift25GB}  
		\label{fig:approx:efficiency:k:hdd:sift:25GB}
	\end{subfigure}
	\begin{subfigure}{0.32\columnwidth}
		\centering
		\includegraphics[scale=0.23]{{exact_k_time_deep_25GB_96_de_100_nocache}}
		\caption{Deep25GB}  
		\label{fig:approx:efficiency:k:hdd:deep:25GB}
	\end{subfigure}\\
	\begin{subfigure}{0.32\columnwidth}
		\centering
		\includegraphics[scale=0.23]{exact_k_time_synthetic_250GB_256_de_100_nocache}
		\caption{Rand250GB} 
		\label{fig:approx:efficiency:k:hdd:rand:250GB}
	\end{subfigure}
	\begin{subfigure}{0.32\columnwidth}
		\centering
		\includegraphics[scale=0.23]{exact_k_time_sift_250GB_128_de_100_nocache}
		\caption{Sift250GB} 
		\label{fig:approx:efficiency:k:hdd:sift:250GB}
	\end{subfigure}
	\begin{subfigure}{0.32\columnwidth}
		\centering
		\includegraphics[scale=0.23]{exact_k_time_deep_250GB_96_de_100_nocache}
		\caption{Deep250GB} 
		\label{fig:approx:efficiency:k:hdd:deep:250GB}
	\end{subfigure}
	\caption{Efficiency vs. k ($\bm{\epsilon}$-approximate)}	
	\label{fig:approx:efficiency:k:hdd}
	\vspace*{-0.5cm}
\end{figure}

\noindent\textbf{Effect of $\delta$ and $\epsilon$.} 
In Figure~\ref{fig:approx:accuracy_efficiency:delta:epsilon:synthetic:250GB:hdd}, we describe in more detail how varying $\delta$ and $\epsilon$ affects the performance of DSTree and iSAX2+. 
Figure~\ref{fig:approx:efficiency:epsilon:synthetic:250GB:hdd} shows that the throughput of both methods increases dramatically with increasing $\epsilon$. 
For example, a small value of $\epsilon = 5$ increases the throughput of iSAX2+ by two orders of magnitude, when compared to exact search ($\epsilon$ = 0). 
Moreover, note that both methods return the actual exact answers for small $\epsilon$ values, and accuracy drops only as $\epsilon$ goes beyond $2$ (Figure~\ref{fig:approx:accuracy:mapk:epsilon:synthetic:250GB:hdd}).
In addition, Figure~\ref{fig:approx:accuracy:mape:epsilon:synthetic:250GB:hdd} shows that the actual approximation error MRE is well below the user-tolerated threshold (represented by $\epsilon$), even for $\epsilon$ values well above $2$.
The above observations mean that these methods can be used in approximate mode, achieving very high throughput, while still returning answers that are exact (or very close to the exact).

As the probability $\delta$ increases, throughput stays constant and only plummets when search becomes exact ($\delta =1$ in Figure~\ref{fig:approx:efficiency:delta:synthetic:250GB:hdd}).
Similarly, accuracy also stays constant, then slightly increases (for a very high $\delta$ of 0.99), reaching 1 for exact search (Figure~\ref{fig:approx:accuracy:delta:synthetic:250GB:hdd}).
Accuracy plateaus as $\delta$ increases, because the first $ng$-approximate answer found by both algorithms is very close to the exact answer (Figures~\ref{fig:approx:accuracy:mapk:epsilon:synthetic:250GB:hdd} and~\ref{fig:approx:accuracy:mape:epsilon:synthetic:250GB:hdd}) and better than the approximation of $r_\delta$, thus the stopping condition is never triggered. 
When a high value of $\delta$ is used, the stopping condition takes effect for some queries, but the runtime is very close to that of the exact algorithm.

\begin{figure}[tb]
	\captionsetup{justification=centering}
	\captionsetup[subfigure]{justification=centering}
	\begin{subfigure}{0.30\columnwidth}
		\centering
		\includegraphics[width=\columnwidth]{{exact_epsilon_throughput_synthetic_250GB_256_de_100NN_100_nocache}}
		\caption{Time vs. $\bm{\epsilon}$ \\
			($\bm{\delta = 1}$)}  
		\label{fig:approx:efficiency:epsilon:synthetic:250GB:hdd}
	\end{subfigure}
	\begin{subfigure}{0.30\columnwidth}
		\centering
		\includegraphics[width=\columnwidth]{exact_epsilon_mapk_synthetic_250GB_256_de_100NN_100_nocache}
		\caption{\color{black} MAP vs. $\bm{\epsilon}$ \\ ($\bm{\delta = 1}$)}	
		\label{fig:approx:accuracy:mapk:epsilon:synthetic:250GB:hdd}
	\end{subfigure}
	\begin{subfigure}{0.30\columnwidth}
		\centering
		\includegraphics[width=\columnwidth]{exact_epsilon_mape_synthetic_250GB_256_de_100NN_100_nocache}
		\caption{MRE vs. $\bm{\epsilon}$ \\
			($\bm{\delta = 1}$)}  
		\label{fig:approx:accuracy:mape:epsilon:synthetic:250GB:hdd}
	\end{subfigure}\\
	\begin{subfigure}{0.30\columnwidth}
		\centering
		\includegraphics[width=\columnwidth]{{exact_delta_throughput_synthetic_250GB_256_de_100NN_100_nocache}}
		\caption{Time vs. $\bm{\delta}$ \\
			($\bm{\epsilon = 0}$)}  
		\label{fig:approx:efficiency:delta:synthetic:250GB:hdd}
	\end{subfigure}
	\begin{subfigure}{0.34\columnwidth}
		\centering
		\includegraphics[width=\columnwidth]{exact_delta_mapk_synthetic_250GB_256_de_100NN_100_nocache}
		\caption{\color{black} MAP vs. $\bm{\delta}$ \\ ($\bm{\epsilon = 0}$)}
		\label{fig:approx:accuracy:delta:synthetic:250GB:hdd}
	\end{subfigure}
	\begin{subfigure}{0.30\columnwidth}
	%\centering
	%\hspace*{1cm}
	\includegraphics[scale=0.15]{{best_epsilon_legend}}
	\end{subfigure}	
	\caption{Accuracy and efficiency vs. $\bm{\delta}$ and $\bm{\epsilon}$
		%\\ (Dataset= Rand250GB, Queries = 100-NN ($\bm{\delta\epsilon}$)
	}	
	\vspace*{-0.4cm}
	\label{fig:approx:accuracy_efficiency:delta:epsilon:synthetic:250GB:hdd}
\end{figure}

\begin{comment}
\begin{figure*}[!htb]
	\captionsetup{justification=centering}
	\captionsetup[subfigure]{justification=centering}
	\begin{subfigure}{\textwidth}
		\centering
		%\hspace*{1cm}
		\includegraphics[scale=0.23]{{best_epsilon_legend}}\\
	\end{subfigure}	
	\begin{subfigure}{0.18\textwidth}
		\centering
		\includegraphics[width=\textwidth]{{exact_epsilon_throughput_synthetic_250GB_256_de_100NN_100_nocache}}
		\caption{Efficiency vs. $\bm{\epsilon}$ \\
			($\bm{\delta = 1}$)}  
		\label{fig:approx:efficiency:epsilon:synthetic:250GB:hdd}
	\end{subfigure}
	\begin{subfigure}{0.18\textwidth}
		\centering
		\includegraphics[width=\textwidth]{exact_epsilon_mapk_synthetic_250GB_256_de_100NN_100_nocache}
		\caption{MAP@K vs. $\bm{\epsilon}$ \\
			($\bm{\delta = 1}$)}  
		\label{fig:approx:accuracy:mapk:epsilon:synthetic:250GB:hdd}
	\end{subfigure}
	\begin{subfigure}{0.18\textwidth}
		\centering
		\includegraphics[width=\textwidth]{exact_epsilon_mape_synthetic_250GB_256_de_100NN_100_nocache}
		\caption{MAPE vs. $\bm{\epsilon}$ \\
			($\bm{\delta = 1}$)}  
		\label{fig:approx:accuracy:mape:epsilon:synthetic:250GB:hdd}
	\end{subfigure}
	\begin{subfigure}{0.18\textwidth}
		\centering
		\includegraphics[width=\textwidth]{{exact_delta_throughput_synthetic_250GB_256_de_100NN_100_nocache}}
		\caption{Efficiency vs. $\bm{\delta}$ \\
			($\bm{\epsilon = 0}$)}  
		\label{fig:approx:efficiency:delta:synthetic:250GB:hdd}
	\end{subfigure}
	\begin{subfigure}{0.18\textwidth}
		\centering
		\includegraphics[width=\textwidth]{exact_delta_mapk_synthetic_250GB_256_de_100NN_100_nocache}
		\caption{Accuracy vs. $\bm{\delta}$ \\
			($\bm{\epsilon = 0}$)} 
		\label{fig:approx:accuracy:delta:synthetic:250GB:hdd}
	\end{subfigure}
	\caption{Accuracy and efficiency vs. $\bm{\delta}$ and $\bm{\epsilon}$\\ (100-NN ($\bm{\delta\epsilon}$))
		(Dataset = Rand250GB, Data series length=256)}	
	\label{fig:approx:accuracy_efficiency:delta:epsilon:synthetic:250GB:hdd}
\end{figure*}
\end{comment}

%\noindent\textbf{BSF Analysis.}

}


\begin{comment}
\begin{figure*}[!htb]
	\captionsetup{justification=centering}
	\captionsetup[subfigure]{justification=centering}
	\begin{subfigure}{\textwidth}
	\centering
	%\hspace*{1cm}
	\includegraphics[scale=0.15]{{box_plot_k_data_partial_legend}}\\
	\end{subfigure}	
	\begin{subfigure}{0.16\textwidth}
		\centering
		\includegraphics[width=\textwidth]{exact_k_data_synthetic_25GB_256_de_map1_100_nocache}
		\caption{Rand25GB} 
		\label{fig:approx:k:data:synthetic:25GB:256:hdd:de:map1:100:nocache}
	\end{subfigure}
	\begin{subfigure}{0.16\textwidth}
		\centering
		\includegraphics[width=\textwidth]{exact_k_data_sift_25GB_128_de_map1_100_nocache}
		\caption{Sift25GB} 
		\label{fig:approx:k:data:sift:25GB:128:hdd:de:map1:100:nocache}
	\end{subfigure}
	\begin{subfigure}{0.16\textwidth}
		\centering
		\includegraphics[width=\textwidth]{exact_k_data_deep_25GB_96_de_map1_100_nocache}
		\caption{Deep25GB} 
		\label{fig:approx:k:data:deep:25GB:96:hdd:de:map1:100:nocache}
	\end{subfigure}	
    \begin{subfigure}{0.16\textwidth}
		\centering
		\includegraphics[width=\textwidth]{exact_k_data_deep_25GB_96_de_map1_bis_100_nocache}
		\caption{Deep25GB-new-gt} 
		\label{fig:approx:k:data:deep:250GB:96:hdd:de:map1:100:nocache}
    \end{subfigure}
    \begin{subfigure}{0.16\textwidth}
		\centering
		\includegraphics[width=\textwidth]{exact_k_data_deep_25GB_96_de_map1_rand_100_nocache}
		\caption{Deep25GB-new-rand} 
		\label{fig:approx:k:data:deep:250GB:96:hdd:de:map1:100:nocache}
	\end{subfigure}
	\caption{Percent of accessed data vs. k for different datasets (Dataset Size = 25GB, MAP = 1)}	
	\label{fig:approx:k:data:maps:1:25GB:hdd}
\end{figure*}

\begin{figure*}[!htb]
	\captionsetup{justification=centering}
	\captionsetup[subfigure]{justification=centering}
	\begin{subfigure}{\textwidth}
		\centering
		%\hspace*{1cm}
		\includegraphics[scale=0.15]{{box_plot_k_data_partial_legend}}\\
	\end{subfigure}	
%	\begin{subfigure}{0.16\textwidth}
%		\centering
%		\includegraphics[width=\textwidth]{exact_k_data_synthetic_250GB_256_de_map1_100_nocache}
%		\caption{Rand250GB} 
%		\label{fig:approx:k:data:synthetic:250GB:256:hdd:de:map1:100:nocache}
%	\end{subfigure}
	\begin{subfigure}{0.16\textwidth}
	\centering
	\includegraphics[width=\textwidth]{exact_k_data_synthetic_250GB_256_de_map1_100_nocache}
	\caption{Rand250GB} 
	\label{fig:approx:k:data:synthetic:250GB:256:hdd:de:map1:100:nocache}
	\end{subfigure}
	\begin{subfigure}{0.16\textwidth}
		\centering
		\includegraphics[width=\textwidth]{exact_k_data_sift_250GB_128_de_map1_100_nocache}
		\caption{Sift250GB} 
		\label{fig:approx:k:data:sift:250GB:128:hdd:de:map1:100:nocache}
	\end{subfigure}
		\begin{subfigure}{0.16\textwidth}
		\centering
		\includegraphics[width=\textwidth]{exact_k_data_deep_250GB_96_de_map1_100_nocache}
		\caption{Deep250GB} 
		\label{fig:approx:k:data:deep:250GB:96:hdd:de:map1:100:nocache}
	\end{subfigure}
	\begin{subfigure}{0.16\textwidth}
%		\centering
%		\includegraphics[width=\textwidth]{exact_k_data_sift_250GB_128_de_map1_100_nocache}
%		\caption{Sift250GB} 
%		\label{fig:approx:k:data:sift:250GB:128:hdd:de:map1:100:nocache}
		\centering
		\includegraphics[width=\textwidth]{exact_k_data_deep_250GB_96_de_map1_bis_100_nocache}
		\caption{Deep250GB-new-gt} 
		\label{fig:approx:k:data:deep:250GB:96:hdd:de:map1:100:nocache}
	\end{subfigure}
    \begin{subfigure}{0.16\textwidth}
		\centering
		\includegraphics[width=\textwidth]{exact_k_data_deep_250GB_96_de_map1_rand_100_nocache}
		\caption{Deep250GB-new-rand} 
		\label{fig:approx:k:data:deep:250GB:96:hdd:de:map1:100:nocache}
	\end{subfigure}
	\caption{Percent of accessed data vs. k for different datasets (Dataset Size = 250GB, MAP = 1)}	
	\label{fig:approx:k:data:maps:1:250GB:hdd}
\end{figure*}
\end{comment}

\begin{comment}

\begin{figure*}[!htb]
	\captionsetup{justification=centering}
	\captionsetup[subfigure]{justification=centering}
	\begin{subfigure}{\textwidth}
		\centering
		%\hspace*{1cm}
		\includegraphics[scale=0.23]{{full_epsilon_legend_new}}\\
	\end{subfigure}	
	\begin{subfigure}{0.16\textwidth}
	\centering
	\includegraphics[width=\textwidth]{exact_mapk_random_synthetic_25GB_256_ng_100NN_100_nocache}
	\caption{Rand25GB\\(100-NN (ng))} 
	\label{fig:approx:accuracy:random:synthetic:25GB:256:hdd:ng:100NN:100:nocache}
	\end{subfigure}
	\begin{subfigure}{0.16\textwidth}
	\centering
	\includegraphics[width=\textwidth]{exact_mapk_random_synthetic_25GB_256_de_100NN_100_nocache}
	\caption{Rand25GB\\(100-NN ($\bm{\delta\epsilon}$))} 
	\label{fig:approx:accuracy:random:synthetic:25GB:256:hdd:de:100NN:100:nocache}
	\end{subfigure}
	\begin{subfigure}{0.16\textwidth}
		\centering
		\includegraphics[width=\textwidth]{exact_mapk_random_sift_25GB_128_ng_100NN_100_nocache}
		\caption{Sift25GB\\(100-NN (ng))} 
		\label{fig:approx:accuracy:random:sift:25GB:128:hdd:ng:100NN:100:nocache}
	\end{subfigure}
	\begin{subfigure}{0.16\textwidth}
		\centering
		\includegraphics[width=\textwidth]{exact_mapk_random_sift_25GB_128_de_100NN_100_nocache}
		\caption{Sift25GB\\(100-NN ($\bm{\delta\epsilon}$))} 
		\label{fig:approx:accuracy:random:sift:25GB:128:hdd:de:100NN:100:nocache}
	\end{subfigure}
	\begin{subfigure}{0.16\textwidth}
		\centering
		\includegraphics[width=\textwidth]{{exact_mapk_random_deep_25GB_96_ng_100NN_100_nocache}}
		\caption{Deep25GB\\(100-NN (ng))} 
		\label{fig:approx:accuracy:random:deep:25GB:96:hdd:ng:100NN:100:nocache}
	\end{subfigure}
	\begin{subfigure}{0.16\textwidth}
		\centering
		\includegraphics[width=\textwidth]{exact_mapk_random_deep_25GB_96_de_100NN_100_nocache}
		\caption{Deep25GB\\(100-NN ($\bm{\delta\epsilon}$))} 
		\label{fig:approx:accuracy:random:deep:25GB:96:hdd:de:100NN:100:nocache}
	\end{subfigure}
	\caption{Random accesses vs. accuracy}	
	\label{fig:approx:accuracy:random:25GB:hdd}
\end{figure*}

\begin{figure*}[!htb]
	\captionsetup{justification=centering}
	\captionsetup[subfigure]{justification=centering}
	\begin{subfigure}{\textwidth}
		\centering
		%\hspace*{1cm}
		\includegraphics[scale=0.23]{{full_epsilon_legend}}\\
	\end{subfigure}	
	\begin{subfigure}{0.16\textwidth}
		\centering
		\includegraphics[width=\textwidth]{{exact_epsilon_throughtput_deep_25GB_96_ng_100NN_100_nocache}}
		\caption{Efficiency vs. $\bm{\epsilon}$ \\
		($\bm{\delta = 1}$)}  
		\label{fig:approx:efficiency:epsilon:deep:250GB:hdd}
	\end{subfigure}
	\begin{subfigure}{0.16\textwidth}
		\centering
		\includegraphics[width=\textwidth]{exact_epsilon_mapk_deep_250GB_96_de_100NN_100_nocache}
		\caption{Accuracy vs. $\bm{\epsilon}$ \\
		($\bm{\delta = 1}$)}  
		\label{fig:approx:accuracy:epsilon:deep:250GB:hdd}
	\end{subfigure}
	\begin{subfigure}{0.16\textwidth}
		\centering
		\includegraphics[width=\textwidth]{{exact_epsilon_throughtput_deep_25GB_96_ng_100NN_100_nocache}}
		\caption{Efficiency vs. $\bm{\delta}$ \\
			($\bm{\epsilon = 0}$)}  
		\label{fig:approx:efficiency:epsilon:deep:250GB:hdd}
	\end{subfigure}
	\begin{subfigure}{0.16\textwidth}
		\centering
		\includegraphics[width=\textwidth]{exact_epsilon_mapk_deep_250GB_96_de_100NN_100_nocache}
		\caption{Accuracy vs. $\bm{\delta}$ \\
		 ($\bm{\epsilon = 0}$)} 
		\label{fig:approx:accuracy:epsilon:deep:250GB:hdd}
	\end{subfigure}
	\caption{Accuracy and efficiency vs. $\bm{\delta}$ and $\bm{\epsilon}$\\
			(Dataset = Deep250GB, Data series length=96)}	
		\label{fig:approx:accuracy_efficiency:epsilon:deep:250GB:hdd}
\end{figure*}

\begin{figure*}[!htb]
	\captionsetup{justification=centering}
	\captionsetup[subfigure]{justification=centering}
	\begin{subfigure}{\textwidth}
		\centering
		%\hspace*{1cm}
		\includegraphics[scale=0.23]{{full_epsilon_legend}}\\
	\end{subfigure}	
	\begin{subfigure}{0.16\textwidth}
		\centering
		\includegraphics[width=\textwidth]{{exact_epsilon_throughput_sift_250GB_128_de_100NN_100_nocache}}
		\caption{Efficiency vs. $\bm{\epsilon}$ \\
			($\bm{\delta = 1}$)}  
		\label{fig:approx:efficiency:epsilon:sift:250GB:hdd}
	\end{subfigure}
	\begin{subfigure}{0.16\textwidth}
		\centering
		\includegraphics[width=\textwidth]{exact_epsilon_mapk_sift_250GB_128_de_100NN_100_nocache}
		\caption{Accuracy vs. $\bm{\epsilon}$ \\
			($\bm{\delta = 1}$)}  
		\label{fig:approx:accuracy:epsilon:sift:250GB:hdd}
	\end{subfigure}
	\begin{subfigure}{0.16\textwidth}
		\centering
		\includegraphics[width=\textwidth]{{exact_delta_throughput_sift_250GB_128_de_100NN_100_nocache}}
		\caption{Efficiency vs. $\bm{\delta}$ \\
			($\bm{\epsilon = 0}$)}  
		\label{fig:approx:efficiency:epsilon:sift:250GB:hdd}
	\end{subfigure}
	\begin{subfigure}{0.16\textwidth}
		\centering
		\includegraphics[width=\textwidth]{exact_delta_mapk_sift_250GB_128_de_100NN_100_nocache}
		\caption{Accuracy vs. $\bm{\delta}$ \\
			($\bm{\epsilon = 0}$)} 
		\label{fig:approx:accuracy:epsilon:sift:250GB:hdd}
	\end{subfigure}
	\caption{Accuracy and efficiency vs. $\bm{\delta}$ and $\bm{\epsilon}$\\
		(Dataset = Sift250GB, Data series length=128)}	
	\label{fig:approx:accuracy_efficiency:epsilon:deep:250GB:hdd}
\end{figure*}
\end{comment}



\begin{comment}
\begin{figure*}[!htb]
	\captionsetup{justification=centering}
	\begin{subfigure}{\textwidth}
	\centering
	%\hspace*{1cm}
	\includegraphics[scale=0.23]{{full_epsilon_legend}}\\
	\end{subfigure}	

\begin{subfigure}{0.35\textwidth}
	\centering
	\includegraphics[width=\textwidth] {{incremental_cache_combined_synthetic_25GB_10NN_100_noylabel_zoom_full_exact}}
	\caption{Exact Query Answering}
	\label{fig:exact:datasize:time:idxproc:cache:combined:indexing}
\end{subfigure}
\begin{subfigure}{0.35\textwidth}
	\centering{
		\includegraphics[width=\textwidth]{{incremental_cache_combined_synthetic_25GB_10NN_100_noylabel_zoom_full_approx}}
		\caption{Approximate Query Answering}}
		\label{fig:exact:datasize:time:idxproc:cache:combined:100exact}
	\end{subfigure}
		
		\caption{Breakdown of kNN Search Cost by k\\
			(Dataset Size = 25GB, Data Series Length=256, Queries = 100 10-NN)
			%),} {\color{black} \sout{Workload = Synth-Rand}})
			%		\\(HDD, Synthetic Datasets, Data Series Length = 256)
		}	
		\label{fig:exact:datasize:time:epsilon:combined}
	\end{figure*}

\end{comment}

\begin{comment}
\begin{figure*}[!htb]
	\captionsetup{justification=centering}
	\captionsetup{justification=centering}
	\begin{subfigure}{\textwidth}
	\centering
	%\hspace*{1cm}
	\includegraphics[scale=0.23]{{full_epsilon_legend_new}}\\
	\end{subfigure}	

\begin{subfigure}{0.30\textwidth}
	\centering
	\includegraphics[width=\textwidth] {{exact_epsilon_bsf_time_ratio_synthetic_250GB_256_de_1NN_100_nocache}}
	\caption{Time to find the exact answer}
\label{fig:exact:datasize:time:idxproc:cache:combined:idx100exact}
\end{subfigure}
%\begin{subfigure}{0.30\textwidth}
%\centering
%\includegraphics[width=\textwidth] {{exact_synthetic_epsilon_bsf_snapshots_cache_combined_100}}
%\caption{Synthetic BSF Steps} 
%\label{fig:exact:datasize:time:idxproc:cache:combined:idx10Kexact}
%\end{subfigure}	
%\begin{subfigure}{0.30\textwidth}
%\centering
%\includegraphics[width=\textwidth] {{exact_deep1b_epsilon_bsf_snapshots_cache_combined_100}}
%\caption{Deep1B BSF Steps}
%\label{fig:exact:datasize:time:idxproc:cache:combined:idx10Kexact}
%\end{subfigure}	

\caption{BSF Analysis \\
(Dataset Size = 250GB, Data Series Length=256, Queries = 100 1-NN)
%),} {\color{black} \sout{Workload = Synth-Rand}})
%		\\(HDD, Synthetic Datasets, Data Series Length = 256)
}	
\label{fig:exact:datasize:time:epsilon:combined}
\end{figure*}
\{comment}


\begin{comment}
\begin{figure*}[!htb]
	\captionsetup{justification=centering}
	\captionsetup{justification=centering}
\begin{subfigure}{0.30\textwidth}
\centering
\includegraphics[width=\textwidth] {{approx_epsilon_time_proc_maxpolicy_cache_combined_100_10NN_synthetic}}
\caption{Synthetic}
\label{fig:exact:datasize:time:idxproc:cache:combined:idx10Kexact}
\end{subfigure}	

\caption{Node scheduling policy comparison \\
(Dataset Size = 250GB, Data Series Length=256, Queries = 100 10-NN)
%),} {\color{black} \sout{Workload = Synth-Rand}})
%		\\(HDD, Synthetic Datasets, Data Series Length = 256)
}	
\label{fig:exact:datasize:time:epsilon:combined}
\end{figure*}
\end{comment}


\begin{comment}
exact_datasize_time_idxproc_nocache_isax2+
\begin{figure*}[tb]

\captionsetup{justification=centering}
	\begin{subfigure}\columnwidth}
\centering
\hspace*{0.5cm}
\includegraphics[scale=0.25]{{epsilon_legend}}\\
\end{subfigure}	

\begin{subfigure}{0.24\textwidth}
	\centering
	\includegraphics[width=\textwidth] {{exact_datasize_time_exactproc_cache_combined_100_exact_eps0_noylabel}}
	\caption{\bm{\epsilon = 0 \ (exact)} }
	\label{fig:exact:datasize:time:idxproc:cache:combined:indexing}
\end{subfigure}
\begin{subfigure}{0.24\textwidth}
	\centering
	\includegraphics[width=\textwidth]{exact_datasize_time_exactproc_cache_combined_100_exact_eps1_noylabel}
	\caption{\bm{\epsilon = 1}}
	\label{fig:exact:datasize:time:idxproc:cache:combined:100exact}
\end{subfigure}
\begin{subfigure}{0.24\textwidth}
	\centering
	\includegraphics[width=\textwidth] {{exact_datasize_time_exactproc_cache_combined_100_exact_eps5_noylabel}}
	\caption{\bm{\epsilon = 5}}
	\label{fig:exact:datasize:time:idxproc:cache:combined:idx100exact}
\end{subfigure}
\begin{subfigure}{0.24\textwidth}
	\centering
	\includegraphics[width=\textwidth] {{exact_datasize_time_exactproc_cache_combined_100_exact_eps10_noylabel}}
	\caption{\bm{\epsilon = 10}}
	\label{fig:exact:datasize:time:idxproc:cache:combined:idx10Kexact}
\end{subfigure}	

\caption{Scalability comparison (Different epsilon) %),} {\color{black} \sout{Workload = Synth-Rand}})
	%		\\(HDD, Synthetic Datasets, Data Series Length = 256)
}	
\label{fig:exact:datasize:time:epsilon:combined}
\end{figure*}


\begin{figure*}[tb]
	%\hspace*{\fill}
\begin{subfigure}{0.24\textwidth}
	\centering
	\includegraphics[width=\textwidth]{exact_datasize_recall_cache_combined_100_eps1_noylabel}
	\caption{\bm{\epsilon = 1}}
	\label{fig:exact:datasize:time:idxproc:cache:combined:100exact}
\end{subfigure}
\begin{subfigure}{0.24\textwidth}
	\centering
	\includegraphics[width=\textwidth] {{exact_datasize_recall_cache_combined_100_eps5_noylabel}}
	\caption{\bm{\epsilon = 5}}
	\label{fig:exact:datasize:time:idxproc:cache:combined:idx100exact}
\end{subfigure}
\begin{subfigure}{0.24\textwidth}
	\centering
	\includegraphics[width=\textwidth] {{exact_datasize_recall_cache_combined_100_eps10_noylabel}}
	\caption{\bm{\epsilon = 10}}
	\label{fig:exact:datasize:time:idxproc:cache:combined:idx10Kexact}
\end{subfigure}	

\caption{Recall (Different epsilon) %),} {\color{black} \sout{Workload = Synth-Rand}})
	%		\\(HDD, Synthetic Datasets, Data Series Length = 256)
}		\label{fig:exact:datasize:recall:combined}
\end{figure*}


\begin{figure*}[tb]
	%\hspace*{\fill}
	\begin{subfigure}{0.24\textwidth}
		\centering
		\includegraphics[width=\textwidth]{exact_datasize_error_cache_combined_100_eps1_noylabel}
	\caption{\bm{\epsilon = 1}}
		\label{fig:exact:datasize:time:idxproc:cache:combined:100exact}
	\end{subfigure}
	\begin{subfigure}{0.24\textwidth}
		\centering
		\includegraphics[width=\textwidth] {{exact_datasize_error_cache_combined_100_eps5_noylabel}}
	\caption{\bm{\epsilon = 5}}
		\label{fig:exact:datasize:time:idxproc:cache:combined:idx100exact}
	\end{subfigure}
	\begin{subfigure}{0.24\textwidth}
		\centering
		\includegraphics[width=\textwidth] {{exact_datasize_error_cache_combined_100_eps10_noylabel}}
	\caption{\bm{\epsilon = 10}}
		\label{fig:exact:datasize:time:idxproc:cache:combined:idx10Kexact}
	\end{subfigure}	
	
	\caption{Relative Error (Different epsilon) %),} {\color{black} \sout{Workload = Synth-Rand}})
		%		\\(HDD, Synthetic Datasets, Data Series Length = 256)
	}
\label{fig:exact:datasize:error:combined}
\end{figure*}
\end{comment}


% \begin{figure*}[!htb]
% 	\captionsetup{justification=centering}
% 	\captionsetup[subfigure]{justification=centering}
% 	\begin{subfigure}{0.31\textwidth}
% 		\centering
% 		\includegraphics[scale=0.3]{exact_query_time_cache_combined_25GB_100_exact}
% 		\caption{Dataset size = 25GB}
% 		\label{fig:exact:query:time:cache:combined:25:100:exact}
% 	\end{subfigure}
% 	\begin{subfigure}{0.31\textwidth}
% 		\centering
% 		\includegraphics[scale=0.3]{exact_query_time_cache_combined_50GB_100_exact}
% 		\caption{Dataset size = 50GB}
% 		\label{fig:exact:query:time:cache:combined:50:100:exact}
% 	\end{subfigure}
% 	\begin{subfigure}{0.31\textwidth}
% 		\centering
% 		\includegraphics[scale=0.3]{exact_query_time_cache_combined_100GB_100_exact}
% 		\caption{Dataset size = 100GB}
% 		\label{fig:exact:query:time:cache:combined:100:100:exact}
% 	\end{subfigure}
% 	\begin{subfigure}{0.31\textwidth}
% 		\centering
% 		\includegraphics[scale=0.3]{exact_query_time_cache_combined_250GB_100_exact}
% 		\caption{Dataset size = 250GB}	\label{fig:exact:query:time:cache:combined:250:100:exact}
% 	\end{subfigure}
% 	\begin{subfigure}{0.31\textwidth}
% 		\centering
% 		\includegraphics[scale=0.3]{exact_query_time_cache_combined_500GB_100_exact}
% 		\caption{Dataset size = 500GB}	\label{fig:exact:query:time:cache:combined:250:100:exact}
% 	\end{subfigure}
% 	\begin{subfigure}{0.31\textwidth}
% 		\centering
% 		\includegraphics[scale=0.3]{exact_query_time_cache_combined_1TB_100_exact}
% 		\caption{Dataset size = 1TB}	\label{fig:exact:query:time:cache:combined:250:100:exact}
% 	\end{subfigure}
% 	\caption{Exact Methods Cumulative Query Cost For Synthetic Datasets \\ (Data Series Length = 256)}
% 	\label{fig:exact:query:time:cache:combined}
% \end{figure*}


% \begin{figure*}[!htb]
% 	\captionsetup{justification=centering}
% 	\captionsetup[subfigure]{justification=centering}
% 	\begin{subfigure}{0.24\textwidth}
% 		\centering
% 		\includegraphics[scale=0.22]{exact_query_time_cache_combined_25GB_100_hard_exact}
% 		\caption{Dataset size = 25GB}
% 		\label{fig:exact:query:time:cache:combined:25:100:exact}
% 	\end{subfigure}
% 	\hspace*{\fill} % separation between the subfigures
% 	\begin{subfigure}{0.24\textwidth}
% 		\centering
% 		\includegraphics[scale=0.22]{exact_query_time_cache_combined_50GB_100_hard_exact}
% 		\caption{Dataset size = 50GB}
% 		\label{fig:exact:query:time:cache:combined:25:100:exact}
% 	\end{subfigure}
% 	\hspace*{\fill} % separation between the subfigures
% 	\begin{subfigure}{0.24\textwidth}
% 		\centering
% 		\includegraphics[scale=0.22]{exact_query_time_cache_combined_100GB_100_hard_exact}
% 		\caption{Dataset size = 100GB}
% 		\label{fig:exact:query:time:cache:combined:25:100:exact}
% 	\end{subfigure}
% 	\hspace*{\fill} % separation between the subfigures
% 	\begin{subfigure}{0.24\textwidth}
% 		\centering
% 		\includegraphics[scale=0.22]{exact_query_time_cache_combined_250GB_100_hard_exact}
% 		\caption{Dataset size = 250GB}
% 		\label{fig:exact:query:time:cache:combined:250:100:exact}
% 	\end{subfigure}
% 	\caption{Exact Methods Cumulative Query Cost For Synthetic Datasets \\ (Hard Queries, Data Series Length = 256) {\color{black} {\bf DO WE KEEP THIS?}}}
% 	\label{fig:exact:hard:query:time:cache:combined}
% \end{figure*}

\begin{comment}
content..The Input and Output times measure the times for reading and writing data to disk and the CPU time measures the time it takes to peform ime (time to traverse index and compare query and target time series), and Input IO time (time to read time series from disk).
\end{comment}



% Figure \ref{fig:exact:query:time:cache:combined} {\color{black} {\bf @Themi: Which of the subfigures to keep?}}.
%  depicts the cumulative query cost for each index and dataset.
%  We can see that it takes the UCR Suite the same time to answer each query whereas indexes pay a higher cost to answer certain queries.
%  This is because it always scans the complete file.
 % This is because the UCR Suite retrieves the raw data sequentially from a single file whereas an index needs to retrieve data from several leaves before finding the exact answer.
 % For indexes, the cost of answering a given query depends on the number of leaves retrieved, the actual number of data series stored in each leaf and whether the data of this leaf is present in memory or in the caches.
 % This trend is more pronounced for the iSAX2+ and ADS+ indexes since they have a much larger number of leaves compared to the DSTree per Figure~\ref{fig:exact:datasize:leaves:combined}.
 % For the 100GB and 250GB datasets, we can see this behavior throughout the query workload.
 % Whereas for the 50GB dataset, it is only observed for the first 25 queries.
 % After answering these queries and since the dataset fits entirely in memory, data for all the leaves is still in memory, so the cost for answering subsequent queries becomes cheaper. As the dataset get larger, we can clearly see that indexes outperform sequential scan and that the DSTree is the overall winner.

\begin{comment}
\noindent{\textbf{Non normalized-data.}}
We also ran experiments on non-normalized data {\color{black} {\bf FILL IN THE BLANKS!}}
\end{comment}

\begin{comment}


\noindent\textbf{Pruning Ratio.}
We measure the pruning ratio (higher is better) for all indexes across datasets and data series lengths. For the $Synth$-$Rand$ workload on synthetic datasets, we varied the size from 25GB to 1TB and the length from 128 to 16384. We observed that the pruning ratio remained stable for each method and that overall 
ADS+ and VA+file have the best pruning ratio, followed by DSTree, iSAX2+ and SFA. We also ran experiments with a real workload ($Deep$-$Orig$), a controlled workload on the 100GB synthetic dataset ($Synth$-$Ctrl$), and controlled workloads on the real datasets ($Astro$-$Ctrl$, $Deep$-$Ctrl$, $SALD$-$Ctrl$ and $Seismic$-$Ctrl$). In the controlled workloads, we extract series from the dataset and add noise.
Figure~\ref{fig:exact:data::pruning} summarizes these results. For lack of space, we only report the pruning ratio for the real datasets (all of size 100GB) and the 100GB synthetic dataset. 
%The trend is different for the real datasets. 
The pruning ratio for $Synth$-$Rand$ is the highest for all methods. We observe that the $Synth$-$Ctrl$ workload is more varied than $Synth$-$Rand$ since it contains harder queries with lower pruning ratios. The trend remains the same with ADS+ and VA+file having the best pruning ratio overall, followed by DSTree, iSAX2+ then SFA. For real dataset workloads, ADS+ and VA+file achieve the best pruning, followed by iSAX2+, DSTree, and then SFA. 
%The pruning ratio for the SALD dataset is high for all methods whereas that of Astro is low ({\color{black}why?}). 
%For the Deep1B queries, all methods have a lower pruning ratio on Deep-Ours except SFA which behaves poorly on Deep-Orig. 
The relatively low pruning ratio for the SFA is most probably due to the large leaf size of 1,000,000. Once a leaf is retrieved, SFA accesses all series in the leaf, which reduces the pruning ratio significantly. 
VA+file has a slightly better pruning ratio than ADS+, because it performs less random and sequential I/O, thanks to its tighter lower bound. 
We note that the pruning ratio alone does not predict the performance of an index. In fact, this ratio provides a good estimate of the number of sequential operations that a method will perform, but it should be considered along with other measures like the number of random disk I/Os.
\end{comment}




{\color{black}
\section{Discussion}
\label{sec:discussion}

In the approximate NN search literature, experimental evaluations ignore the %approximate query 
answering capabilities of data series methods. % {(\color{red}\sout{with the exception of~\cite{conf/icde/Ciaccia2000}})}. 
%Our work 
This is the first %in-depth 
study that aims to fill %bridge 
this gap. % and examine the continuum between exact and approximate query answering. 
%Despite the fact that the tasks involved to ensure a fair and thorough comparison were difficult and time consuming, the deep insights gained from the study clearly demonstrate that the effort is worthwhile.
%This section summarizes the most important insights gained.
%We undertook a challenging and laborious task, where we re-implemented from scratch four algorithms: iSAX2+, SFA trie, DSTree, and VA+file, and optimized memory management problems (swapping, and out-of-memory errors) in R*-tree, M-tree, and Stepwise.
%all methods, except for tADS+, UCR Suite, and MASS.
%Choosing C/C++ provided considerable performance gains, but also required %low-level memory management optimizations.
%We believe the effort involved was well worth it since the results of our experimental evaluation emphatically demonstrate the importance of the experimental setup on the relative performance of the various methods.
%In fact, it is of paramount importance that for each method the following 6 rules are followed: a) same programming language used, b) the parameters are fine-tuned, c) the experiments are run on the same hardware and software platform, d) the type and size of the query workload is varied, e) both in-memory and out-of-memory datasets are used, and f) all relevant optimization techniques are applied.
%To further disseminate our research results, we also make available our source code and additional plots in~\cite{url/DSSeval2}.
%Thanks to this work, we are now able to discriminate the real value of each method, its strengths and weaknesses.
% Our study brings to the forefront one relatively old technique, the DSTree, which thanks to its data-adaptive summarization (and our optimized implementation) wins over competing methods in many scenarios.
%This section summarizes the lessons learned in this study.

\noindent{\bf Unexpected Results.}
%Before running experiments with the Stepwise method, we expected it to be competitive with the other approaches, particularly since it was reported to surpass sequential scan and iSAX~\cite{conf/kdd/Karras2011}.
%For some of the algorithms our experimental evaluation revealed some unexpected results. 
Some of the results are surprising:
\\
(1) \emph{Effectiveness of $\delta$.} LSH techniques (like SRS and QALSH) exploit both $\delta$ and $\epsilon$ to tune the efficiency/accuracy tradeoff. We consider that they still fall short of expectations, because for a low $\epsilon$, high values of $\delta$ still produce low MAP and low values of $\delta$ still result in slow execution (Figure~\ref{fig:approx:accuracy:efficiency:synthetic:25GB:inmemory:hdd}). 
In the case of our extended methods, using $\epsilon$ yielded excellent empirical results, but introducing the probabilistic stop condition based on $\delta$ was ineffective (Figures~\ref{fig:approx:accuracy_efficiency:delta:epsilon:synthetic:250GB:hdd}-d,\ref{fig:approx:accuracy_efficiency:delta:epsilon:synthetic:250GB:hdd}-e). 
We believe that this is due to the inaccuracy of the (histogram-based) approximation of $r_{\delta}$. 
Therefore, %interesting research directions for improving probabilistic search in exact algorithms include 
improving the approximation of $r_{\delta}$, or devising novel approaches are interesting open research directions that will further improve the efficiency of these methods.  

(2) \emph{Approximate Query Answering with Data Series Indexes Performed Better than LSH.} 
Approximate query answering with DSTree and iSAX2+ outperfom SRS and QALSH (state-of-the-art LSH-based methods) both in space and time, while supporting better theoretical guarantees. 
This surprising result opens up exciting research opportunities, that is, devising efficient disk-based techniques that support both $ng$-approximate and $\delta$-$\epsilon$-approximate search with top performance~\cite{conf/vldb/echihabi19}. 
{\color{black} Note that data series indexes developed for distributed platforms~\cite{dpisax,conf/icde/zhang2019} also have the potential of outperforming LSH techniques~\cite{conf/cikm/bahmani2012,journal/pvdlb/sundaram2013} if extended following the ideas discussed in Section~\ref{sec:problem}.}

%\begin{comment}
(3) \emph{Our results vs. the literature.}
Our results for the in-memory experiments are in-line with those reported in the literature, confirming that HNSW achieves the best accuracy/efficiency tradeoff when only query answering is considered~\cite{conf/sisap/martin17} (Figures~\ref{fig:approx:accuracy:qefficiency:synthetic:25GB:256:hdd:ng:100NN:100:nocache},~\ref{fig:approx:accuracy:qefficiency:sift:25GB:ng:hdd:100NN:100:nocache},~\ref{fig:approx:accuracy:qefficiency:deep:25GB:96:hdd:ng:100NN:100:nocache}). 
However, when indexing time is taken into account, HNSW loses its edge to iSAX2+/DSTree for both small  (Figures~\ref{fig:approx:accuracy:efficiency:synthetic:25GB:256:hdd:ng:100NN:100:nocache},~\ref{fig:approx:accuracy:efficiency:sift:25GB:ng:hdd:100NN:100:nocache},~\ref{fig:approx:accuracy:efficiency:deep:25GB:96:hdd:ng:100NN:100:nocache}) and large (Figures~\ref{fig:approx:accuracy:efficiency:synthetic:25GB:256:hdd:ng:100NN:10K:nocache},~\ref{fig:approx:accuracy:efficiency:sift:25GB:ng:hdd:100NN:10K:nocache},~\ref{fig:approx:accuracy:efficiency:deep:25GB:96:hdd:ng:100NN:10K:nocache}) query workloads.

Our results for IMI show a dramatic decrease in accuracy, in terms of MAP and Avg\_Recall for the Sift250GB and Deep250GB datasets, while high Avg\_Recall values have been reported in the literature for the full Sift1B and Deep1B datasets~\cite{conf/cvpr/yandex16,url/faiss}. 
We thoroughly investigated the reason behind such a discrepancy and ruled out the following factors: the Z-normalization of the Sift1B/Deep1B datasets, the size of the queries, and the number of NN. 
We believe that our results are different for the following reasons: 
(a) our queries return only the number of NN requested, while the smallest candidate list considered in~\cite{conf/cvpr/yandex16} is 10,000 for a 1-NN query; and (b) the results in~\cite{url/faiss} were obtained using training on a GPU with un-reported training sizes and times (we believe both were very large), while our focus was to evaluate methods on a CPU and account for training time. % in the overall index building time. 
The difference in the accuracy results is most probably due to the fact that the training samples used in~\cite{url/faiss} were larger than the recommended numbers we used (1 million/4 million for the 25GB/250GB datasets, respectively). 
We tried to support this claim by running experiments with different training sizes: 
(i) we observed that increasing the training sizes for the smaller datasets improves the accuracy (the best results are reported in this study); 
(ii) we could not run experiments on the CPU with large training sizes for the 250GB datasets, because training was very slow: we stopped the execution after 48 hours; 
(iii) we tried a GPU-enabled server for training, but ran into a documented bug\footnote{ https://github.com/facebookresearch/faiss/issues/67}. 

%arger dataseare inline with However, our results for IMI are not Different results for IMI: the benchmarks report always Recall for 1NN answer, not 100NN nor 10NN. Their autotuning is also for 1NN.

%run with all optimizations: precompute tables
%another without: no multithreading, no precomputed tables 

%What is different in our experiments:
%-queries. they use the full 10K query workload, we use 100 queries randomly generated from the 10K workload.
%-number of NNs, we use 100NN and they only use 1NN
%-normalized data
%-A more rigorous accuracy measure: for 1NN benchmarks they use 1-recall@1, 1-recall@10 and 
%1-recall@100. We use Avg\_Recall and MAP. For 1NN, MAP = Avg\_Recall =1-recall@1. 

%Our results are in-line with~\cite{journal/pami/babenko15}: they use a candidate list length T, then rerank these candidates and report R@1. Their numbers for R@1 are the same as ours.


%\end{comment}

%Explain different way of calculating error: MAP, Recall, Recall@100
%small datasets for HNSW
%GPU (large training sizes) for IMI

%Faiss: 
%DEEP1B
%Note that the accuracy numbers we obtain for IMI are in-line with %(https://www.cv-foundation.org/openaccess/content_cvpr_2016/papers/Babenko_Efficient_Indexing_of_CVPR_2016_paper.pdf)
%check fig 2 showing recall very low for small R. The recall values of 0.45 are with reranking which is not supported by the faiss implementation.

%So either we use the original version which supports reranking or try faiss-ivf.
%FAISS-IVF:
%deep25GB: OPQ32_128,IVF65536_HNSW32,PQ32 --> training takes 40 hours with training size 2555904
%and got 0.05 recall for 100NN
%-try 1NN and see what is the recall: best is 0.05
%-build another IVF index with a larger training set using maxtrain 0 for autotune
%autotune gives training size of 4194304 = 256 * 2x14 for IMI, 250GB
%autotune gives training size of 13107200 for IVF,HNSW

%-use gpu for training just to prove point and : use larger training size. not possible gpu version seg faults.
%
%less than the ones reported in the Faiss benchmarks~\cite{url/faiss} As a coarse quantizer, we tried IMI2x11 (4M centroids), IMI2x12 (16M centroids), IVF65536\_HNSW32 and IVF262144\_HNSW32. Note that for 1M and 4M centroids we trained the vocabulary on GPU before building the index, otherwise k-means is very slow. All other operations are on CPU.

%OPQ/HNSW may return -1 if not tuned properly.
%that probabilistic nearest neighbor queries with theoretically-guaranteed sub-linear time performance and a probabilistically-bounded approximation error.
\noindent{\bf Practicality of QALSH, IMI and HNSW.} 
{\color{black} Although QALSH provides better accuracy than SRS, it does so at a high cost: it needs to build a different index for each desired query accuracy. This is a serious drawback, while our extended methods offer a neat alternative since the index is built once and the desired accuracy is determined at query time.} Although LSH methods (such as SRS) provide guarantees on the accuracy of search results, they are expensive both in time and space. The $ng$-approximate methods overcome these limitations. IMI and HNSW are considered the state-of-the-art in this category, and while they deliver better speed-accuracy tradeoffs than QALSH and SRS, they suffer from two major limitations: (a) having no guarantees can lead them to return incomplete result sets, for instance retrieving only a subset of the neighbors for a k-NN query and returning null values for the others; (b) they are very difficult to tune, which hinders their practicality. 
In fact, the speed-accuracy tradeoff is not determined only at query time, but also during index building, which means that an index may need to be built many times using different parameters before finding the right speed-accuracy tradeoff. 
This means that the optimal settings may differ across datasets, and even for different dataset sizes of the same dataset. 
Moreover, if the analyst builds an index with a particular accuracy target, and then their needs change, they will have to rebuild the index from scratch and go through the same process of determining the right parameter values.

For example, we built the IMI index for the Deep250GB dataset 8 times. 
During each run that required over 42 hours, we varied the PQ encoding sizes, the number of centroids, and training sizes but still could not achieve the desired accuracy. 
Regarding HNSW, we tried three different combinations of parameters (M/efConstruction = 4/500, 16/500, 48/200) for each dataset before choosing the optimal one; each run took over 40 hours on the small  Deep25GB.
Overall, we observe that using IMI and HNSW in practice is cumbersome and time consuming. 
Developing auto-tuning methods for these techniques is both an interesting problem and a necessity.

\noindent{\bf Importance of guarantees}. 
%Establishing guarantees on search results is extremely important as it relates to data quality, which itself is directly tied to business performance, customer service and compliance~\cite{conf/icde/saha2014,journal/bdic/hoeren2018}. 
%In this study, we are particularly interested in one important dimension of data quality, which is data accuracy~\cite{journal/tkde/wang1995}. 
In the approximate search literature, accuracy has been evaluated using recall, and approximation error. 
LSH techniques are considered the state-of-the-art in approximate search with theoretically proven sublinear time performance and probabilistic guarantees on accuracy (approximation error). 
Our results indicate that using the approximate search functionality of data series techniques provides tighter bounds than LSH (since $\delta$ can be equal to 1), and a much better performance in practice, with experimental accuracy levels well above the theoretical accuracy guarantees (Figure~\ref{fig:approx:accuracy_efficiency:delta:epsilon:synthetic:250GB:hdd}c). Note that LSH techniques can only provide probabilistic answers ($\delta < 1$), whereas our extended methods can also answer exact and $\epsilon$-approximate queries ($\delta = 1)$. 
A promising research direction is to improve the existing guarantees on these new methods, or establish additional ones: (1) by adding guarantees on query time performance; or (2) by developing probabilistic or deterministic guarantees on the recall or MAP value of a result set, instead of the commonly used distance approximation error. 
Remember that recall and MAP are better indicators of accuracy, because even small approximation errors may still result in low recall/MAP values (Figure~\ref{fig:approx:map:mre:sift:25GB:128:ng:hdd}). 

\noindent{\bf Improvement of ng-approximate methods.}  
Our results indicate that $ng$-approximate query answering with exact methods offers a viable alternative to existing methods, particularly because index building is much faster and query efficiency/accuracy tradeoffs can be determined at query time.
Besides, the performance of DSTree and iSAX2+ supporting {\color{black}ng-approximate and} $\delta$-$\epsilon$-approximate search can be greatly improved by exploiting modern hardware (including SIMD vectorization, multi-cores, multi-sockets, and GPUs).

%\noindent{\bf Potential for building auto-tuning methods}. 
%Faiss IMI autotunes the search parameters to find the best recall/speed tradeoff, and recommends the best training size given an index key and dataset size. 
%However, it does so without taking hardware requirements (CPU, GPU) or time tradeoffs into consideration. 
%While a large training size can help build an index that provides better recall at query time, the index building time may be too long, or the available memory not enough. 
%The most tricky and time consuming step is building the right index with the right index key given the hardware limitations. 
%For these methods to work, they need to assist the user determine which index from the factory to use, the optimal encodings size, and the training size given the available memory, the data size/characteristics and the hardware.
%{\bf ??? check content with karima; remove this para? ???}

%\noindent{\bf Approximate Query Answering.} Approximate answers provide a viable alternative when exact solutions are not required, with many methods having a small effective error and great performance (Figure~\ref{fig:exact:data::efferror}
%A detailed evaluation of approximate methods is part of our future work.


%\noindent{\bf Recall@1 vs. Recall@K and MAP@K}

%\noindent{\bf Accuracy vs number of kNN}
%for OPQ, recall improves with higher k because the candidates returned are not ordered by distance (ids returned instead of distances).
%add two plots x=k, y=recall, x=k, y=map (fixing delta and epsilon)

%\noindent{\bf Batch vs single queries.}

%\noindent{\bf In memory vs. out-of-core methods}

\noindent{\bf Incremental approximate k-NN}. We established that, on some datasets, a kNN query incurs a much higher time cost as k increases. Therefore, a future research direction is to build $\delta$-$\epsilon$-approximate methods that support incremental search, i.e., returning the neighbors one by one as they are found. The current approaches return the k nearest neighbors all at once which impedes their interactivity. 

\noindent{\bf Progressive Query Answering.} The excellent empirical results with approximate search using exact methods paves the way for another very promising research direction: progressive query answering~\cite{DBLP:conf/edbt/GogolouTPB19}. 
New approaches can be devised to return intermediate results %for each of the k-NN neighbors 
with increasing accuracy until the exact answers are found.

\noindent{\bf Recommendations.}
Choosing the best approach to answer an approximate similarity search query depends on a variety of factors including the accuracy desired, the dataset characteristics, the size of the query workload, the presence of an existing index and the hardware. 
Figure~\ref{fig:recommendations} illustrates a decision matrix that recommends the best technique to use for answering a query workload using an existing index. 
Overall, DSTree is the best performer, with the exceptions of ng-approximate queries, where iSAX2+ also exhibits excellent performance, and of in-memory datasets, where HNSW is the overall winner. 
Accounting for index construction time as well, DSTree becomes the method of choice across the board, except for small workloads, where iSAX2+ wins.

\begin{figure}[tb]
	\captionsetup{justification=centering}
	\includegraphics[width =\columnwidth]{{100_approx_queries_decision_matrix.pdf}}
	\vspace*{-0.5cm}
	\caption{Recommendations (query answering).		
	\vspace*{-0.3cm}
		}
	\label{fig:recommendations}
\end{figure}

}
\begin{comment}
\begin{table*}[!htb]
	\scriptsize
%	\centering
%	\begin{tabular*}{\linewidth}{|*{6}{c|}} 		
	\begin{tabular*}{\columnwidth}{|*{6}{c|}} 		
		 \hhline{~~----}		 			 	    
		\multicolumn{1}{c}{}& &
		 \multicolumn{4}{c|}{Data Series Length}  & \hhline{~-----}		 			 
		\multicolumn{1}{c|}{}&\multicolumn{1}{c|}{Dataset Size} & 128 & 256 & 2048 & 16384   \\ 		
		\hhline{------}		 			 
		\multicolumn{1}{|c}{\multirow{2}{*}{{ In-Memory }}}
		\multicolumn{1}{c}{}&\multicolumn{1}{|c|}{25GB} & iSAX2+  &  iSAX2+ & ADS+  &  ADS+	\\ 	
		\hhline{~-----}		 			 		
		\multicolumn{1}{|c}{}&\multicolumn{1}{|c|}{50GB} &   iSAX2+/DSTree &  iSAX2+/DSTree & ADS+  & ADS+ \\
		\hhline{------}		 			 		
		\multicolumn{1}{|c}{\multirow{2}{*}{{ Disk-Resident }}}
l		\multicolumn{1}{c}{}&\multicolumn{1}{|c|}{100GB} &  DSTree &  DSTree & ADS+/DSTree &  ADS+\\
		\hhline{~-----}		 			 
		\multicolumn{1}{|c}{}&\multicolumn{1}{|c|}{250GB} &  &  DSTree &  &  \\
		\hhline{------}		 			 
	\end{tabular*}
	%\centering
	\caption{Recommendations}
	\label{tab:recommendations}
\end{table*}
\end{comment}



{\color {black}
\section{Conclusions}
\label{sec:conclusions}

We presented a taxonomy for data series similarity search techniques, proposed extensions of exact data series methods that can answer $\delta$-$\epsilon$-approximate queries, and conducted a thorough experimental evaluation of the state-of-the-art techniques from both the data series and vector indexing communities. 
The results reveal the pros and cons of the various methods, demonstrate the benefits and potential of approximate data series methods, and point to unexplored research directions in the approximate similarity search field. 
%Our future work includes designing a truly scalable index for massive data series that supports progressive similarity search, {\color{cyan} and studying subsequence matching. }}
%We examined which approaches are good candidates for parallelization and modern hardware optimizations after assessing their strengths and weaknesses. We also pin-pointed the approaches which better withstand the curse of dimensionality, those which scale better with large datasets and workloads and identified the SFA trie and the DSTree to be the best choices overall, with a preference towards the DSTree since it is parameter-free. 
%Our results paint a clear picture of the strengths and weaknesses of the various approaches, and indicate promising research directions.
%Part of our future work is the experimental comparison of approximate methods,  $r$-range queries and sub-sequence matching.
% $\epsilon$-approximate and $\delta$-$\epsilon$-approximate methods, $k$-NN (for k$>$1), $r$-range queries, as well data series subsequence matching. 
%Our longer-term goal will be to explore access path selection, pruning ratio prediction and cost based optimization, as well as optimizing index structures using the gathered insights.
%Our long-term goal is to build a hybrid whole matching similarity method that includes the best features of the current methods (ADS: adaptive indexing, DSTree: adaptive segmentation). 


	
	
\vspace*{0.3cm}
\noindent{\bf Acknowledgments.} Work partially supported by program Investir l'Avenir and Univ. of Paris IDEX Emergence en Recherche ANR-18-IDEX-0001, EU project NESTOR (MSCA {\#}748945), and FMJH Program PGMO in conjunction with EDF-THALES.		
%We sincerely thank all authors for generously sharing their code, and M. Linardi for his implementation of MASS~\cite{journal/dmkd/Yeh2017}. Work partially supported by EU project NESTOR (Marie Curie #748945).
	
	
	\balance
	
	% The following two commands are all you need in the
	% initial runs of your .tex file to
	% produce the bibliography for the citations in your paper.
	\bibliographystyle{abbrv}
	%\vspace*{-5pt}
	\def\thebibliography#1{
		\section*{References}
		\normalsize                  % smaller; put \normalsize after bib --dt
		%\footnotesize
		%\small
		%\scriptsize
		\list
		{[\arabic{enumi}]}
		{\settowidth\labelwidth{[#1]}
			\leftmargin\labelwidth
			\parsep 1pt                % tighter --dt
			\itemsep 0.6pt               % tighter --dt
			\advance\leftmargin\labelsep
			\usecounter{enumi}
		}
		\def\newblock{\hskip .11em plus .33em minus .07em}
		\sloppy\clubpenalty10000\widowpenalty10000
		\sfcode`\.=1000\relax
	}
	\bibliography{ref}  % vldb_sample.bib is the name of the Bibliography in this case
	% You must have a proper ".bib" file
	%  and remember to run:
	% latex bibtex latex latex
	% to resolve all references
	
\end{document}
