{\color {black}
\section{Conclusions}
\label{sec:conclusions}

We presented a taxonomy for data series similarity search techniques, proposed extensions of exact data series methods that can answer $\delta$-$\epsilon$-approximate queries, and conducted a thorough experimental evaluation of the state-of-the-art techniques from both the data series and vector indexing communities. 
The results reveal the pros and cons of the various methods, demonstrate the benefits and potential of approximate data series methods, and point to unexplored research directions in the approximate similarity search field. 
%Our future work includes designing a truly scalable index for massive data series that supports progressive similarity search, {\color{cyan} and studying subsequence matching. }}
%We examined which approaches are good candidates for parallelization and modern hardware optimizations after assessing their strengths and weaknesses. We also pin-pointed the approaches which better withstand the curse of dimensionality, those which scale better with large datasets and workloads and identified the SFA trie and the DSTree to be the best choices overall, with a preference towards the DSTree since it is parameter-free. 
%Our results paint a clear picture of the strengths and weaknesses of the various approaches, and indicate promising research directions.
%Part of our future work is the experimental comparison of approximate methods,  $r$-range queries and sub-sequence matching.
% $\epsilon$-approximate and $\delta$-$\epsilon$-approximate methods, $k$-NN (for k$>$1), $r$-range queries, as well data series subsequence matching. 
%Our longer-term goal will be to explore access path selection, pruning ratio prediction and cost based optimization, as well as optimizing index structures using the gathered insights.
%Our long-term goal is to build a hybrid whole matching similarity method that includes the best features of the current methods (ADS: adaptive indexing, DSTree: adaptive segmentation). 

